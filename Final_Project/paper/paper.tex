\documentclass[letterpaper,12pt,leqno]{article}
\usepackage{paper}
\usepackage{pdflscape}
\usepackage{float}
\bibliographystyle{bibliography}

% Enter paper title:
\hypersetup{pdftitle={Improving Rural Accessibility in Indonesia: Fuel Subsidy versus Infrastructure Development}}

% Enter permanent URL to paper
\codeavailable{https://github.com/maghfiraer/ECON7023-Metrics-II/tree/main/Final_Project}

% Enter BibTeX file with references:
\newcommand{\bib}{bibliography.bib}

% Enter PDF file with figures here:
\newcommand{\pdf}{figures.pdf}

% Custom math
\newcommand\iid{\stackrel{\mathclap{iid}}{\sim}}
\newcommand\asym{\stackrel{\mathclap{a}}{\sim}}
\newcommand\convprob{\xrightarrow{p}}
\newcommand\convdist{\xrightarrow{d}}
\newcommand{\N}{\mathbb{N}}
\newcommand{\Z}{\mathbb{Z}}
\newcommand{\E}{\text{E}}
\newcommand{\V}{\text{Var}}
\newcommand{\Av}{\text{Avar}}
\newcommand{\se}{\text{se}}
\newcommand{\corr}{\text{Corr}}
\newcommand{\cov}{\text{Cov}}
\newcommand{\norm}{\text{Normal}}
\newcommand{\indep}{\perp \!\!\! \perp}
\newcommand{\Hy}{\text{H}}

% Fill out paper:
\begin{document}
\title{Improving Rural Accessibility in Indonesia: Fuel Subsidy versus Infrastructure Development}
\author{Maghfira Ramadhani
\thanks{Maghfira Ramadhani: Ph.D. student at Georgia Institute of Technology, maghfira.ramadhani@gatech.edu}}
\date{April 2023}                       
\begin{titlepage}\maketitle

Indonesia has been subsidizing transportation costs for a long time.

\end{titlepage}\section{Introduction}\label{s:introduction}
 
\paragraph{Research question} Although Indonesia has been subsidizing fuel for a long time, high fuel prices were still observed in rural areas in the last decade since they were unable to get the fuel from the official fuel supply chain \citep{liputan_2016, jawapos_2017}. Therefore since 2016, the government initiated the One Price Fuel program to guarantee the availability of subsidized fuel in these areas. The government in collaboration with the National Oil Company started to build a new gas station in the targeted outermost and less-developed village so that they get the fuel at the same price as any other gas station. This program is expected to bring development to the village level by reducing energy costs which could improve the overall village's economic activity. On the other hand, decentralization of development to the village level has been implemented since 2014, when the central government initiate an annual fiscal transfer to the village government to improve their financial capability in bringing development to the last miles\footnote{See Village Law No. 6 of 2014}. This research measures the impact of these government programs in improving rural accessibility at the village level and exercising the efficiency of each specific program.

\paragraph{Answer to the question} In addressing the research question, this research uses unit transportation cost to measure accessibility. Specifically in rural areas, \citet{sambodo_2019} find that transportation spending dominates energy spending which could limit people's mobility and slow economic development. On the other hand, the lack of adequate and reliable infrastructure drives up the transportation cost \citep{sandee_2016}. I treat the unit transportation cost as the willingness to pay for transportation in rural areas, therefore we control for other factors affecting the demand structure to get the causal effect. I use panel data analysis to measure the impact of the two programs in improving rural accessibility. Regarding the fuel program, I obtain the list of 55 government-appointed new distributor locations from the National Oil Company (NOC). I use the village fund transferred to the village government as a proxy for village infrastructure development. From a political economy perspective, infrastructure development is managed by the government directly, while the fuel subsidy is managed through the National Oil Company (NOC) as a delivery agent \citep{ichsan_2022}. This research evaluates the efficiency of the program by comparing the reduction of transportation cost per budget spent as the benefit-cost ratio.

\paragraph{Related literature} This paper builds on the literature on rural development and fossil fuel subsidy in Indonesia. Related literature has indicated that both subsidizing fuel and inter-government transfer contributed to improving the general economic condition in rural areas \citep{sambodo_2019,ichsan_2021,hartojo_2022}. \citep{sambodo_2019} find that villages with better access to energy tend to divert away their spending more productively thus having better health outcomes. \citep{ichsan_2021} argued that the fuel program is a short-term remedy for reducing transportation costs in rural areas, and believes that infrastructure development is the sustainable way to do so. On the effect of village fund transfer, \citet{hartojo_2022} show that the transfer effectively improved rural economic growth in rural areas. This research is the first to evaluate and compare the impact of both programs in improving accessibility at the village level.

\paragraph{Outline} The rest of the paper is organized as follows. I develop the institutional context and conceptual framework for the discussion in Section \ref{s:framework}. Section \ref{s:data} describes the data and its summary statistics. Section \ref{s:result} discusses the findings and provides a robustness check and evaluates the benefit-cost of the programs. Section \ref{s:conclusion} provides a concluding remark of the discussion.

\section{Institutional Context and Conceptual framework}\label{s:framework}

In this section, I discussed the institutional context of Indonesia and build a conceptual framework for understanding the impact of how the fossil fuel program and village development can improve rural accessibility.

\subsection{Accessibility in rural area}

\paragraph{Accessibility challenge in Indonesia}
Following \citet{sandee_2016}, regarding rural accessibility in Indonesia, the challenge mainly related to intra-island connectivity ---links within individual islands--- is linking underdeveloped regions to growth centers. In the densely populated part of Java island, the city is the center of growth, the challenges of connectivity are mostly congestion-based challenges causing high-cost for mobility. In contrast, in the rural areas of Java, we can still find a village that we can only access by motorcycle or even only by foot. This challenge is somewhat similar in other main islands such as Sumatra, Kalimantan, Sulawesi, and Papua. In these other mainlands, the challenges are the existence of adequate and reliable infrastructure that drives up transportation costs. The government initiatives in attracting foreign capital and facilitating public-private partnerships in bringing a large-scale infrastructure development have been a policy priority since 2015 \citep{pwc_2016}.


\paragraph{Transportation cost as a measure of accessibility} We can treat unit transportation cost as the willingness to pay or demand for transportation. We follow previous literature on willingness to pay for rural transportation from revealed preference, the affecting factors include travel time, convenience, and trip purpose i.e. work or education.



\subsection{Fossil fuel subsidy regime}\label{ss:fuelsubsidy}

\paragraph{Political economy of fuel subsidy} Indonesia used to be a large oil exporter in the oil boom period in the 1970s and 1980s and the National Oil Company (NOC), Pertamina contributed a big part in delivering the fuel subsidy to the public \citep{ichsan_2022}. However, as the production declined, there has been increased pressure for subsidy reform. The Indonesian government has gradually decreased the amount of fuel subsidies. The decision to reduce fuel subsidies is politically hard in Indonesia as it is usually opposed by the parliament as well as protests by the general public.

Even without the removal of the fuel subsidies, people in rural areas already experienced high-priced fuel \citep{liputan_2016, jawapos_2017}. The extreme price gap indicates poor infrastructure in remote areas, including the lack of roads, pipelines, and storage facilities \citep{ichsan_2021}. To address this issue, the government introduced the One Price Fuel policy in 2016 aimed at ensuring the distribution of Solar and Premium at a single price across Indonesia, including 500 remote areas, by 2024. The government claimed that One Price Fuel would improve the affordability of subsidized fuels for people in rural areas, where prices varied from Rp 7,000 to 100,000/liter\footnote{See BBM 1 Harga Keadilan Energi Untuk Masyarakat 3T di NKRI, BPH Migas, Jarakta (2019)}. However, Pertamina's profitability was also affected by the introduction of the One Price Fuel policy\footnote{See Regulation of Minister of Energy and Mineral Resources No. 36/2016}. Without the allocation of a supporting budget from the government, the assigned companies (Pertamina and AKR Corporindo) were forced to invest in infrastructure directly. In some cases, where the cost of such infrastructure was prohibitive, the companies resorted to extremely expensive short-term measures such as transporting fuel to remote regions by air\footnote{See \href{https://www.merdeka.com/uang/pertamina-angkut-bbm-papua-gunakan-pesawat-kapasitas-4000-liter.html}{https://www.merdeka.com/uang/pertamina-angkut-bbm-papua-gunakan-pesawat-...}}. To add, only 2.6 percent of Pertamina fueling stations is fully owned and operated by Pertamina in 2019 \citep{pertamina_2020}. This implies that Pertamina needs to rely on partnerships with local entrepreneurs to have the capacity to build a new remote gas station. Thus, the selection of the candidates for the Fuel Program is not random and follows certain screening criteria set up by the government, and then the feasibility is evaluated financially by Pertamina.

\paragraph{Fuel price regime} The introduction of the One Price Fuel policy implies a single fuel price at every gas station in Indonesia. Previously the government used to set the fuel price based on its economic price which is different for each specific area. The historical price control for the Solar and Premium is shown in Figure \ref{f:1}.



\subsection{Decentralization of development}\label{ss:decentral}

Developing countries believe decentralization and local government reform are more efficient in bringing local development \citep{vazquez_2017} and providing public goods better than central government \citep{arends2020}. In 2014 the government enacted village fund transfer to implement decentralization at the village level. The allocation amount takes into account village conditions, i.e. poverty and geographic difficulty into account.

\paragraph{Village fund allocation mechanism} Describe the village fund transfer mechanism.

\section{Data}\label{s:data}

\paragraph{Data source} I obtained the Village Potential Statistics data for the years 2014 and 2018 from Indonesia's Central Bureau of Statistics complemented with village fund transfer data from the Ministry of Village Development. The original survey covers general information about the village, population and employment, housing and environment, natural disaster, education and health, sport and leisure, transportation and communication, land use, economic activity, security, local government, community empowerment, and agriculture. However, not all the variables are published due to potential inaccuracy issues. The original data consists of consecutively 82,190 and 83,931 observations of villages in rural and urban areas. The data is then cleaned to only include non-urban areas.

\paragraph{Data description} Natural disaster data reported are up to three years before the survey, but I only use the data on landfall and earthquake occurrence from the previous year as it represents the most recent physical condition of the village from the years 2014 and 2018. Geographic data available are binary variables about mainland topography, location relative to the sea, location relative to the forest, and the use of rivers for transportation. The mainland topography variable is equal to 1 if the village is located on a slope or valley and equal to 0 if it is located on a vast land. The binary variable on sea relative location is equal to 1 if the village has a border with the sea, and equal to 0 if otherwise. Similarly, the binary variable on forest relative location is equal to 1 if the village is inside or has a border with the forest, and equal to 0 if otherwise. The river transportation binary variable with the value of 1 indicates that the river is used for transportation in the village.

Regarding infrastructure status, data on education infrastructure and electrification are available. The data on education infrastructure range from the number of schools from elementary to university in the area. However, we only use the number of Junior High Schools and Senior High Schools data as people usually require daily travel for this level of education, thus will affect the demand for transportation. For elementary education, parents are more likely to send their children to the one closest to their village. While for university besides it is usually unavailable at the village level, students usually migrate and move out of their parent's house to live close to the university. I use the number of electricity customer households from the National Electricity Company (PLN).

Regarding the village's economic condition, the only available data from the survey is the number poverty letter statement released by the village government for the year. Other demographic data are not available in the survey. These data are usually available only at the district or sub-district level at the lowest administrative level. In this research, I use the number of poverty letter statements as a proxy for the economic condition of the village.

A variety of data on transportation are available including travel duration (in hours), travel distance (in kilometers), and travel cost (in thousand Rp.). All the variables are measured from the village office to the sub-district office, the district office, the other sub-district office closest to the village, and the other district office closest to the village. For this analysis, I only use the variable measure from the village office to the sub-district office.

I measure rural accessibility using the unit transportation cost (in Rp/km) of each individual village $i$ for the year $t$. I define unit transportation cost, $y_{it}$, as the transportation cost from the village $i$'s office to the sub-district office (in thousands Rp) at year $t$, $c_{it}$, divided by the distance from the village office $i$'s to the sub-district office (in km) at year $t$, $d_{it}$ as shown in equation \eqref{e:1}.
\begin{equation}
        y_{it}=c_{it}/d_{it}\label{e:1}
\end{equation}

I obtain the list of 55 government-appointed new distributor village locations that were built in 2016 and 2017 from the NOC. Figure \ref{f:3} show the detailed map of these locations. I define all the villages that are in the same sub-district as treated by the program, i.e. $D_{it}=1$ in the year 2018. Note that in the year 2014, all villages have $D_{it}=0$. For example, suppose the government in 2016 gives the order for the NOC to build a new distribution point at village $A$. Village $A$ is in the same sub-district as villages $B,C$, and $D$. Then all villages $A,B,C$, and $D$ are treated. To extend the discussion, I use two different samples at the province level and the district level. The other main explanatory variable is the village fund transfer measured in millions Rp.

The summary statistics of the variables are presented in Table \ref{t:1} and Table \ref{t:2}. As seen in both tables, the  unit transportation cost is on average lower in 2018 than in 2014. Also the village fund transfer on average increased in 2018 than in 2014.


\section{Empirical strategy}\label{s:strategy}

I follow the following model specification
\begin{equation}
    y_{it}=\alpha_i+\theta_t+\delta_1 D_{it}+\gamma_1 VF_{it}+\textbf{X}\pmb{\beta}+\varepsilon_{it},\label{e:2}
\end{equation}
where $y_{it}$ represents the unit transportation cost in village $i$ in year $t$,
$\alpha_i$ represents the village fixed effects, $\theta_t$ represents time trend or time fixed effects, $D_{it}$ and $VF_{it}$ is the main variable of interests, and $\textbf{X}$ as vector of covariates affecting demand. 

In equation \eqref{e:2} the main effect to be identified are the impact of the fuel program on the transport cost given by the coefficient $\delta_1$; and the impact of the village fund transfer on the transport cost given by the coefficient $\gamma_1$. Serial correlation should not be an issue in this model since the panel is 4 years apart. The following paragraphs discuss the strategy to estimate the above model.

\paragraph{Endogeneity} As we have discussed in subsection \ref{ss:fuelsubsidy} and \ref{ss:decentral}, there are possible endogeneity threats from Village Fund Transfer $(VF_{it})$ and Treatment $(D_{it})$. In the process of allocating village fund transfers, the central and provincial governments take into account the village conditions such as poverty and geographic difficulty. Also, in deciding which location to build a new remote gas station, the government may take into account the village conditions as well when proposing a target area. After the recommended location is proposed to the NOC, the NOC will evaluate the economic feasibility of the new location and assess whether or not an interested business partner to invest in the remote area. 

Based on the potential endogeneity problem, I need to identify whether there is a good instrument available in the dataset. For this purpose, I run several first-stage regressions to evaluate potential instrumental variables candidates for $(VF_{it})$ and $(D_{it})$ as shown in Table \ref{t:FSVF} and Table \ref{t:FSD}. From the possible combination of instruments for both variables, I decided to choose poverty statements, number of electricity customers, earthquake frequency, sea border status, and river transportation use as the instrumental variables as it gives reasonably high $F$-statistics in the first stage regressions.

If the two main variables are indeed endogenous, the Fixed Effect Instrumental Variable (FEIV) approach (similar to First Difference Instrumental Variable for $T=2$) can be applied to identify $\delta_1$ and $\gamma_1$ in equation \eqref{e:2}. However, if 

\paragraph{Panel data specification} I perform \citet{hausman78} specification test to test against the random effect specification. I test against $H_0$ that the difference in coefficients is not systematic or that the individual unobserved heterogeneity is correlated with the regressors. We can compute the Hausman statistic as \begin{equation*}
        H=(\hat{\pmb{\delta}}_{FE}-\hat{\pmb{\delta}}_{RE})'\left[\hat{\Av}(\hat{\pmb{\delta}}_{FE})-\hat{\Av}(\hat{\pmb{\delta}}_{RE})\right]^{-1}(\hat{\pmb{\delta}}_{FE}-\hat{\pmb{\delta}}_{RE}) \asym \chi_M^2 \end{equation*}
The test yields a $\chi^2$-statistics of 15.96 with $P$-value of 0.0003, thus we cannot reject the null hypothesis. This suggests using fixed effects instead of random effects for panel data analysis. This test is valid under RE1-RE3, thus we can not strictly argue with it.

\paragraph{Estimation strategy}

\section{Results}\label{s:result}
\subsection{Main findings}\label{ss:findings}

\paragraph{Findings from Panel OLS estimates}
        Panel OLS estimates indicates that \al{only the fuel program significantly affected unit transportation costs} in rural areas when using the \al{province as sample}, and \alg{both program significantly affects unit transportation cost} when using the \alg{district as sample}. 
        
\paragraph{Findings from FEIV estimates}
        FEIV estimates yield insignificant estimates. Endogeneity tests against $H_0$ that village fund transfer and fuel program are exogenous yield a $P$-value of consecutively 0.9103 and 0.5125, indicating that \al{the null hypothesis can not be rejected}. It might be \alg{better to use Panel OLS} instead.

\paragraph{Findings from Panel OLS estimates with interaction terms}
        I enrich the analysis by generating an interaction variable between programs to explore the complementary effect of both programs. The interaction term is \al{statistically insignificant}.

\subsection{Benefit-cost analysis}
Following the estimates from subsection \ref{ss:findings}, the village fund transfer reduces unit transport cost by 1 Rp/km per million Rp spent, while the fuel program reduces unit transport costs by 2,380 Rp/km per gas station built. The capital cost of building one remote gas station is reported to be around 3 billion Rp and the profit margin of sales is 195 Rp/Liter \footnote{See \href{https://www.cnbcindonesia.com/news/20201213090547-4-208715/bangun-spbu-bbm-satu-harga-ternyata-lama-balik-modalnya}{https://www.cnbcindonesia.com/news/20201213090547-4-208715/bangun-spbu-bbm-satu-harga-ternyata-lama-balik-modalnya}} and average total sales per location of 383 kL in the 2016-2017 period \footnote{The figures is obtained from Indonesian Downstream Oil and Gas Authority}.
Thus we can estimate the benefit/net cost of the fuel program as
\begin{align*}
   B/C_{Fuel}&=\left.\frac{2,321 \text{ Rp/km}}{\text{location}}\right/ \underbrace{\frac{3,000 \text{ million Rp}-\frac{195\text{ Rp}}{kL}\times 383\text{ kL}\times\frac{\text{ million Rp}}{10^6\text{ Rp}}}{\text{location}}}_{\displaystyle \text{Net Cost} = \text{Capital Cost} - \text{Sales Profit}}.\\
   &= \frac{1000 \times |\hat{\delta}_1|}{2999.925} \ \frac{\text{Rp/km}}{\text{millions Rp spent}}=0.77\ \frac{\text{Rp/km}}{\text{millions Rp spent}}.
\end{align*}
While we treat the benefit/cost of the village fund transfer as it is, $\hat{\gamma}_1$. From the point estimates, we suggest that the efficiency of the fuel program is lower than the village fund transfers. Thus, we can use the Wald test for the following linear hypothesis.
\begin{align*}
    &\Hy_0: B/C_{Fuel}-B/C_{VF}=0.33|\hat{\delta}_1|-|\hat{\gamma}_1|>0 \Leftrightarrow -0.33\hat{\delta}_1+\hat{\gamma}_1<0
\end{align*}
since we know that both $\hat{\delta}_1,\hat{\gamma}_1<0$. The Wald test yields a $t$-statistics of 1.97 which means we can not reject the null hypothesis even at 5\% significance level. Thus, we can accept with reasonable certainty that the Village fund transfer is more efficient than the fuel program in increasing rural accessibility.

\section{Conclusion}\label{s:conclusion}

This research finds out that both rural fuel distribution program and inter-government transfer significantly reduces unit transportation cost in rural areas. 

The village fund transfer successfully reduces unit transport cost by 1 Rp/km per million Rp spent, while the fuel program reduces unit transport cost by 0.77 Rp/km per million Rp spent. This indicates that the fund transfer to local government is more efficient in improving rural accessibility.

\paragraph{Further recommendation}
The limited nature of the data really limits the analysis in this paper, thus I suggest collecting more data by adding one more year of observation to be able to implement a more robust analysis on time-varying variation.

Following \citet{abadie2016}, I suggest implementing Propensity Score Matching to construct an artificial control group by matching each treated unit with a non-treated unit of similar characteristics.

\bibliography{\bib}

% Fill out appendix:
\newpage
\appendix
\begin{landscape}
\section{Tables}\label{a:table}

\begin{table}[H]
\caption{Summary statistics of variables with the province-level as sample} 
\scalebox{0.85}{\begin{tabular}{l*{2}{ccccc}}
\toprule
                &     2014&         &         &         &         &     2018&         &         &         &         \\
                &     Mean&     S.D.&      Min&      Max&     Obs.&     Mean&     S.D.&      Min&      Max&     Obs.\\
\midrule
\emph{Transportation}&         &         &         &         &         &         &         &         &         &         \\
\hspace{0.25cm} Unit transportation cost in 000s Rp./km&     3.29&    19.94&     0.00&  1000.00&    38624&     3.22&     9.36&     0.00&   800.00&    38646\\
\hspace{0.25cm} Travel Duration&     1.17&     1.44&     1.00&    99.00&    38624&     0.50&     1.41&     0.00&    60.50&    38646\\
\vspace{0.05em} \\ \emph{Natural Disaster}&         &         &         &         &         &         &         &         &         &         \\
\hspace{0.25cm} Landfall occurence average per year&     0.11&     0.52&     0.00&     9.00&    38624&     0.15&     0.62&     0.00&     9.00&    38646\\
\hspace{0.25cm} Earthquake occurence average per year&     0.04&     0.29&     0.00&     9.00&    38624&     0.23&     1.07&     0.00&     9.00&    38646\\
\vspace{0.05em} \\ \emph{Infrastructure}&         &         &         &         &         &         &         &         &         &         \\
\hspace{0.25cm} Number of PLN electricity user household&   656.51&   781.91&     0.00& 14460.00&    38624&   738.27&   887.98&     0.00& 17530.00&    38646\\
\hspace{0.25cm} Number of Junior High School&     0.53&     0.82&     0.00&    14.00&    38624&     0.58&     0.86&     0.00&    12.00&    38646\\
\hspace{0.25cm} Number of Senior High School&     0.26&     0.70&     0.00&    40.00&    38624&     0.31&     0.74&     0.00&    11.00&    38646\\
\vspace{0.05em} \\ \emph{Geographic condition}&         &         &         &         &         &         &         &         &         &         \\
\hspace{0.25cm} =1 if slope/valleys, =0 vast land&     0.25&     0.43&     0.00&     1.00&    38624&     0.22&     0.41&     0.00&     1.00&    38646\\
\hspace{0.25cm} =1 if border with sea, =0 no border with sea&     0.19&     0.39&     0.00&     1.00&    38624&     0.19&     0.39&     0.00&     1.00&    38646\\
\hspace{0.25cm} =1 if inside or border with forest, =0 outside forest&     0.31&     0.46&     0.00&     1.00&    38624&     0.28&     0.45&     0.00&     1.00&    38646\\
\hspace{0.25cm} =1 if river used for transportation, =0 otherwise&     0.09&     0.28&     0.00&     1.00&    38624&     0.08&     0.27&     0.00&     1.00&    38646\\
\vspace{0.05em} \\ \emph{Inter-government Transfer}&         &         &         &         &         &         &         &         &         &         \\
\hspace{0.25cm} Revenue from village fund transfer&   116.08&   213.26&     0.00&  7716.00&    38624&   121.07&   143.42&     0.00& 13662.00&    36630\\
\bottomrule
\end{tabular}
}
\note{The sample is all the villages in the provinces where the fuel program exists.}
\label{t:1}\end{table}


\begin{table}[H]
\caption{Summary statistics of variables with the district-level as sample} 
\scalebox{0.85}{\begin{tabular}{l*{2}{ccccc}}
\toprule
                &     2014&         &         &         &         &     2018&         &         &         &         \\
                &     Mean&     S.D.&      Min&      Max&     Obs.&     Mean&     S.D.&      Min&      Max&     Obs.\\
\midrule
\emph{Transportation}&         &         &         &         &         &         &         &         &         &         \\
\hspace{0.25cm} Unit transportation cost in 000s Rp./km&     5.14&    21.02&     0.00&  1000.00&     3407&     4.93&    12.47&     0.00&   400.00&     3411\\
\hspace{0.25cm} Travel duration (hrs)&     1.27&     1.27&     1.00&    30.00&     3407&     0.74&     2.36&     0.00&    60.50&     3411\\
\emph{Natural Disaster}&         &         &         &         &         &         &         &         &         &         \\
\hspace{0.25cm} Landfall occurence average per year&     0.07&     0.37&     0.00&     6.00&     3407&     0.10&     0.49&     0.00&     9.00&     3411\\
\hspace{0.25cm} Earthquake occurence average per year&     0.04&     0.35&     0.00&     7.00&     3407&     0.46&     1.60&     0.00&     9.00&     3411\\
\emph{Infrastructure}&         &         &         &         &         &         &         &         &         &         \\
\hspace{0.25cm} Number of PLN electricity user household&   366.92&   610.20&     0.00&  6726.00&     3407&   422.79&   651.77&     0.00&  6468.00&     3411\\
\hspace{0.25cm} Number of Junior High School&     0.54&     0.85&     0.00&     9.00&     3407&     0.61&     0.89&     0.00&    12.00&     3411\\
\hspace{0.25cm} Number of Senior High School&     0.27&     0.66&     0.00&     7.00&     3407&     0.33&     0.73&     0.00&     8.00&     3411\\
\emph{Geographic condition}&         &         &         &         &         &         &         &         &         &         \\
\hspace{0.25cm} =1 if slope/valleys, =0 vast land&     0.22&     0.42&     0.00&     1.00&     3407&     0.19&     0.39&     0.00&     1.00&     3411\\
\hspace{0.25cm} =1 if border with sea, =0 no border with sea&     0.41&     0.49&     0.00&     1.00&     3407&     0.41&     0.49&     0.00&     1.00&     3411\\
\hspace{0.25cm} =1 if inside or border with forest, =0 outside forest&     0.37&     0.48&     0.00&     1.00&     3407&     0.35&     0.48&     0.00&     1.00&     3411\\
\hspace{0.25cm} =1 if river used for transportation, =0 otherwise&     0.17&     0.37&     0.00&     1.00&     3407&     0.17&     0.38&     0.00&     1.00&     3411\\
\emph{Economic condition}&         &         &         &         &         &         &         &         &         &         \\
\hspace{0.25cm} Number of poverty statement request&    59.08&   141.93&     0.00&  4106.00&     3407&    69.00&   282.43&     0.00&  9999.00&     3411\\
\emph{Inter-government Transfer}&         &         &         &         &         &         &         &         &         &         \\
\hspace{0.25cm} Revenue from village fund transfer&   113.55&   129.92&     0.00&  1253.00&     3407&   158.93&   289.35&     0.00& 13662.00&     3172\\
\bottomrule
\end{tabular}
}
\note{The sample is all the villages in the districts where the fuel program exists.}
\label{t:2}\end{table}


\begin{table}[H]
\caption{First Stage Regression on Village Fund Transfer ($VF_{it}$)}
\scalebox{0.8}{{
\def\sym#1{\ifmmode^{#1}\else\(^{#1}\)\fi}
\begin{tabular}{l*{4}{c}}
\toprule
                    &\multicolumn{1}{c}{(1)}         &\multicolumn{1}{c}{(2)}         &\multicolumn{1}{c}{(3)}         &\multicolumn{1}{c}{(4)}         \\
\midrule
Number of poverty statement request&       0.001\sym{***}&       0.001\sym{**} &       0.001\sym{**} &       0.001\sym{**} \\
                    &     (0.000)         &     (0.000)         &     (0.000)         &     (0.000)         \\
\addlinespace
Number of PLN electricity user household&                     &       0.009\sym{***}&       0.009\sym{***}&       0.015\sym{***}\\
                    &                     &     (0.001)         &     (0.001)         &     (0.001)         \\
\addlinespace
Earthquake frequency [y-1]&                     &                     &       0.357         &       1.348\sym{***}\\
                    &                     &                     &     (0.452)         &     (0.435)         \\
\addlinespace
=1 if slope/valleys, =0 vast land&                     &                     &                     &      -7.731\sym{***}\\
                    &                     &                     &                     &     (1.381)         \\
\addlinespace
=1 if inside or border with forest, =0 outside forest&                     &                     &                     &      13.153\sym{***}\\
                    &                     &                     &                     &     (1.441)         \\
\addlinespace
=1 if border with sea, =0 no border with sea&                     &                     &                     &      27.160\sym{***}\\
                    &                     &                     &                     &     (1.824)         \\
\addlinespace
=1 if river used for transportation, =0 otherwise&                     &                     &                     &      65.724\sym{***}\\
                    &                     &                     &                     &     (3.883)         \\
\addlinespace
Constant            &     120.111\sym{***}&     114.038\sym{***}&     113.977\sym{***}&      97.267\sym{***}\\
                    &     (0.625)         &     (0.863)         &     (0.881)         &     (1.204)         \\
\midrule
Observations        &       87842         &       87842         &       87842         &       87842         \\
\(R^{2}\)           &       0.000         &       0.002         &       0.002         &       0.017         \\
Adjusted \(R^{2}\)  &       0.000         &       0.002         &       0.002         &       0.017         \\
F                   &       8.047         &      55.568         &      37.051         &     134.336         \\
p\_value             &                     &                     &                     &                     \\
\bottomrule
\multicolumn{5}{l}{\footnotesize Standard errors in parentheses}\\
\multicolumn{5}{l}{\footnotesize \sym{*} \(p<0.10\), \sym{**} \(p<0.05\), \sym{***} \(p<0.01\)}\\
\end{tabular}
}
}    
\label{t:FSVF}\end{table}


\begin{table}[H]
\caption{First Stage Regression on Treatment ($D_{it}$)}
\scalebox{0.8}{{
\def\sym#1{\ifmmode^{#1}\else\(^{#1}\)\fi}
\begin{tabular}{l*{5}{c}}
\toprule
                    &\multicolumn{1}{c}{(1)}         &\multicolumn{1}{c}{(2)}         &\multicolumn{1}{c}{(3)}         &\multicolumn{1}{c}{(4)}         &\multicolumn{1}{c}{(5)}         \\
\midrule
\hspace{0.25cm} Number of poverty statement request&      -0.000\sym{***}&       0.000         &      -0.000         &      -0.000         &      -0.000         \\
                    &     (0.000)         &     (0.000)         &     (0.000)         &     (0.000)         &     (0.000)         \\
\addlinespace
\hspace{0.25cm} Number of PLN electricity user household&                     &      -0.000\sym{***}&      -0.000\sym{***}&      -0.000\sym{***}&      -0.000\sym{***}\\
                    &                     &     (0.000)         &     (0.000)         &     (0.000)         &     (0.000)         \\
\addlinespace
\hspace{0.25cm} Earthquake occurence average per year&                     &                     &       0.002\sym{***}&       0.002\sym{***}&       0.002\sym{***}\\
                    &                     &                     &     (0.001)         &     (0.001)         &     (0.001)         \\
\addlinespace
\hspace{0.25cm} =1 if slope/valleys, =0 vast land&                     &                     &                     &      -0.001\sym{**} &                     \\
                    &                     &                     &                     &     (0.001)         &                     \\
\addlinespace
\hspace{0.25cm} =1 if inside or border with forest, =0 outside forest&                     &                     &                     &       0.000         &                     \\
                    &                     &                     &                     &     (0.001)         &                     \\
\addlinespace
\hspace{0.25cm} =1 if border with sea, =0 no border with sea&                     &                     &                     &       0.011\sym{***}&       0.011\sym{***}\\
                    &                     &                     &                     &     (0.001)         &     (0.001)         \\
\addlinespace
\hspace{0.25cm} =1 if river used for transportation, =0 otherwise&                     &                     &                     &       0.004\sym{***}&       0.004\sym{***}\\
                    &                     &                     &                     &     (0.001)         &     (0.001)         \\
\addlinespace
Constant            &       0.004\sym{***}&       0.006\sym{***}&       0.006\sym{***}&       0.003\sym{***}&       0.003\sym{***}\\
                    &     (0.000)         &     (0.000)         &     (0.000)         &     (0.000)         &     (0.000)         \\
\midrule
Observations        &       87826         &       87826         &       87826         &       87826         &       87826         \\
\(R^{2}\)           &       0.000         &       0.001         &       0.001         &       0.006         &       0.006         \\
Adjusted \(R^{2}\)  &       0.000         &       0.001         &       0.001         &       0.006         &       0.006         \\
F                   &      12.311         &     102.613         &      69.105         &      39.775         &      55.156         \\
\bottomrule
\multicolumn{6}{l}{\footnotesize Standard errors in parentheses}\\
\multicolumn{6}{l}{\footnotesize \sym{*} \(p<0.10\), \sym{**} \(p<0.05\), \sym{***} \(p<0.01\)}\\
\end{tabular}
}
}    
\label{t:FSD}\end{table}
\end{landscape}

\begin{table}[H]
\caption{Panel OLS estimates} 
\scalebox{0.8}{{
\def\sym#1{\ifmmode^{#1}\else\(^{#1}\)\fi}
\makebox[\linewidth][c]{\begin{tabular}{l*{8}{D{.}{.}{-1}}}
\toprule
                &\multicolumn{1}{c}{(1)}         &\multicolumn{1}{c}{(2)}         &\multicolumn{1}{c}{(3)}         &\multicolumn{1}{c}{(4)}         &\multicolumn{1}{c}{(5)}         &\multicolumn{1}{c}{(6)}         &\multicolumn{1}{c}{(7)}         &\multicolumn{1}{c}{(8)}         \\
\midrule
Treatment       &   -2.260\sym{**} &   -2.191\sym{*}  &   -2.244\sym{**} &   -2.173\sym{*}  &   -2.159\sym{*}  &  \marktopleft{a1} -2.380\sym{**} &   -2.134\sym{*}  &   -2.430\sym{**} \\
                &  (1.142)         &  (1.146)         &  (1.142)         &  (1.147)         &  (1.147)         &  (1.184)         &  (1.141)         &  (1.178)         \\
\addlinespace
Village Fund transfer&    0.000         &    0.000         &    0.000         &    0.000         &   -0.001\sym{*}  &   -0.001\sym{*}  &   -0.001\sym{*}  &   -0.001\sym{*}  \\
                &  (0.000)         &  (0.000)         &  (0.000)         &  (0.000)         &  (0.001)         &  (0.001)         &  (0.001)         &  (0.001)         \\
\midrule
Observations    &    75254         &    75254         &    75254         &    75254         &     6579         &     6579         &     6579         &     6579         \\
\(R^{2}\)       &    0.000         &    0.000         &    0.000         &    0.000         &    0.007         &    0.007         &    0.003         &    0.003         \\
Adjusted \(R^{2}\)&    0.000         &    0.000         &    0.000         &    0.000         &    0.006         &    0.006         &    0.002         &    0.002         \\
Sample          & Province         & Province         & Province         & Province         & District         & District         & District         & District         \\
Controls        &      Yes         &      Yes         &       No         &       No         &      Yes         &      Yes         &       No         &       No         \\
Time Fixed Effects&       No         &      Yes         &       No         &      Yes         &       No         &      Yes         &       No         &      Yes         \\
Village Fixed Effects&      Yes         &      Yes         &      Yes         &      Yes         &      Yes         &      Yes   \markbottomright{a1}      &      Yes         &      Yes         \\
\bottomrule
\multicolumn{9}{l}{\footnotesize Standard errors in parentheses}\\
\multicolumn{9}{l}{\footnotesize \sym{*} \(p<0.10\), \sym{**} \(p<0.05\), \sym{***} \(p<0.01\)}\\
\end{tabular}
}}
}
\label{t:pols}\end{table}

\begin{table}[H]
\caption{FEIV estimates with $D_{it}$ and $VF_{it}$ as endogenous variables}
\scalebox{0.8}{{
\def\sym#1{\ifmmode^{#1}\else\(^{#1}\)\fi}
\begin{tabular}{l*{8}{D{.}{.}{-1}}}
\toprule
                &\multicolumn{1}{c}{(1)}         &\multicolumn{1}{c}{(2)}         &\multicolumn{1}{c}{(3)}         &\multicolumn{1}{c}{(4)}         &\multicolumn{1}{c}{(5)}         &\multicolumn{1}{c}{(6)}         &\multicolumn{1}{c}{(7)}         &\multicolumn{1}{c}{(8)}         \\
\midrule
Treatment       &   18.297         & -132.775         &   16.617         & -135.896         &    3.825         &  -18.464         &    5.232         &  -19.610         \\
                & (14.523)         &(198.007)         & (13.596)         &(206.572)         &  (6.469)         & (28.313)         &  (6.040)         & (25.312)         \\
\addlinespace
Village Fund transfer&    0.003         &    0.017         &    0.003         &    0.018         &   -0.005         &   -0.008         &   -0.005         &   -0.008         \\
                &  (0.003)         &  (0.018)         &  (0.003)         &  (0.019)         &  (0.007)         &  (0.010)         &  (0.006)         &  (0.009)         \\
\addlinespace
y18             &                  &    1.104         &                  &    1.100         &                  &    2.210         &                  &    2.438         \\
                &                  &  (1.809)         &                  &  (1.838)         &                  &  (3.170)         &                  &  (2.837)         \\
\midrule
Observations    &    73228         &    73228         &    73228         &    73228         &     6336         &     6336         &     6336         &     6336         \\
Sample          & Province         & Province         & Province         & Province         & District         & District         & District         & District         \\
Controls        &      Yes         &      Yes         &       No         &       No         &      Yes         &      Yes         &       No         &       No         \\
Time Fixed Effects&       No         &      Yes         &       No         &      Yes         &       No         &      Yes         &       No         &      Yes         \\
Village Fixed Effects&      Yes         &      Yes         &      Yes         &      Yes         &      Yes         &      Yes         &      Yes         &      Yes         \\
\bottomrule
\end{tabular}
}
}    
\label{t:FEIV}
\end{table}

\begin{table}[H]
\caption{FEIV estimates with $VF_{it}$ as endogenous variable}
\scalebox{0.8}{{
\def\sym#1{\ifmmode^{#1}\else\(^{#1}\)\fi}
\begin{tabular}{l*{8}{D{.}{.}{-1}}}
\toprule
                &\multicolumn{1}{c}{(1)}         &\multicolumn{1}{c}{(2)}         &\multicolumn{1}{c}{(3)}         &\multicolumn{1}{c}{(4)}         &\multicolumn{1}{c}{(5)}         &\multicolumn{1}{c}{(6)}         &\multicolumn{1}{c}{(7)}         &\multicolumn{1}{c}{(8)}         \\
\midrule
Village Fund transfer&    0.005\sym{*}  &    0.004\sym{*}  &    0.005\sym{*}  &    0.004\sym{*}  &   -0.001         &   -0.004         &    0.000         &   -0.004         \\
                &  (0.003)         &  (0.003)         &  (0.003)         &  (0.003)         &  (0.004)         &  (0.006)         &  (0.004)         &  (0.006)         \\
\addlinespace
Treatment       &   -2.722\sym{**} &   -2.611\sym{**} &   -2.740\sym{**} &   -2.593\sym{**} &   -2.211\sym{*}  &   -2.207\sym{*}  &   -2.279\sym{*}  &   -2.259\sym{*}  \\
                &  (1.194)         &  (1.192)         &  (1.198)         &  (1.193)         &  (1.287)         &  (1.287)         &  (1.274)         &  (1.274)         \\
\addlinespace
y18             &                  &   -0.081         &                  &   -0.082         &                  &    0.323         &                  &    0.398         \\
                &                  &  (0.111)         &                  &  (0.113)         &                  &  (0.433)         &                  &  (0.415)         \\
\midrule
Observations    &    73228         &    73228         &    73228         &    73228         &     6336         &     6336         &     6336         &     6336         \\
Sample          & Province         & Province         & Province         & Province         & District         & District         & District         & District         \\
Controls        &      Yes         &      Yes         &       No         &       No         &      Yes         &      Yes         &       No         &       No         \\
Time Fixed Effects&       No         &      Yes         &       No         &      Yes         &       No         &      Yes         &       No         &      Yes         \\
Village Fixed Effects&      Yes         &      Yes         &      Yes         &      Yes         &      Yes         &      Yes         &      Yes         &      Yes         \\
\bottomrule
\multicolumn{9}{l}{\footnotesize Standard errors in parentheses}\\
\multicolumn{9}{l}{\footnotesize \sym{*} \(p<0.10\), \sym{**} \(p<0.05\), \sym{***} \(p<0.01\)}\\
\end{tabular}
}
}    
\label{t:FEIV1}
\end{table}

\begin{table}[H]
\caption{FEIV estimates with $D_{it}$ as endogenous variable}
\scalebox{0.8}{{
\def\sym#1{\ifmmode^{#1}\else\(^{#1}\)\fi}
\begin{tabular}{l*{4}{D{.}{.}{-1}}}
\toprule
                &\multicolumn{1}{c}{(1)}         &\multicolumn{1}{c}{(2)}         &\multicolumn{1}{c}{(3)}         &\multicolumn{1}{c}{(4)}         \\
\midrule
Treatment       &   43.346         &   43.225         &  -20.020         &  -20.884\sym{*}  \\
                & (39.999)         & (39.007)         & (12.649)         & (11.491)         \\
\addlinespace
Village Fund transfer&   -0.001         &   -0.001         &   -0.000         &   -0.000         \\
                &  (0.001)         &  (0.001)         &  (0.001)         &  (0.001)         \\
\addlinespace
Time trend      &   -0.499         &   -0.487         &    2.060         &    2.222\sym{*}  \\
                &  (0.417)         &  (0.408)         &  (1.258)         &  (1.150)         \\
\midrule
Observations    &    73228         &    73228         &     6336         &     6336         \\
Sample          & Province         & Province         & District         & District         \\
Controls        &      Yes         &       No         &      Yes         &       No         \\
Time Fixed Effects&      Yes         &      Yes         &      Yes         &      Yes         \\
Village Fixed Effects&      Yes         &      Yes         &      Yes         &      Yes         \\
\bottomrule
\multicolumn{5}{l}{\footnotesize Standard errors in parentheses}\\
\multicolumn{5}{l}{\footnotesize \sym{*} \(p<0.10\), \sym{**} \(p<0.05\), \sym{***} \(p<0.01\)}\\
\end{tabular}
}
}    
\label{t:FEIV2}\end{table}


\begin{table}[H]
\caption{Panel OLS estimates with interaction terms} 
\scalebox{0.8}{{
\def\sym#1{\ifmmode^{#1}\else\(^{#1}\)\fi}
\begin{tabular}{l*{8}{D{.}{.}{-1}}}
\toprule
                &\multicolumn{1}{c}{(1)}         &\multicolumn{1}{c}{(2)}         &\multicolumn{1}{c}{(3)}         &\multicolumn{1}{c}{(4)}         &\multicolumn{1}{c}{(5)}         &\multicolumn{1}{c}{(6)}         &\multicolumn{1}{c}{(7)}         &\multicolumn{1}{c}{(8)}         \\
\midrule
Interaction terms&    0.001         &    0.001         &    0.001         &    0.001         &    0.002         &    0.002         &    0.003         &    0.003         \\
                &  (0.003)         &  (0.003)         &  (0.003)         &  (0.003)         &  (0.003)         &  (0.003)         &  (0.003)         &  (0.003)         \\
\addlinespace
Treatment       &   -2.525         &   -2.456         &   -2.491         &   -2.419         &   -2.648         &   -2.890\sym{*}  &   -2.669         &   -2.995\sym{*}  \\
                &  (1.613)         &  (1.617)         &  (1.613)         &  (1.617)         &  (1.647)         &  (1.674)         &  (1.631)         &  (1.659)         \\
\addlinespace
Village Fund transfer&    0.000         &    0.000         &    0.000         &    0.000         &   -0.001\sym{*}  &   -0.001\sym{*}  &   -0.001\sym{*}  &   -0.001\sym{*}  \\
                &  (0.000)         &  (0.000)         &  (0.000)         &  (0.000)         &  (0.001)         &  (0.001)         &  (0.001)         &  (0.001)         \\
\midrule
Observations    &    75254         &    75254         &    75254         &    75254         &     6579         &     6579         &     6579         &     6579         \\
\(R^{2}\)       &    0.000         &    0.000         &    0.000         &    0.000         &    0.007         &    0.007         &    0.003         &    0.003         \\
Adjusted \(R^{2}\)&    0.000         &    0.000         &    0.000         &    0.000         &    0.006         &    0.006         &    0.002         &    0.002         \\
Sample          & Province         & Province         & Province         & Province         & District         & District         & District         & District         \\
Controls        &      Yes         &      Yes         &       No         &       No         &      Yes         &      Yes         &       No         &       No         \\
Time Fixed Effects&       No         &      Yes         &       No         &      Yes         &       No         &      Yes         &       No         &      Yes         \\
Village Fixed Effects&      Yes         &      Yes         &      Yes         &      Yes         &      Yes         &      Yes         &      Yes         &      Yes         \\
\bottomrule
\multicolumn{9}{l}{\footnotesize Standard errors in parentheses}\\
\multicolumn{9}{l}{\footnotesize \sym{*} \(p<0.10\), \sym{**} \(p<0.05\), \sym{***} \(p<0.01\)}\\
\end{tabular}
}
}
\label{t:pols2}\end{table}
\\
\pagebreak
\section{Figures}

\begin{figure}[h!]
\includegraphics[scale=0.7]{Final_Project/image/bbm-price-2014-2018.jpg}
\caption{Subsidized Fuel Price at Government's Price Control 2014-2018}
\note{Source: \citet{ichsan_2022}}
\label{f:1}
\end{figure}


\begin{figure}[h!]
\subcaptionbox{Village status in 2014\label{f:panel1}}{\includegraphics[scale=0.4]{Final_Project/image/vdi2014.png}}\hfill
\subcaptionbox{Village status in 2018\label{f:panel2}}{\includegraphics[scale=0.41]{Final_Project/image/vdi2018.jpg}}
\caption{Indonesia's Village Development Index status}
\note{Source: Statistics Indonesia from \citet{hartojo_2022}}
\label{f:2}\end{figure}

\begin{landscape}
\begin{figure}[h!]
\includegraphics[scale=0.8]{Final_Project/image/BBM Satu Harga.png}
\caption{Location of the One Price Fuel Program in 2016-2017}
\label{f:3}
\end{figure}
\end{landscape}
 
\end{document}
