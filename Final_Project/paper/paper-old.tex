% AEJ-Article.tex for AEA last revised 22 June 2011
\documentclass[ECON7023]{AEA}

%%%%%% NOTE FROM OVERLEAF: The mathtime package is no longer publicly available nor distributed. We recommend using a different font package e.g. mathptmx if you'd like to use a Times font.
\usepackage{mathptmx}

% The mathtime package uses a Times font instead of Computer Modern.
% Uncomment the line below if you wish to use the mathtime package:
%\usepackage[cmbold]{mathtime}
% Note that miktex, by default, configures the mathtime package to use commercial fonts
% which you may not have. If you would like to use mathtime but you are seeing error
% messages about missing fonts (mtex.pfb, mtsy.pfb, or rmtmi.pfb) then please see
% the technical support document at http://www.aeaweb.org/templates/technical_support.pdf
% for instructions on fixing this problem.

% Note: you may use either harvard or natbib (but not both) to provide a wider
% variety of citation commands than latex supports natively. See below.

% Uncomment the next line to use the natbib package with bibtex 
%\usepackage{natbib}

% Uncomment the next line to use the harvard package with bibtex
\usepackage[abbr]{harvard}
\bibliography{Final_Project/paper/references}
\bibliographystyle{aea.bst}

% This command determines the leading (vertical space between lines) in draft mode
% with 1.5 corresponding to "double" spacing.
\draftSpacing{1.5}

\begin{document}

\title{Measuring Accessibility in Rural Area}
\shortTitle{Measuring Accessibility in Rural Area}
\author{Maghfira Ramadhani \thanks{%
Ramadhani: Georgia Institute of Technology, (email: maghfira.ramadhani@gatech.edu). Acknowledgements}}
\date{\today}
\pubMonth{March}
\pubYear{2023}
\pubVolume{Vol}
\pubIssue{Issue}
\JEL{}
\Keywords{}

\begin{abstract}
We use th
\end{abstract}

\maketitle

The government of Indonesia is  \\

2nd par: existing literature summary\\

3rd par: talk about data, strategy, limitation\\

4th par: hypothesis and finding\\

5tth par: Structure of the paper

\section{Conceptual Framework}

In this section, I highlight a conceptual framework on how.\\

T

Sample figure:

\begin{figure}
Figure here.

\caption{Caption for figure below.}
\begin{figurenotes}
Figure notes without optional leadin.
\end{figurenotes}
\begin{figurenotes}[Source]
Figure notes with optional leadin (Source, in this case).
\end{figurenotes}
\end{figure}

\section{Context and Background}

Indonesia in general

Connectivity challenges in Indonesia \cite{sandee_2016}. Regarding rural accessibility, the challenge mainly related to intra-island connectivity ---links within individual islands--- is linking underdeveloped regions to growth centers. In the densely populated part of Java island, the city as the center of growth, the challenges of connectivity is mostly congestion-based challenge causing high-cost for mobility. In contrast, in the rural areas of Java, we can still find a village that we can only access by motorcycle or even only by foot. This challenge is somewhat similar in other main islands such as Sumatra, Kalimantan, Sulawesi and Papua. In these other mainlands, the challenges are the existence of adequate and reliable infrastructure that drives up the transportation cost.\\

The government initiatives in attracting foreign capital and facilitating public-private partnership in bringing a larga-scale infrastructure development has been a policy priority since 2015 \cite{pwc_2016}.

\section{Data}



Sample table:

\begin{table}
\caption{Caption for table above.}

\begin{tabular}{lll}
& Heading 1 & Heading 2 \\ 
Row 1 & 1 & 2 \\ 
Row 2 & 3 & 4%
\end{tabular}
\begin{tablenotes}
Table notes environment without optional leadin.
\end{tablenotes}
\begin{tablenotes}[Source]
Table notes environment with optional leadin (Source, in this case).
\end{tablenotes}
\end{table}

\section{Empirical Strategy}
American Economics Journal Pointers:

\begin{itemize}
\item Do not use an ``Introduction'' heading. Begin your introductory material
before the first section heading.

\item Avoid style markup (except sparingly for emphasis).

\item Avoid using explicit vertical or horizontal space.

\item Captions are short and go below figures but above tables.

\item The tablenotes or figurenotes environments may be used below tables
or figures, respectively, as demonstrated below.

\item If you have difficulties with the mathtime package, adjust the package
options appropriately for your platform. If you can't get it to work, just
remove the package or see our technical support document online (please
refer to the author instructions).

\item If you are using an appendix, it goes last, after the bibliography.
Use regular section headings to make the appendix headings.

\item If you are not using an appendix, you may delete the appendix command
and sample appendix section heading.

\item Either the natbib package or the harvard package may be used with bibtex.
To include one of these packages, uncomment the appropriate usepackage command
above. Note: you can't use both packages at once or compile-time errors will result.

\end{itemize}

References here (manual or bibTeX). If you are using bibTeX, add your bib file 
name in place of BibFile in the bibliography command.
% Remove or comment out the next two lines if you are not using bibtex.
\bibliographystyle{aea}
\bibliography{Final_Project/paper/references}

% The appendix command is issued once, prior to all appendices, if any.
\appendix

%\section{Mathematical Appendix}

\end{document}

