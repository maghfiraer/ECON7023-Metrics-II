\documentclass[letterpaper,11pt,leqno]{article}
\usepackage{proposal}
\usepackage{pdflscape}
\bibliographystyle{bibliography}

% Enter paper title:
\hypersetup{pdftitle={Paper Example}}

% Enter permanent URL to paper
%\available{https://github.com/pmichaillat/latex-paper}

% Enter BibTeX file with references:
\newcommand{\bib}{bibliography.bib}

% Enter PDF file with figures here:
\newcommand{\pdf}{figures.pdf}

% Fill out paper:
\begin{document}
\title{Improving Rural Accessibility in Indonesia: Fuel Subsidy versus Infrastructure Development}
\author{Maghfira Ramadhani}
\date{April 2023}       
\maketitle
 
\paragraph{Research Question} This paper builds on literature on rural development and fossil fuel subsidy in Indonesia. Related literature has indicated the two policies of subsidizing fuel and inter-government transfer have contributed to improving accessibility in rural areas. This research measures the magnitude of these mechanisms and tests whether they are complementary or supplementary. From a political economy perspective, infrastructure development is managed by the government directly, while the fuel subsidy is managed through the National Oil Company as a delivery vehicle \citep{ichsan_2022}. This research exercises the cost-benefit of the options to substantially inform decision-makers.

\paragraph{Related Literature} Although Indonesia has been subsidizing fuel for a long time, high fuel prices were easily observed in rural areas during the last decade. In the rural area, \citet{sambodo_2019} find that transportation spending dominates energy spending which could limit people's mobility and hamper development. Linking underdeveloped regions to growth centers is a challenge, even in Java, we can still find a village that we can only access by motorcycle or even only on foot. A somewhat similar pattern is also found in other main islands, the inexistent of adequate and reliable infrastructure drives up the transportation cost \citep{sandee_2016}.


\paragraph{Empirical Strategy} I observed the village level as the unit of analysis. I obtained proprietary Village Potential Statistics data for the years 2011, 2014, and 2018 from Indonesia's Central Bureau of Statistics and complement these data with village fund transfer data acquired from the Ministry of Village, Development of Disadvantaged Regions, and Transmigration for the year 2018. The descriptive statistics for the variables are shown in Table \ref{t1}.

I measure rural accessibility using the unit transportation cost (in Rp/km) of each individual village, which I define as the transportation cost from the village office to the sub-district office (in thousands Rp), divided by the distance (in km). The main explanatory variables are the treatment variable of fuel provision and the fund transfer from the central government to the village. These two variables potentially have endogeneity problems.  Other  data on geographic characteristics, number of schools, natural disaster occurrence, electricity customers, and poverty are used as covariates or instruments. I also add village-level fixed effects to account for different village characteristics that are not explained by other variables.

%\begin{landscape}
\begin{table}[t]
\caption{Summary Statistics}
\begin{tabular}{l*{2}{c}}
\hline\hline
                    &        2014&        2018\\
\hline
Transportation cost from Village Office to Subdistrict Office in 000s Rp./km&        2.55&        2.96\\
                    &      (7.11)&      (7.97)\\
                    &       64587&       64604\\
[1em]
=1 if slope/valleys, =0 vast land&        0.22&        0.19\\
                    &      (0.42)&      (0.39)\\
                    &       64587&       64604\\
[1em]
=1 if border with sea, =0 no border with sea&        0.15&        0.15\\
                    &      (0.36)&      (0.35)\\
                    &       64587&       64604\\
[1em]
=1 if inside or border with forest, =0 outside forest&        0.25&        0.23\\
                    &      (0.43)&      (0.42)\\
                    &       64587&       64604\\
[1em]
=1 if river used for transportation, =0 otherwise&        0.09&        0.08\\
                    &      (0.28)&      (0.28)\\
                    &       64587&       64604\\
[1em]
Landfall frequency [y-1]&        0.10&        0.14\\
                    &      (0.50)&      (0.61)\\
                    &       64587&       64604\\
[1em]
Earthquake frequency [y-1]&        0.05&        0.21\\
                    &      (0.39)&      (0.92)\\
                    &       64587&       64604\\
[1em]
Number of PLN electricity user household&      682.23&      772.14\\
                    &    (868.18)&    (984.28)\\
                    &       64587&       64604\\
[1em]
Number of Non-PLN electricity user household&       27.76&       22.42\\
                    &    (124.50)&    (109.47)\\
                    &       64587&       64604\\
[1em]
Number of Elementary School&        1.99&        2.00\\
                    &      (1.75)&      (1.72)\\
                    &       64587&       64604\\
[1em]
Number of Junior High School&        0.56&        0.61\\
                    &      (0.83)&      (0.88)\\
                    &       64587&       64604\\
[1em]
Number of poverty statement request&       80.47&       83.64\\
                    &    (176.76)&    (313.59)\\
                    &       64587&       64604\\
[1em]
Revenue from village fund transfer&      114.99&      117.46\\
                    &    (200.56)&    (126.81)\\
                    &       64587&       62403\\
\hline\hline
\end{tabular}

\label{t1}\end{table}



\bibliography{\bib}
%\end{landscape}
\end{document}
