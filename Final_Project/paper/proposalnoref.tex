\documentclass[letterpaper,11pt,leqno]{article}
\usepackage{proposal}
\usepackage{pdflscape}
\bibliographystyle{bibliography}

% Enter paper title:
\hypersetup{pdftitle={Paper Example}}

% Enter permanent URL to paper
%\available{https://github.com/pmichaillat/latex-paper}

% Enter BibTeX file with references:
\newcommand{\bib}{bibliography.bib}

% Enter PDF file with figures here:
\newcommand{\pdf}{figures.pdf}

% Fill out paper:
\begin{document}
\title{Improving Rural Accessibility in Indonesia: Fuel Subsidy versus Infrastructure Development}
\author{Maghfira Ramadhani}
\date{April 2023}       
\maketitle
 
\paragraph{Research Question} This paper builds on literature on rural development and fossil fuel subsidy in Indonesia. Related literature has indicated that both subsidizing fuel and inter-government transfer have contributed to improving accessibility in rural areas. This research measures the magnitude of these mechanisms and tests whether they are complementary or supplementary. From a political economy perspective, infrastructure development is managed by the government directly, while the fuel subsidy is managed through the National Oil Company (NOC) as a delivery agent \citep{ichsan_2022}. This research exercises the cost-benefit of the options to substantially inform decision-makers.

\paragraph{Related Literature} Although Indonesia has been subsidizing fuel for a long time, high fuel prices were still observed in rural areas in the last decade. Therefore since 2016, the government initiated the One Price Fuel program to guarantee the availability of subsidized fuel in the appointed subdistricts. Specifically rural areas, \citet{sambodo_2019} find that transportation spending dominates energy spending which could limit people's mobility and slow economic development. Regarding rural connectivity, linking underdeveloped regions to growth centers is a challenge. Even in Java, there are villages that can only be accessed by motorcycle or even only on foot. A similar pattern is found in other main islands. The lack of adequate and reliable infrastructure drives up the transportation cost \citep{sandee_2016}.


\paragraph{Empirical Strategy} In this paper, I observe the village as the unit of analysis. I obtained proprietary Village Potential Statistics data for the years 2014 and 2018 from Indonesia's Central Bureau of Statistics and complement these data with village fund transfer data acquired from the Ministry of Village, Development of Disadvantaged Regions, and Transmigration for the year 2018. I obtain a list of 58 government-appointed subdistricts of the program from the NOC. I assume all villages that are in the same subdistrict as the treatment group. The summary statistics for the main variables are shown in Table \ref{t1}.

I measure rural accessibility using the unit transportation cost (in thousands Rp/km) of each individual village, which is the transportation cost from the village office to the sub-district office (in thousands Rp), divided by the distance (in km). The main explanatory variables are the treatment variable and the fund transfer from the central government to the village. These two variables potentially have endogeneity problems.  Other  data on geographic characteristics and poverty are used as covariates. Data on the number of schools, natural disaster occurrences, and electricity customers are candidates for instruments. I also add village-level fixed effects to account for different village characteristics that are not explained by other variables.

\begin{landscape}
\begin{table}[t]
\caption{Summary statistics of main variables}
\begin{tabular}{l*{2}{ccccc}}
\toprule
                &     2014&         &         &         &         &     2018&         &         &         &         \\
                &     Mean&     S.D.&      Min&      Max&     Obs.&     Mean&     S.D.&      Min&      Max&     Obs.\\
\midrule
\emph{Transportation Cost}&         &         &         &         &         &         &         &         &         &         \\
\hspace{0.25cm} Unit transportation cost in 000s Rp./km&    22.35&   304.78&     0.00& 25000.00&    65917&     3.52&    29.19&     0.00&  5000.00&    65934\\
\hspace{0.25cm} Transportation cost in 000s Rp.&   106.42&  1205.17&     0.00& 55000.00&    65917&    35.41&   325.19&     0.00& 50000.00&    65934\\
\vspace{0.05em} \\ \emph{Natural Disaster}&         &         &         &         &         &         &         &         &         &         \\
\hspace{0.25cm} Landfall occurence average per year&     0.10&     0.50&     0.00&     9.00&    65917&     0.14&     0.61&     0.00&     9.00&    65934\\
\hspace{0.25cm} Earthquake occurence average per year&     0.06&     0.41&     0.00&     9.00&    65917&     0.21&     0.91&     0.00&     9.00&    65934\\
\vspace{0.05em} \\ \emph{Infrastructure}&         &         &         &         &         &         &         &         &         &         \\
\hspace{0.25cm} Number of PLN electricity user household&   679.68&   871.06&     0.00& 19714.00&    65917&   769.27&   988.12&     0.00& 23755.00&    65934\\
\hspace{0.25cm} Number of Junior High School&     0.55&     0.83&     0.00&    22.00&    65917&     0.60&     0.88&     0.00&    12.00&    65934\\
\hspace{0.25cm} Number of Senior High School or Vocational High School&     0.28&     0.71&     0.00&    40.00&    65917&     0.34&     0.77&     0.00&    13.00&    65934\\
\vspace{0.05em} \\ \emph{Inter-government Transfer}&         &         &         &         &         &         &         &         &         &         \\
\hspace{0.25cm} Revenue from village fund transfer&   115.53&   203.86&     0.00&  7792.00&    65917&   117.93&   127.47&     0.00& 13662.00&    63682\\
\bottomrule
\end{tabular}

\label{t1}\end{table}
\bibliography{\bib}
\end{landscape}

\end{document}
