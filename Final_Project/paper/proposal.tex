\documentclass[letterpaper,11pt,leqno]{article}
\usepackage{proposal}
\bibliographystyle{bibliography}

% Enter paper title:
\hypersetup{pdftitle={Paper Example}}

% Enter permanent URL to paper
%\available{https://github.com/pmichaillat/latex-paper}

% Enter BibTeX file with references:
\newcommand{\bib}{bibliography.bib}

% Enter PDF file with figures here:
\newcommand{\pdf}{figures.pdf}

% Fill out paper:
\begin{document}
\title{Improving Rural Accessibility in Indonesia: Fuel Subsidy versus Infrastructure Development}
\author{Maghfira Ramadhani
\thanks{Maghfira Ramadhani: Georgia Institute of Technology.}}
\date{April 2023}       
\maketitle
 
\paragraph{Research Question} This paper builds on rural infrastructure development and fossil fuel subsidy in Indonesia. Related literature have indicated the two policy of subsidizing or building have contributed to improving accessibility in rural areas. This research will try to find the magnitude of these mechanism and whether they are complementary or supplementary. From political economy perspective, the infrastructure development is managed by the government budget while the fuel subsidy is managed through NOC as a vehicle. These research can also exercise the cost benefit of the options and will substantially helps inform decision maker.

Regarding rural accessibility in Indonesia, the main challenge underdeveloped regions to growth centers. In the densely populated center of growth in Java, the challenge is mostly congestion-based causing high-cost for mobility. In contrast, in the rural areas of Java, we can still find a village that we can only access by motorcycle or even only by foot. Somewhat similar pattern also found  in other main islands such as Sumatra, Kalimantan, Sulawesi and Papua. In these other mainlands, the inexistent of adequate and reliable infrastructure drives up the transportation cost.

Indonesia have been subsidizing fuel for a long time, and the National Oil Company (NOC), Pertamina, contributed a big part in delivering the fuel subsidy to the public. However, high fuel prices is easily observed in rural regions in Indonesia and for years have limited the mobility of local communities and their economic development. In addressing the problem, since 2016 the government has implemented an intervention called One Price Fuel to guarantee the provision of fuels at the government's price control in the last miles.


\paragraph{Empirical Strategy} The main challenge of researching at village-level is data availability. I obtained a proprietary Village Potential Statistics data for the year 2011, 2014 and 2018 from Indonesia's Central Bureau of Statistics consisting of more than 70,000 observations each years. I also complement these data with village fund transfer data acquired from Ministry of Village, Development of Disadvantaged Regions, and Transmigration for the year 2018.

As an empirical strategy, I observed village level as the unit of analysis. I measure rural accessibility using the unit transportation cost (in Rp/km) of each individual village.  I define unit transportation cost, as the transportation cost from the village office to the sub-district office (in thousands Rp), divided by the distance from the village office to the sub-district office (in km). The main explanatory variable will be on the treatment variable of fuel provision and the fund transfer from the central government to the village. These two variables potentially have endogeneity problems.  Other  data on geographic characteristics, number of school, natural disaster occurrence, electricity customer, and poverty is used as covariates or instruments. I also add village level fixed effect to account for different village characteristic that are not explained by other variables.



\bibliography{\bib}

% Fill out appendix:
\newpage
\appendix
\section{Appendix title}\label{a:appendix1}

We extend the model of section~\ref{s:section} by introducing a quadratic cost. Obcaecati cupiditate non provident, similique sunt in culpa, qui officia deserunt mollitia animi, id est laborum et dolorum fuga. 

\subsection{Subsection title} 

At vero eos et accusamus et iusto odio dignissimos ducimus, qui blanditiis praesentium voluptatum deleniti atque corrupti, quos dolores et quas molestias excepturi sint, obcaecati cupiditate non provident, similique sunt in culpa, qui officia deserunt mollitia animi, id est laborum et dolorum fuga. 

\subsection{Another subsection}

 Et harum quidem rerum facilis est et expedita distinctio. Nam libero tempore, cum soluta nobis est eligendi optio cumque nihil impedit quo minus id quod maxime placeat facere possimus, omnis voluptas assumenda est, omnis dolor repellendus \citep{MS22a}. A corrolary in the appendix is as follows:

\begin{corollary} Similique sunt in culpa, qui officia deserunt mollitia animi, id est laborum et dolorum fuga:
\begin{equation*}
\mathbb{E}(\Omega) = \mathbb{P}(\omega\cdot \mu - \xi) - \sum_{i=0}^{m}\sum_{j=-\infty}^{n} \sigma(i,j) + 123^{56}.
\end{equation*}\end{corollary}

\paragraph{A paragraph with some math} Temporibus autem quibusdam $\xi$ et aut officiis debitis aut rerum necessitatibus saepe eveniet ut et voluptates repudiandae sint et molestiae non recusandae $1-\gamma$. Itaque earum rerum hic $S(z^*)$ tenetur a sapiente delectus $\mathcal{B}^\theta$, ut aut reiciendis voluptatibus maiores alias consequatur aut perferendis doloribus asperiores repellat $\mathcal{V}^i$. Aggregating these scenarios, we obtain the continuation value:
\begin{equation}
\mathbb{V}^r = (1-\gamma) \times 0 +\gamma S(z^*) v^s+\gamma [1-S(z^*)] \mathcal{V}^i-c.
\label{e:appendix1}\end{equation}

\paragraph{Paragraph with links to appendix equations} Ut enim ad minima veniam, quis nostrum exercitationem ullam corporis suscipit laboriosam, nisi ut aliquid ex ea commodi consequatur $\mathcal{C}$? Quis autem vel eum iure reprehenderit qui in ea voluptate velit esse quam nihil molestiae consequatur, vel illum qui dolorem eum fugiat quo voluptas nulla pariatur? Equation \eqref{e:appendix1} shows that autem vel eum iure reprehenderit qui in ea voluptate velit esse quam nihil molestiae consequatur.

\subsection{Larger figure, without panel, in the appendix} 

At vero eos et accusamus et iusto odio dignissimos ducimus, qui blanditiis praesentium voluptatum deleniti atque corrupti, quos dolores et quas molestias excepturi sint, obcaecati cupiditate non provident, similique sunt in culpa, qui officia deserunt mollitia animi, id est laborum et dolorum fuga in figure \ref{f:appendix1}.

\begin{figure}[t]
\includegraphics[scale=\mfig,page=1]{\pdf}
\caption{Caption for the larger graph}
\note{Note for the larger graph. Nam libero tempore, cum soluta nobis est eligendi optio, cumque nihil impedit, quo minus id, quod maxime placeat, facere possimus.}
\label{f:appendix1}\end{figure}

\section{Another section}\label{a:appendix2}

At vero eos et accusamus et iusto odio dignissimos ducimus, qui blanditiis praesentium voluptatum deleniti atque corrupti.

\subsection{Even larger figure, without panel, in the appendix} 

At vero eos et accusamus et iusto odio dignissimos ducimus, qui blanditiis praesentium voluptatum deleniti atque corrupti, quos dolores et quas molestias excepturi sint, obcaecati cupiditate non provident, similique sunt in culpa, qui officia deserunt mollitia animi, id est laborum et dolorum fuga, as showed in figure \ref{f:appendix2}.

\begin{figure}[t]
\includegraphics[scale=\lfig,page=3]{\pdf}
\caption{Caption for the even larger graph}
\note{Note for the larger graph. Nam libero tempore, cum soluta nobis est eligendi optio, cumque nihil impedit, quo minus id, quod maxime placeat, facere possimus.}
\label{f:appendix2}\end{figure}


\subsection{Final subsection with footnote and references}\label{a:subappendix}

Nemo enim ipsam voluptatem quia voluptas sit aspernatur aut odit aut fugit, sed quia consequuntur magni dolores eos qui ratione voluptatem sequi nesciunt $\mathcal{V}^i > \mathcal{V}^r$ \citep{MS21b}.\footnote{The reference goes to the reference list at the end of the main text.} Sed ut perspiciatis unde omnis iste natus error sit voluptatem accusantium doloremque laudantium, totam rem aperiam, eaque ipsa quae ab illo inventore veritatis et quasi architecto beatae vitae dicta sunt explicabo. This results are summarized in a repository with the following URL: \url{https://github.com/pmichaillat/latex-paper}.


\end{document}
