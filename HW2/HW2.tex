\documentclass[10pt]{article}
 
\usepackage[margin=1in]{geometry} 
\usepackage{amsmath,amsthm,amssymb, graphicx, multicol, array}
\usepackage{mathtools}

\newcommand\iid{\stackrel{\mathclap{iid}}{\sim}}
\newcommand\asym{\stackrel{\mathclap{a}}{\sim}}
\newcommand\convprob{\xrightarrow{p}}
\newcommand\convdist{\xrightarrow{d}}
\newcommand{\N}{\mathbb{N}}
\newcommand{\Z}{\mathbb{Z}}
\newcommand{\E}{\text{E}}
\newcommand{\V}{\text{Var}}
\newcommand{\Av}{\text{Avar}}
\newcommand{\se}{\text{se}}
\newcommand{\corr}{\text{Corr}}
\newcommand{\cov}{\text{Cov}}
\newcommand{\norm}{\text{Normal}}
\newcommand{\indep}{\perp \!\!\! \perp}

 
\newenvironment{problem}[2][Problem]{\begin{trivlist}
\item[\hskip \labelsep {\bfseries #1}\hskip \labelsep {\bfseries #2.}]}{\end{trivlist}}

\begin{document}
 
\title{Homework 2}
\author{ECON 7023: Econometrics II\\
Maghfira Ramadhani\\
February 9, 2022}
\date{Spring 2023}
\maketitle

\section*{Chapter 4}
\subsection*{Problem 4.11}
\begin{enumerate}
\item[a.] In Example 4.3, use \textit{KWW} and \textit{IQ} simultaneously as proxies for ability in equation (4.29). Compare the estimated return to education without a proxy for ability and with \textit{IQ} as the only proxy for ability.
\\ Answer: \\

\item[b.] Test \textit{KWW} and \textit{IQ} for joint significance in the estimated equation from part a.
\\ Answer:\\

\item[c.] When \textit{KWW} and \textit{IQ} are used as proxies for \textit{abil}, does the wage differential between nonblacks and blacks dissapear? What is the estimated differential?
\\ Answer:\\

\item[d.] Add the interaction $educ(IQ-100)$ and $educ(KWW-\overline{KWW})$ to the regression from part a, where $\overline{KWW}$ is the average score in the sample. Are these terms jointly significant using a standard \textit{F} test? Does adding them affect any important conclusions?
\\ Answer:\\
\end{enumerate}


\subsection*{Problem 4.13}
Use the data in CORNWELL.RAW (from Cornwell and Trumball, 1994) to estimate a model of county-level crime rates, using the year 1987 only.
\begin{enumerate}
\item[a.] Using logarithms of all variables, estimate a model relating the crime rate to the deterrent variables $prbarr, prbconv, prbpris$, and $avgsen.$
\\ Answer:\\


\item[b.] Add $\log(crmrte)$ for 1986 as an additional explanatory variable, and comment on how the estimated elasticities differ from part a.
\\ Answer:\\


\item[c.] Compute the $F$ statistic for joint significance of all of the wage variables (again in logs), using the restricted model from part b.
\\ Answer:\\


\item[d.] Redo part c, but make the test robust to heteroskedasticity of unknown form.
\\ Answer:\\
\end{enumerate}

\subsection*{Problem 4.14}
Use the data in ATTEND.RAW to answer this question
\begin{enumerate}
\item[a.] To determine the effects of attending lecture on final exam performance, estimate a model relating $stndfnl$ (the standardized final exam score) to $atndrte$ (the percent of lectures attended). Include the binary variables $frosh$ and $soph$ as explanatory variables. Interpret the coefficient on $atndrte$, and discuss its significance.
\\ Answer:\\

\item[b.] How confident are you that the OLS estimates from part a are estimating the causal effect of attendance? Explain.
\\ Answer:\\

\item[c.] As proxy variables for student ability, add to the regression $priGPA$ (prior cumulative GPA) and $ACT$ (achievement test score). Now what is the effect of $atndrte$? Discuss how the effect differs from that in part a.
\\ Answer:\\

\item[d.] What happens to the significance of the dummy variables in part c as compared with part a? Explain.
\\ Answer:\\

\item[e.] Add the squares of $priGPA$ and $ACT$ to the equation. What happens to the coefficient $atndrte$? Are the quadratics jointly significant?
\\ Answer:\\

\item[f.] To test for a nonlinear effect of $atndrte$, add its square to the equation from part e. What do you conclude? 
\\ Answer:\\
\end{enumerate}

\subsection*{Problem 4.15}
Assume that $y$ and $x_j$ have finite second moments, and write the linear projection of $y$ on $(1,x_1,\ldots,x_K)$ as
\begin{align*}
    &y=\beta_0+\beta_1 x_1+\cdots +\beta_K x_K +u=\beta_0 + \textbf{x}\pmb{\beta}+u,\\
    &E(u)=0,\ \ \ \ \ E(x_j u)=0, \ \ \ \ \ j=1,2,\ldots,K.
\end{align*}
\begin{enumerate}
\item[a.] Show that $\sigma_y^2=\V(\textbf{x}\pmb{\beta})+\sigma_u^2.$ 
\\ Answer:\\

\item[b.] For a random draw $i$ from the population, write $y_i=\beta_0+\textbf{x}_i\pmb{\beta}+u_i.$ Evaluate the following assumption, which has been known to appear in econometrics textbooks: $``\V(u_i)=\sigma^2=\V(y_i) \text{ for all }i."$
\\ Answer:\\

\item[c.] Define the population \textit{R}-squared by $\rho^2 \equiv 1-\sigma_u^2 / \sigma_y^2=\V(\textbf{x}\pmb{\beta})/\sigma_y^2.$ Show that the \textit{R}-squared, $R^2=1-SSR/SST,$ is a consistent estimator of $\rho^2,$ where SSR is the OLS sum of squared residuals and SST$=\sum_{i=1}^N (y_i-\bar{y})^2$ is the total sum of squares.
\\ Answer:\\

\item[d.] Evaluate the following statement: ``In the presence of heteroskedasticity, the \textit{R}-squared from an OLS regression is meaningless" (This kind of statement also tends to appear in econometrics texts.)
\\ Answer:\\
\end{enumerate}

\section*{Chapter 5}
\subsection*{Problem 5.1}
In this problem you are to establish the algebraic equivalence between 2SLS and OLS estimation of an equation containing additional regressor. Although the result is completely general, for simplicity consider a model with a single (suspected) endogenous variable:
\begin{align*}
    &y_1=\textbf{z}_1\pmb{\delta}_1+\alpha_1y_2+u_1,\\
    &y_2=\textbf{z}\pmb{\pi}_2+v_2.
\end{align*}
For notational clarity, we use $y_2$ as the suspected endogenous variable and \textbf{z} as the vector of all exogenous variables. The second equation is the reduced form for $y_2.$ Assume that \textbf{z} has at least one more element than $\textbf{z}_1.$
We know that one estimator of $(\pmb{\delta}_1,\alpha_1)$ is the 2SLS estimator using instruments \textbf{x}. Consider an alternative estimator of $(\pmb{\delta}_1,\alpha_1):$ (a) estimate the reduced form by OLS, save the residuals $\hat{v}_2$; (b) estimate the following equation by OLS:
\begin{align}
    y_1=\textbf{z}_1\pmb{\delta}_1+alpha_1+y_2+\rho_1+\hat{v}_2+error. \tag{5.52} \label{5.52}
\end{align}
Show that the OLS estimates of $\pmb{\delta}_1$ and $alpha_1$ from this regression are identical to the 2SLS estimators. (Hint: Use the partitioned regression algebra of OLS. In particular, if $\hat{y}=\textbf{x}_1\hat{\pmb{\beta}}_1+\textbf{x}_2\hat{\pmb{\beta}}_2$ is an OLS regression, $\hat{\pmb{\beta}}_1$ can be obtained by first regressing $\textbf{x}_1$ on $\textbf{x}_2$, getting the residuals, say $\ddot{\textbf{x}}_1$, and then regressing $y$ on $\ddot{\textbf{x}}_1$; see, for example, Davidson and MacKinnon (1993, Section 1.4). You must also use the fact that $\textbf{z}_1$ and $\hat{v}_2$ are orthogonal in the sample.)\\ 
Answer: \\

\subsection*{Problem 5.3}
Consider the following model to estimate the effects of several variables, including cigarette smoking, on the weight of newborns:
\begin{align}
    \log(bwght)=\beta_0+\beta_1male+\beta_2parity+\beta_3\log(faminc)+\beta_4packs+u, \tag{5.54}\label{5.54}
\end{align}
where $male$ is a binary indicator equal to one if the child is male, $parity$ is the birth order of this child, $faminc$ is family income, and $packs$ is the average number of packs of cigarettes smoked per day during pregnancy.
\begin{enumerate}
\item[a.] Why might you expect $packs$ to be correlated with $u$?
\\ Answer:\\

\item[b.] Suppose that you have data on average cigarette price in each woman's state of residence. Discuss whether this information is likely to satisfy the properties of a good instrumental variable for packs.
\\ Answer:\\

\item[c.] Use the data in BWGHT.RAW to estimate equation \eqref{5.54}. First, use OLS. Then, use 2SLS, where $cigprice$ is an instrument for packs. Discuss any important differences in the OLS and 2SLS estimates.
\\ Answer:\\

\item[d.] Estimate the reduced form for $packs$. What do you conclude about identification of equation \eqref{5.54} using $cigprice$ as an instrument for $packs$? What bearing does this conclusion have on your answer from part c?
\\ Answer:\\
\end{enumerate}

\subsection*{Problem 5.5}
One occasionally sees the following reasoning used in applied work for choosing instrumental variables in the context of omitted variables. The model is
\begin{align*}
    y_1=\textbf{z}_1\pmb{\delta}_1+\alpha_1y_2+\gamma q+a_1.
\end{align*}
where $q$ is the omitted factor. We assume that $a_1$ satisfies the structural error assumption $\E(a_1|\textbf{z}_1,y_2,q)=0$, that $\textbf{z}_1$ is exogenous in the sense that $\E(q|\textbf{z}_1)=0,$ but that $y_2$ and $q$ may be correlated. Let $\textbf{z}_2$ be a vector of instrumental variable candidates for $y_2$. Suppose it is known that $\textbf{z}_2$ appears in the linear projection of $y_2$ onto $(\textbf{z}_1,\textbf{z}_2)$, and so the requirement that $\textbf{z}_2$ be partially correlated with $y_2$ is satisfied. Also, we are willing to assume that $\textbf{z}_2$ is redundant in the structural equation, so that $a_1$ is uncorrelated with $\textbf{z}_2$. What we are unsure of is whether $\textbf{z}_2$ is correlated with the omitted variable $q$, in which case $\textbf{z}_2$ would not contain valid IVs.
To ``test" whether $\textbf{z}_2$ is in fact uncorrelated with $q$, it has been suggested to use OLS on the equation
\begin{align}
    y=\textbf{z}_1\pmb{\delta}_1+\alpha_1y_2+\textbf{z}_2\pmb{\psi}_1+u_1, \tag{5.55}\label{5.55}
\end{align}
where $u_1=\gamma q+a_1,$ and test $\text{H}_0:\pmb{\psi}_1=0.$ Why does this method not work?
\\ Answer:\\

\section*{Non-textbook Problem}
Show that IV estimation can be implemented as the 2SLS procedure.
\\ Answer:\\

\end{document}