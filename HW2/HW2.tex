\documentclass[10pt]{article}
 
\usepackage[margin=1in]{geometry} 
\usepackage{amsmath,amsthm,amssymb, graphicx, multicol, array}
\usepackage{mathtools}

\newcommand\iid{\stackrel{\mathclap{iid}}{\sim}}
\newcommand\asym{\stackrel{\mathclap{a}}{\sim}}
\newcommand\convprob{\xrightarrow{p}}
\newcommand\convdist{\xrightarrow{d}}
\newcommand{\N}{\mathbb{N}}
\newcommand{\Z}{\mathbb{Z}}
\newcommand{\E}{\text{E}}
\newcommand{\V}{\text{Var}}
\newcommand{\Av}{\text{Avar}}
\newcommand{\se}{\text{se}}
\newcommand{\corr}{\text{Corr}}
\newcommand{\cov}{\text{Cov}}
\newcommand{\norm}{\text{Normal}}
\newcommand{\indep}{\perp \!\!\! \perp}

 
\newenvironment{problem}[2][Problem]{\begin{trivlist}
\item[\hskip \labelsep {\bfseries #1}\hskip \labelsep {\bfseries #2.}]}{\end{trivlist}}

\begin{document}
 
\title{Homework 1}
\author{ECON 7023: Econometrics II\\
Maghfira Ramadhani\\
January 26, 2022}
\date{Spring 2023}
\maketitle

\section*{Part 1}
\subsection*{Problem 3.5}
Let $\left\{y_i:i=1,2,3,\ldots\right\}$ be an independent, identically distributed sequence with $\E(y_i^2)<\infty$. Let $\mu=\E(y_i)$ and $\sigma^2=\V(y_i).$
\begin{enumerate}
\item[a.] Let $\bar{y}_N$ denote the sample average based on a sample size of $N$. Find $\V[\sqrt{N}(\bar{y}_N-\mu)].$
\\ Answer: \\
By definition of sample mean, we have $\V(\bar{y}_N) = \V \left(\frac{1}{N}\sum_{i=1}^{N}{y_i}\right).$
Recall property of variance that $\V(ax)=a^2 \V(x)$ for random variable $x$ and constant $a$. Since $N$ is constant, thus we have
$\V(\bar{y}_N) = \left(\frac{1}{N}\right)^2 \V \left(\sum_{i=1}^{N}{y_i}\right).$
It is given that $y_i\iid(\mu,\sigma^2) \text{ for } i\in\left\{1,2,..,n\right\}$. Therefore $\V(y_i)=\sigma^2$ and is constant, and  since $y_i$'s are independent we have $\V \left(\sum_{i=1}^{N}{y_i}\right) = \sum_{i=1}^{N}{\V(y_i)}.$
Finally, combining these results, we have
\begin{align*}
    \V(\bar{y}_N) = \left(\frac{1}{N}\right)^2 \sum_{i=1}^{N}{\V(y_i)}=\left(\frac{1}{N}\right)^2 N\sigma^2=\sigma^2/N.
\end{align*}
Now, we can calculate $\V[\sqrt{N}(\bar{y}_N-\mu)].$ Since $N$ and $\mu$ are constant, we have
\begin{align*}
    \V[\sqrt{N}(\bar{y}_N-\mu)]=\V[\sqrt{N}(\bar{y}_N)]=(\sqrt{N})^2\V(\bar{y}_N)=N(\sigma^2/N)=\sigma^2.
\end{align*}\qed

\item[b.] What is the asymptotic variance of $\sqrt{N}(\bar{y}_N-\mu)$?
\\ Answer:\\
By Central Limit Theorem, since $y_i\iid(\mu,\sigma^2)$, then we have $\sqrt{N}(\bar{y}_N-\mu)\asym\norm(0,\sigma^2)$. So, the asymptotic variance of  $\sqrt{N}(\bar{y}_N-\mu)$ is $\Av[\sqrt{N}(\bar{y}_N-\mu)]=\sigma^2.$\\ \qed

\item[c.] What is the asymptotic variance of $\bar{y}_N$? Compare this with $\V(\bar{y}_N).$
\\ Answer:\\
From the result in (c) we have $\Av[\sqrt{N}(\bar{y}_N-\mu)]=\sigma^2.$ Since $N$ and $\mu$ are constant we have $\Av[\sqrt{N}(\bar{y}_N-\mu)]=\Av[\sqrt{N}(\bar{y}_N)]$. Recall property of variance that $\V(ax)=a^2 \V(x)$ for random variable $x$ and constant $a$. Applying this property to the asymptotic variance we have $\Av(\bar{y}_N))=\frac{1}{N}\Av[\sqrt{N}(\bar{y}_N)]=\frac{1}{N}\Av[\sqrt{N}(\bar{y}_N-\mu)]=\sigma^2/N.$ We get the same result as in (a).\\ \qed

\item[d.] What is the asymptotic standard deviation of $\bar{y}_N$?
\\ Answer:\\
By definition, the asymptotic standard deviation is the square root of the asymptotic variance that we have from previous result, formally, $\sqrt{\Av(\bar{y}_N)}=\sqrt{\sigma^2/N}=\sigma/\sqrt{N}.$\\ \qed
\item[e.] How would you obtain the asymptotic standard error of $\bar{y}_N$?
\\ Answer:\\
Applying \textbf{Definition 3.10} from the textbook, the asymptotic standard error of $\bar{y}_N$ is $\se(\bar{y}_N)=\hat{\sigma}/\sqrt{N},$ with $\hat{\sigma}$ a consistent estimator for $\sigma$, formally, $\hat{\sigma}^2\convprob\sigma^2$. Because we deal with $y_i$ as a single random variable, the consistent estimator for $\sigma^2$ will be $\hat{\sigma}^2=\frac{1}{N-1}\sum_{i=1}^N (y_i-\bar{y_N})^2.$ Then take the $|\sqrt{\hat{\sigma}^2}|$ as the $\hat{\sigma}$.
\begin{proof}
We need to show that $\hat{\sigma}^2=\frac{1}{N-1}\sum_{i=1}^N (y_i-\bar{y_N})^2$ is a consistent estimator for $\sigma^2$.
\end{proof}
\end{enumerate}


\subsection*{Problem 4.1}
Consider a standard $\log(wage)$ equation for men under the assumption that all explanatory variables are exogenous:
\begin{align*}
    &\log(wage)=\beta_0+\beta_1 married +\beta_2 educ+\mathbf{z\pmb{\gamma}}+u, \tag{4.49} \label{h1.0}\\
    &\E(u|married,educ,\mathbf{z})=0,
\end{align*}
where $\mathbf{z}$ contains factors other than marital status and education that can affect wages. When $\beta_1$ is small, $100\cdot \beta_1$ is approximately the ceteris paribus difference in wages between married and unmarried men. When $\beta_1$ is large, it might be preferable to use the exact percentage difference in $\E(wage|married,educ,\mathbf{z})$. Call this $\theta_1$.
\begin{enumerate}
\item[a.] Show that if $u$ is independent of all explanatory variables in equation \eqref{h1.0}, then $\theta_1=100\cdot[\exp(\beta_1)-1]$. (Hint: Find $\E(wage|married,educ,\mathbf{z})$ for $married=1$ and $married=0$, and find the percentage difference). A natural, consistent, estimator of $\theta_1$ is $\hat{\theta}_1=100\cdot[\exp(\hat{\beta}_1)-1]$, where $\hat{\beta_1}$ is the OLS estimator from equation \eqref{h1.0}.
\\ Answer:\\
Rewrite equation \eqref{h1.1},
\begin{align}
    &\log(wage)=\beta_0+\beta_1 married +\beta_2 educ+\mathbf{z\pmb{\gamma}}+u \notag\\
    &\Leftrightarrow wage=\exp{(\beta_0+\beta_1 married +\beta_2 educ+\mathbf{z\pmb{\gamma}}+u)} \notag\\
    &\Leftrightarrow wage=\exp{(u)}\exp{[\beta_0+\beta_1 married +\beta_2 educ+\mathbf{z\pmb{\gamma}})}. \label{h1.1}
\end{align}
Now take the expectation of \eqref{h1.1} with respect to all explanatory variables, we have
\begin{align}
    &\E(wage|married,educ,\mathbf{z})=\E(\exp{(u)}|married,educ,\mathbf{z})\exp{(\beta_0+\beta_1 married +\beta_2 educ+\mathbf{z}\pmb{\gamma})}.\label{h1.2}
\end{align}
If $u$ is independent of all explanatory variables, then $exp(u)$ will also be independent with all those variables, i.e, $\E(\exp{(u)}|married,educ,\mathbf{z})=E(\exp(u))=c,$ where $c$ is a constant. Now from \eqref{h1.2} we have
\begin{align}
    \E(wage|married,educ,\mathbf{z})=c\exp{(\beta_0+\beta_1 married +\beta_2 educ+\mathbf{z}\pmb{\gamma})}. \label{h1.3}
\end{align}
Then, we can compute the exact percentage different $\theta_1=100\cdot\frac{\E(wage|married=1)-\E(wage|married=0)}{\E(wage|married=0)}$ from \eqref{h1.3}, we have
\begin{align}
    \theta_1&=100\cdot\left[\frac{c\exp{(\beta_0+\beta_1 (1) +\beta_2 educ+\mathbf{z}\pmb{\gamma})}-c\exp{(\beta_0+\beta_1 (0) +\beta_2 educ+\mathbf{z}\pmb{\gamma})}}{c\exp{(\beta_0+\beta_1 (0) +\beta_2 educ+\mathbf{z}\pmb{\gamma})}}\right] \notag \\
    &=100\cdot\left[\frac{\exp{(\beta_0+\beta_1 +\beta_2 educ+\mathbf{z}\pmb{\gamma})}}{\exp{(\beta_0÷ +\beta_2 educ+\mathbf{z}\pmb{\gamma})}}-1\right] \notag \\
    &=100\cdot[\exp{(\beta_1)}-1]. \notag
\end{align} \qed

\item[b.] Use the delta method (see Section 3.5.2) to show that asymptotic standard error of $\hat{\theta}_1$ is $100\cdot[\exp(\hat{\beta}_1)]\cdot \se(\hat{\beta_1})$.
\\ Answer:\\
From previous result we have $\theta_1:\beta_1\to\mathbb{R}$ defined as $\theta_1=100\cdot[\exp{(\beta_1)}-1].$ We can use the OLS estimator $\hat{\beta_1}$ from \eqref{h1.0} to estimate $\hat{\theta}_1=100\cdot[\exp(\hat{\beta}_1)-1]$. We are interested in finding $\se(\hat{\theta}_1).$ By Delta Method we have
\begin{align}
    \Av[\hat{\theta}_1(\hat{\beta}_1)]=\left(\frac{\partial\hat{\theta}_1}{\partial \hat{\beta}_1} \right)^2 \Av(\hat{\beta}_1), \label{h1.4}
\end{align}
with the Jacobian in this case is only $1\times 1$, i.e, a scalar. From \eqref{h1.4} we take the square root to find $\se(\hat{\theta}_1),$ we have
\begin{align*}
    \se[\hat{\theta}_1(\hat{\beta}_1)]&=\left(\frac{\partial\hat{\theta}_1}{\partial \hat{\beta}_1} \right) \se(\hat{\beta}_1) \\
    &=100\cdot\exp(\hat{\beta}_1)\cdot \se(\hat{\beta_1}).
\end{align*}\qed

\item[c.] Repeat parts a and b by finding the exact percentage change in $\E(wage|married,educ,\mathbf{z})$ for any given change in $educ,\ \Delta educ.$ Call this $\theta_2.$ Explain how to estimate $\theta_2$ and obtain its asymptotic standard error.
\\ Answer:\\
The step involved is similar with the previous result. Recall \eqref{h1.3}, we have
\begin{align*}
    \E(wage|married,educ,\mathbf{z})=c\exp{(\beta_0+\beta_1 married +\beta_2 educ+\mathbf{z}\pmb{\gamma})}.
\end{align*}
Then, we can compute the exact percentage different for an additional years of education $\Delta educ$, that is $\theta_2=100\cdot\frac{\E(wage|educ=\epsilon+\Delta educ)-\E(wage|educ=\epsilon)}{\E(wage|educ=\epsilon)}$ from \eqref{h1.3}, we have
\begin{align}
    \theta_2&=100\cdot\left[\frac{c\exp{(\beta_0+\beta_1 married +\beta_2 (\epsilon+ \Delta educ)+\mathbf{z}\pmb{\gamma})}-c\exp{(\beta_0+\beta_1 married +\beta_2 (\epsilon)+\mathbf{z}\pmb{\gamma})}}{c\exp{(\beta_0+\beta_1 married +\beta_2 (\epsilon)+\mathbf{z}\pmb{\gamma})}}\right] \notag \\
    &=100\cdot\left[\frac{c\exp{(\beta_0+\beta_1 married +\beta_2 (\epsilon+ \Delta educ)+\mathbf{z}\pmb{\gamma})}}{c\exp{(\beta_0+\beta_1 married +\beta_2 (\epsilon)+\mathbf{z}\pmb{\gamma})}}-1\right] \notag \\
    &=100\cdot[\exp{(\beta_2\Delta educ)}-1]. \notag
\end{align} 
We can use the OLS estimator $\hat{\beta_2}$ from \eqref{h1.0} to estimate $\hat{\theta}_2=100\cdot[\exp{(\beta_2)}\Delta educ-1]$ for a given change in $educ,\ \Delta educ.$ 
\\ \qed \\
To estimate the asymptotic standard error we can use the Delta Method. We have that $\theta_2:\beta_2\to\mathbb{R}$ defined as $\theta_2=100\cdot[\exp{(\beta_2\Delta educ)} -1].$ Here we treat $\Delta educ$ as a constant because it is given. By Delta Method we have
\begin{align}
    \Av[\hat{\theta}_2(\hat{\beta}_2)]=\left(\frac{\partial\hat{\theta}_2}{\partial \hat{\beta}_2} \right)^2 \Av(\hat{\beta}_2), \label{h1.5}
\end{align}
with the Jacobian in this case is only $1\times 1$, i.e, a scalar. From \eqref{h1.5} we take the square root to find $\se(\hat{\theta}_2),$ we have
\begin{align*}
    \se[\hat{\theta}_2(\hat{\beta}_2)]&=\left(\frac{\partial\hat{\theta}_2}{\partial \hat{\beta}_2} \right) \se(\hat{\beta}_2) \\
    &=100\cdot |\Delta educ| \cdot\exp(\hat{\beta}_2 \Delta educ)\cdot \se(\hat{\beta_2}).
\end{align*}\qed


\item[d.] Use the data in NLS80.RAW to estimate equation \eqref{h1.0}, where \textbf{z} contains the remaining variables in equation (4.29) (except ability, of course). Find $\hat{\theta}_1$ and its standard error; find $\hat{\theta}_2$ and its standard error when $\Delta educ=4.$ 
\\ Answer:\\
From the data we get these regression result in the following table
\begin{table}[htbp]\centering
\def\sym#1{\ifmmode^{#1}\else\(^{#1}\)\fi}
\caption{Basic regression table \label{reg1}}
\begin{tabular}{l*{1}{c}}
\toprule
                    &\multicolumn{1}{c}{(1)}         \\
\midrule
years of work experience&       0.014\sym{***}\\
                    &     (0.003)         \\
\addlinespace
years with current employer&       0.012\sym{***}\\
                    &     (0.003)         \\
\addlinespace
=1 if married       &       0.199\sym{***}\\
                    &     (0.040)         \\
\addlinespace
=1 if live in south &      -0.091\sym{***}\\
                    &     (0.027)         \\
\addlinespace
=1 if live in SMSA  &       0.184\sym{***}\\
                    &     (0.027)         \\
\addlinespace
=1 if black         &      -0.188\sym{***}\\
                    &     (0.037)         \\
\addlinespace
years of education  &       0.065\sym{***}\\
                    &     (0.006)         \\
\addlinespace
Constant            &       5.395\sym{***}\\
                    &     (0.113)         \\
\midrule
Observations        &         935         \\
\bottomrule
\multicolumn{2}{l}{\footnotesize Standard errors in parentheses}\\
\multicolumn{2}{l}{\footnotesize Data: websuse nlswork}\\
\multicolumn{2}{l}{\footnotesize Second line note}\\
\multicolumn{2}{l}{\footnotesize \sym{*} \(p<0.10\), \sym{**} \(p<0.05\), \sym{***} \(p<0.01\)}\\
\end{tabular}
\end{table}

Recall the previous result, we have
\begin{align*}
    \hat{\theta}_1&=100\cdot[\exp{(\hat{\beta}_1)}-1]\\
    \se[\hat{\theta}_1]&=100\cdot\exp(\hat{\beta}_1)\cdot \se(\hat{\beta_1})\\
    \hat{\theta}_2&=100\cdot[\exp{(\hat{\beta_2}\Delta educ)}-1] \\
    \se[\hat{\theta}_2]&=100\cdot |\Delta educ| \cdot\exp(\hat{\beta}_2 \Delta educ)\cdot \se(\hat{\beta_2}).
\end{align*}
\end{enumerate}

\subsection*{Problem 4.2}
\begin{enumerate}
\item[a.] Show that, under random sampling and the zero conditional mean assumption $\E(u|\mathbf{x})=0,\E(\hat{\pmb{\beta}}|\mathbf{X})=\pmb{\beta}$ if $\mathbf{X}'\mathbf{X}$ is nonsingular.
\\ Answer:\\

Recall our OLS estimator
\begin{align*}
\hat{\pmb{\beta}}=\left(N^{-1}\sum_{i=1}^N\mathbf{x}_i'\mathbf{x}_i\right)^{-1}\left(N^{-1}\sum_{i=1}^N\mathbf{x}_i'y_i\right)=\pmb{\beta}+\left(N^{-1}\sum_{i=1}^N\mathbf{x}_i'\mathbf{x}_i\right)^{-1}\left(N^{-1}\sum_{i=1}^N\mathbf{x}_i'u_i\right),
\end{align*}
or in matrix notation
\begin{align}
\hat{\pmb{\beta}}=(\mathbf{X'X})^{-1}(\mathbf{X'y})=\pmb{\beta}+(\mathbf{X'X})^{-1}(\mathbf{X'u}), \label{h1.6}
\end{align}
with $\mathbf{X}$ a $N\times K$ data matrix of regressors with \textit{i}th row $\textbf{x}_i$, $\mathbf{y}$ a $N\times 1$ data vector with \textit{i}th element $y_i$, $\textbf{u}$ a $N\times 1$ matrix of errors. Now take the expectation of \eqref{h1.6} with respect to $\textbf{X}$, we have
\begin{align}
    \E(\hat{\pmb{\beta}}|\textbf{X})=\pmb{\beta}+(\mathbf{X'X})^{-1}\mathbf{X'}\E(\textbf{u}|\textbf{X})=\pmb{\beta}+(\mathbf{X'X})^{-1}\mathbf{X'}\textbf{0}=\pmb{\beta}. \label{h1.7}
\end{align}
In order for equation \eqref{h1.7} to hold, we need the zero conditional mean assumption, that is given, and we also need $(\mathbf{X'X})$ to be nonsingular so that the inverse, $(\mathbf{X'X})^{-1}$, exists.\\ \qed

\item[b.] In addition to the assumption from part a, assume that $\V(u|\mathbf{x})=\sigma^2.$ Show that $\V({\hat{\pmb{\beta}}}|\mathbf{X})=\sigma^2(\mathbf{X}'\mathbf{X})^{-1}.$
\\ Answer:\\
Recall previous result from (a) in equation \eqref{h1.6} we have
\begin{align*}
    \hat{\pmb{\beta}}=\pmb{\beta}+(\mathbf{X'X})^{-1}(\mathbf{X'u}).
\end{align*}
In this $\pmb{\beta}$ a vector of constants because it is a population property. Thus, if we take the conditional variance, we have
\begin{align}
    \V(\hat{\pmb{\beta}}|\textbf{X})&=\V((\mathbf{X'X})^{-1}\mathbf{X'u}|\textbf{X}) \notag \\
    &=[(\mathbf{X'X})^{-1}\mathbf{X'}]\V({u}|\textbf{X})[(\mathbf{X'X})^{-1}\textbf{X}']' \notag \\
    &=[(\mathbf{X'X})^{-1}\mathbf{X'}]\V({u}|\textbf{X})[\textbf{X}(\mathbf{X'X})^{-1}] \label{h1.8}
\end{align}
In equation \eqref{h1.8} we need to get rid of $\V({\textbf{u}}|\textbf{X})$, so recall that
\begin{align*}
\V({\textbf{u}}|\textbf{X})&=\E[\mathbf{u'u}|\textbf{X}]-\E[\mathbf{u}|\textbf{X}]^2 \\
&=\E[\mathbf{u'u}|\textbf{X}],
\end{align*}
which is a variance covariance matrix, since it is given in (a) that $\E[\mathbf{u}|\textbf{X}]=0.$ We are also given $\V(u_i|\mathbf{x}_i)=\E(u_i^2|\mathbf{x}_i)=\sigma^2$. Note that since $\textbf{x}_i$ is iid across $i$ and that the matrix is symmetric, we have \[\E(u_i^2|\mathbf{x}_i)=\E(u_i^2|\mathbf{x}_i,\mathbf{x}_j)=\E(u_i^2|\mathbf{X})=\sigma^2,\] 
where $u_i^2$ is a diagonal of variance covariance matrix. To transform the matrix quadratic form in \eqref{h1.8} to scalar multiplication, we need the covariance part of the matrix to be zero, i.e., $\E(u_iu_j|\mathbf{X})=\E(u_iu_j|\mathbf{x}_i,\mathbf{x}_j)=0$.  Note that since $u_i$ and $\textbf{x}_i$ are iid across $i$, by Law of Iterated Expectation we have 
\begin{align*}
    \E(u_iu_j|\mathbf{x}_i,\mathbf{x}_j)&=\E(u_iu_j|\mathbf{x}_i,\mathbf{x}_j,u_j)& &[\text{since } u_i,u_j \text{ are independent}]\\
    &=u_j\E(u_i|\mathbf{x}_i,\mathbf{x}_j,u_j)&\\
    &=u_j\E(u_i|\mathbf{x}_i,\mathbf{x}_j)&\\
    &=u_j\cdot 0& &[\text{from given assumption in point (a)}]\\
    &=0.
\end{align*}
Now we have that $\V({\textbf{u}}|\textbf{X})=\E[\mathbf{u'u}|\textbf{X}]$ is a diagonal matrix with the diagonal elements $\sigma^2$. Thus from \eqref{h1.8} the matrix inside the sandwich form can be transformed into scalar multiplication. Finally, we have
\begin{align*}
    \V(\hat{\pmb{\beta}}|\textbf{X})&=[(\mathbf{X'X})^{-1}\mathbf{X'}]\V({u}|\textbf{X})[\textbf{X}(\mathbf{X'X})^{-1}] \\
    &=\sigma^2[(\mathbf{X'X})^{-1}\mathbf{X'}\textbf{X}(\mathbf{X'X})^{-1}] \\
    &=\sigma^2[(\mathbf{X'X})^{-1}\textbf{I}] \\
    &=\sigma^2(\mathbf{X'X})^{-1}.
\end{align*}\qed
\end{enumerate}

\subsection*{Problem 4.4}
Show that estimator $\hat{\mathbf{B}}\equiv N^{-1} \sum_{i=1}^{N}\hat{u}_i^2 \mathbf{x}_i'\mathbf{x}_i$ is consistent for $\mathbf{B}=\E(u^2\mathbf{x}'\mathbf{x})$ by showing that 
\[N^{-1} \sum_{i=1}^{N}\hat{u}_i^2 \mathbf{x}_i'\mathbf{x}_i=N^{-1} \sum_{i=1}^{N}u_i^2 \mathbf{x}_i'\mathbf{x}_i+o_p(1).\]
(Hint: Write $\hat{u}_i^2=u_i^2-2\mathbf{x}_i u_i (\pmb{\hat{\beta}}-\pmb{\beta})+[\mathbf{x}_i (\pmb{\hat{\beta}}-\pmb{\beta})]^2$, and use the facts that sample averages are $O_p(1)$ when expectations exist and that $\pmb{\hat{\beta}}-\pmb{\beta}=o_p(1).$ Assume that all necessary expectations exist and are finite.)
\\ Answer:\\

\subsection*{Problem 4.9}
Consider a linear model where the dependent variable is in logarithmic form, and the lag of $\log(y)$ is also an explanatory variable:
\[\log(y)=\beta_0+\textbf{x}\pmb{\beta}+\alpha_1\log(y_{-1})+u,\ \ \E(u|\textbf{x},y_{-1})=0,\]
where the inclusion of $y_{-1}$ might be to control for correlation between policy variables in \textbf{x} and a previous value of $y$; see Example 4.4.
\begin{enumerate}
\item[a.] For estimating $\pmb{\beta}$, why do we obtain the same estimator if the \textit{growth} in $y$, $\log(y)-\log(y_{-1}),$ is used instead as the dependent variable?
\\ Answer: \\
Rearrange the equation by defining $\Delta \log(y)=\log(y)-\log(y_{-1})$, we have
\[\Delta \log(y)=\beta_0+\textbf{x}\pmb{\beta}+(\alpha_1-1)\log(y_{-1})+u.\]
So if we regress using the dependent variable $\Delta \log(y)$ we will have the same estimator for $\pmb{\beta}$. The only change in the estimator will happen with the regression on $\log(y_{-1}).$ \\ \qed

\item[b.] Suppose that there are no covariates \textbf{x} in the equation. Show that, if the distributions of $y$ and $y_{-1}$ are identical, then $|\alpha_1|<1$. This is the \textit{regression-to-the-mean}
phenomenon in a dynamic setting. (Hint: Show that $\alpha_1=\corr[\log(y),\log(y_{-1})].$)
\\ Answer: \\
Based on the prompt we will have the following linear model,
\[\log(y)=\beta_0+\alpha_1\log(y_{-1})+u.\]
It is also given that the distribution of $y$ and $y_{-1}$ are identical, formally, $y,y_{-1}\iid(\mu,\sigma^2)$ for some finite mean $\mu$, and variance $\sigma^2$. Recall the coefficient for simple regression, for this case we have
\begin{align}
    \alpha_1=\frac{\cov[\log(y),\log(y_{-1})]}{\V[\log(y_{-1})]}
\end{align}
Recall the correlation relation with covariance and variance, we have
\begin{align}
    \corr[\log(y),\log(y_{-1})]&=\frac{\cov[\log(y),\log(y_{-1})]}{\sqrt{\V[\log(y_{-1})]\V[\log(y)]}} \notag \\
    \Leftrightarrow \cov[\log(y),\log(y_{-1})]&= \corr[\log(y),\log(y_{-1})] \sqrt{\V[\log(y_{-1})]\V[\log(y)]}
\end{align}
Combining both of these equation, and the fact that we assume $\V[\log(y_{-1})]=\V[\log(y)]=\sigma^2$, we have
\begin{align}
    \alpha_1&=\frac{\corr[\log(y),\log(y_{-1})] \sqrt{\V[\log(y_{-1})]\V[\log(y)]}}{\V[\log(y_{-1})]} \notag \\
    &=\frac{\corr[\log(y),\log(y_{-1})] \sqrt{\sigma^2\cdot \sigma^2}}{\sigma^2} \notag \\
    &=\corr[\log(y),\log(y_{-1})]
\end{align}
Since we know $-1< \alpha_1=\corr[\log(y),\log(y_{-1})]<1,$ thus we can conclude that $|\alpha_1|<1.$ \\ \qed
\end{enumerate}

\section*{Part 2}
Show that for a regression model, if a regressor $x_j$ is measured with error, then it will be endogenous.
\\ Answer: \\
Consider a regression with $j$ variables:
\begin{align}
    y=\beta_0+\beta_1x_1+\ldots+\beta_jx_j^*+v,
\end{align}
where $y,x_1,\ldots,x_{j-1}$ are observable, but $x_j^*$ is not, instead we observe $x_j$ that is measured with error. We assume that $v$ has zero mean, i.e $\E(v)=0$, and is uncorrelated with $x_1,\ldots,x_{j-1},x_j^*$. We also assume $v$ is uncorrelated with $x_j$, i.e $\cov(x_j,v)=0$. Now define the measurement error of the population, $u_j$, as follows
\begin{align}
x_j=x_j^*+u_j\Leftrightarrow x_j^*=x_j-u_j.
\end{align}
We assume that $u_j$ has zero mean, i.e $\E(u_j)=0$, and is uncorrelated with $x_1,\ldots,x_{j-1},x_j^*$. Now substituting these into the previous equation we have
\begin{align}
    y&=\beta_0+\beta_1x_1+\ldots+\beta_j(x_j-u_j)+v \notag \\
    \Leftrightarrow y&=\beta_0+\beta_1x_1+\ldots+\beta_jx_j+(v-\beta_ju_j) \notag \\
    \Leftrightarrow y&=\beta_0+\beta_1x_1+\ldots+\beta_jx_j+e
\end{align}
Let's denote $e=v-\beta_ju_j$, and we assume $u_j$ and $v$ are independent. With this measurement error, equation is the actual regression that we will be estimating rather that the original one. Previously, we have assume that $\E(v)=0$, and $\E(u_j)=0$, and are uncorrelated with each $x_i$ including $x_j$, thus, we also have $\E(e)=0$. We want to assume that $x_j$ is exogenous, i.e $\cov(x_j,e)=\E(x_je)=0$. But we will have a problem of endogeneity in this case, that is
\begin{align*}
    \cov(x_j,e)&=\E(x_je)\\
    &=\E((x_j^*+u_j)(v-\beta_ju_j))\\
    &=\E(x_j^*v)+\E(u_jv)-\beta_j\E(x_j^*u_j)-\beta_j\E(u_j^2)\\
    &=0+0+0-\beta_j\E(u_j^2)& &\text{[because }x_j^*\indep v,x_j^*\indep u_j, u_j\indep v]\\
    &=-\beta_j\V(u_j) \\
    &\neq 0.
\end{align*}\qed
\end{document}