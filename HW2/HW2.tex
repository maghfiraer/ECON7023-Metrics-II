\documentclass[10pt]{article}
 
\usepackage[margin=1in]{geometry} 
\usepackage{amsmath,amsthm,amssymb, graphicx, multicol, array}
\usepackage{mathtools}
\usepackage{booktabs}
\usepackage{stata/stata}

\newcommand\iid{\stackrel{\mathclap{iid}}{\sim}}
\newcommand\asym{\stackrel{\mathclap{a}}{\sim}}
\newcommand\convprob{\xrightarrow{p}}
\newcommand\convdist{\xrightarrow{d}}
\newcommand{\N}{\mathbb{N}}
\newcommand{\Z}{\mathbb{Z}}
\newcommand{\E}{\text{E}}
\newcommand{\V}{\text{Var}}
\newcommand{\Av}{\text{Avar}}
\newcommand{\se}{\text{se}}
\newcommand{\corr}{\text{Corr}}
\newcommand{\cov}{\text{Cov}}
\newcommand{\norm}{\text{Normal}}
\newcommand{\indep}{\perp \!\!\! \perp}

 
\newenvironment{problem}[2][Problem]{\begin{trivlist}
\item[\hskip \labelsep {\bfseries #1}\hskip \labelsep {\bfseries #2.}]}{\end{trivlist}}

\begin{document}
 
\title{Homework 2}
\author{ECON 7023: Econometrics II\\
Maghfira Ramadhani\\
February 9, 2022}
\date{Spring 2023}
\maketitle

\section*{Chapter 4}
\subsection*{Problem 4.11}
\begin{enumerate}
\item[a.] In Example 4.3, use \textit{KWW} and \textit{IQ} simultaneously as proxies for ability in equation (4.29). Compare the estimated return to education without a proxy for ability and with \textit{IQ} as the only proxy for ability.
\\ Answer: \\
Table 1 show the estimates for return to education with and without proxies for ability.
\begin{table}[htbp]\centering
\def\sym#1{\ifmmode^{#1}\else\(^{#1}\)\fi}
\caption{Regression result for Problem 4.11.a. \label{reg1}}
\begin{tabular}{l*{3}{c}}
\toprule
                    &\multicolumn{1}{c}{(1)}         &\multicolumn{1}{c}{(2)}         &\multicolumn{1}{c}{(3)}         \\
\midrule
years of work experience&       0.013\sym{***}&       0.014\sym{***}&       0.014\sym{***}\\
                    &     (0.003)         &     (0.003)         &     (0.003)         \\
\addlinespace
years with current employer&       0.011\sym{***}&       0.011\sym{***}&       0.012\sym{***}\\
                    &     (0.002)         &     (0.002)         &     (0.002)         \\
\addlinespace
=1 if married       &       0.192\sym{***}&       0.200\sym{***}&       0.199\sym{***}\\
                    &     (0.039)         &     (0.039)         &     (0.039)         \\
\addlinespace
=1 if live in south &      -0.082\sym{***}&      -0.080\sym{***}&      -0.091\sym{***}\\
                    &     (0.026)         &     (0.026)         &     (0.026)         \\
\addlinespace
=1 if live in SMSA  &       0.176\sym{***}&       0.182\sym{***}&       0.184\sym{***}\\
                    &     (0.027)         &     (0.027)         &     (0.027)         \\
\addlinespace
=1 if black         &      -0.130\sym{***}&      -0.143\sym{***}&      -0.188\sym{***}\\
                    &     (0.040)         &     (0.039)         &     (0.038)         \\
\addlinespace
years of education  &       0.050\sym{***}&       0.054\sym{***}&       0.065\sym{***}\\
                    &     (0.007)         &     (0.007)         &     (0.006)         \\
\addlinespace
IQ score            &       0.003\sym{***}&       0.004\sym{***}&                     \\
                    &     (0.001)         &     (0.001)         &                     \\
\addlinespace
knowledge of world work score&       0.004\sym{**} &                     &                     \\
                    &     (0.002)         &                     &                     \\
\addlinespace
Constant            &       5.176\sym{***}&       5.176\sym{***}&       5.395\sym{***}\\
                    &     (0.128)         &     (0.128)         &     (0.113)         \\
\midrule
Observations        &         935         &         935         &         935         \\
\bottomrule
\multicolumn{4}{l}{\footnotesize Standard errors in parentheses}\\
\multicolumn{4}{l}{\footnotesize Data: NLS80.DTA}\\
\multicolumn{4}{l}{\footnotesize Wooldridge (2011)}\\
\multicolumn{4}{l}{\footnotesize \sym{*} \(p<0.10\), \sym{**} \(p<0.05\), \sym{***} \(p<0.01\)}\\
\end{tabular}
\end{table}
\\
The following code is used to run the regression.
\begin{stlog}reg lwage exper tenure married south urban black educ iq kww
reg lwage exper tenure married south urban black educ iq
reg lwage exper tenure married south urban black educ
\end{stlog}

The regression estimates by using $KWW$ and $IQ$ as proxies is shown in (1) with the estimated return to education of about 5\%. From (2), When only $IQ$ is used as proxy the estimated return is approximately 5.4\%. The case when no proxy is used the return to education estimates is about 6.5\% as shown in (3). OVerall, we have lower estimates when more proxy is used and the inference is still statistically significant.

\item[b.] Test \textit{KWW} and \textit{IQ} for joint significance in the estimated equation from part a.
\\ Answer:\\
We can use the $F$-test for joint significance of both variable with the following hypothesis,
\begin{align*}
    &\text{H}_0:\ \beta_{KWW}=\beta_{IQ}=0\\
    &\text{H}_1:\ \text{at least one of }\beta_{KWW}\text{ or }\beta_{IQ} \text{ is not zero}
\end{align*}
Below are the output for testing joint significance of $KWW$ and $IQ$, the $P$-value is smaller than either $0.1, 0.05$ or $0.01$. Thus we can reject the null hypothesis. And we can conclude that there is enough certainty to say that \textit{KWW} and \textit{IQ} are jointly significant.
\begin{stlog}. quietly reg lwage exper tenure married south urban black educ iq kww
{\smallskip}
. test kww iq
{\smallskip}
 ( 1)  kww = 0
 ( 2)  iq = 0
{\smallskip}
       F(  2,   925) =    8.59
            Prob > F =    0.0002
{\smallskip}
\end{stlog}


\item[c.] When \textit{KWW} and \textit{IQ} are used as proxies for \textit{abil}, does the wage differential between nonblacks and blacks dissapear? What is the estimated differential?
\\ Answer:\\
From estimates (1) in part a, the wage differential between nonblacks and blacks is still significant. Blacks are estimated to have about 13\% lower wage than non blacks holding other factors constant.

\item[d.] Add the interaction $educ(IQ-100)$ and $educ(KWW-\overline{KWW})$ to the regression from part a, where $\overline{KWW}$ is the average score in the sample. Are these terms jointly significant using a standard \textit{F} test? Does adding them affect any important conclusions?
\\ Answer:\\
We can use the $F$-test for joint significance of the interaction variables with the following hypothesis,
\begin{align*}
    &\text{H}_0:\ \beta_{educ(KWW-\overline{KWW})}=\beta_{educ(IQ-100)}=0\\
    &\text{H}_1:\ \text{at least one of }\beta_{educ(KWW-\overline{KWW})}\text{ or }\beta_{educ(IQ-100)} \text{ is not zero}
\end{align*}
\begin{stlog}. gen educiq=educ*(iq-100)
{\smallskip}
. egen meankww=mean(kww)
{\smallskip}
. gen educkww=educ*(kww-meankww)
{\smallskip}
. label variable educkww "educ*(kww-mean(kww))"
{\smallskip}
. label variable educiq "educ*(iq-100)"
{\smallskip}
. quietly reg lwage exper tenure married south urban black educ iq kww educiq educkww
{\smallskip}
. test educiq educkww
{\smallskip}
 ( 1)  educiq = 0
 ( 2)  educkww = 0
{\smallskip}
       F(  2,   923) =    4.19
            Prob > F =    0.0154
{\smallskip}
\end{stlog}

From the $F$-test, the $P$-value is around $0.015$, which indicates significant level of 1.5\%. Thus we can reject the null hypothesis. And we can conclude that there is enough certainty to say that $educ(IQ-100)$ and $educ(KWW-\overline{KWW})$ are jointly significant.\\
Table 2 show the estimates for return to education by adding interaction terms interaction $educ(IQ-100)$ and $educ(KWW-\overline{KWW})$. The return to education is even smaller know compared to estimates in Table 1 from part a, about 4.5\%. Regarding the effect of $KWW$, as it increases above its mean, the return of education will be larger. If the $KWW$ is one point above the mean, the return to education will be 4.5\%+0.02\% or about 4.52\%. This evidence show that the effect of knowledge of world work score to the return of education is positive, consistent with common intuition or at least my instuition.
\begin{table}[htbp]\centering
\def\sym#1{\ifmmode^{#1}\else\(^{#1}\)\fi}
\caption{Regression result for Problem 4.11.d.}
\begin{tabular}{l*{1}{c}}
\toprule
                    &\multicolumn{1}{c}{(1)}         \\
\midrule
years of work experience&       0.012\sym{***}\\
                    &     (0.003)         \\
\addlinespace
years with current employer&       0.011\sym{***}\\
                    &     (0.002)         \\
\addlinespace
=1 if married       &       0.198\sym{***}\\
                    &     (0.039)         \\
\addlinespace
=1 if live in south &      -0.081\sym{***}\\
                    &     (0.026)         \\
\addlinespace
=1 if live in SMSA  &       0.178\sym{***}\\
                    &     (0.027)         \\
\addlinespace
=1 if black         &      -0.138\sym{***}\\
                    &     (0.040)         \\
\addlinespace
years of education  &       0.045\sym{***}\\
                    &     (0.008)         \\
\addlinespace
IQ score            &       0.005         \\
                    &     (0.006)         \\
\addlinespace
knowledge of world work score&      -0.025\sym{**} \\
                    &     (0.011)         \\
\addlinespace
educ*(iq-100)       &      -0.000         \\
                    &     (0.000)         \\
\addlinespace
educ*(kww-mean(kww))&       0.002\sym{***}\\
                    &     (0.001)         \\
\addlinespace
Constant            &       6.080\sym{***}\\
                    &     (0.561)         \\
\midrule
Observations        &         935         \\
\bottomrule
\multicolumn{2}{l}{\footnotesize Standard errors in parentheses}\\
\multicolumn{2}{l}{\footnotesize Data: NLS80.DTA}\\
\multicolumn{2}{l}{\footnotesize Wooldridge (2011)}\\
\multicolumn{2}{l}{\footnotesize \sym{*} \(p<0.10\), \sym{**} \(p<0.05\), \sym{***} \(p<0.01\)}\\
\end{tabular}
\end{table}

\end{enumerate}


\subsection*{Problem 4.13}
Use the data in CORNWELL.RAW (from Cornwell and Trumball, 1994) to estimate a model of county-level crime rates, using the year 1987 only.
\begin{enumerate}
\item[a.] Using logarithms of all variables, estimate a model relating the crime rate to the deterrent variables $prbarr, prbconv, prbpris$, and $avgsen.$
\\ Answer:\\
Table 3 show the estimate a constant elasticities model (log-log model) of crime rate. 
\begin{table}[htbp]\centering
\def\sym#1{\ifmmode^{#1}\else\(^{#1}\)\fi}
\caption{Regression result for Problem 4.13.a and 4.13.b.}
\begin{tabular}{l*{2}{c}}
\toprule
                    &\multicolumn{1}{c}{(1)}         &\multicolumn{1}{c}{(2)}         \\
\midrule
log(prbarr)         &      -0.724\sym{***}&      -0.185\sym{***}\\
                    &     (0.115)         &     (0.063)         \\
\addlinespace
log(prbconv)        &      -0.473\sym{***}&      -0.039         \\
                    &     (0.083)         &     (0.047)         \\
\addlinespace
log(prbpris)        &       0.160         &      -0.127         \\
                    &     (0.206)         &     (0.099)         \\
\addlinespace
log(avgsen)         &       0.076         &      -0.152\sym{*}  \\
                    &     (0.163)         &     (0.078)         \\
\addlinespace
log(crmrte)[t-1]    &                     &       0.780\sym{***}\\
                    &                     &     (0.045)         \\
\addlinespace
Constant            &      -4.868\sym{***}&      -0.767\sym{**} \\
                    &     (0.432)         &     (0.313)         \\
\midrule
Observations        &          90         &          90         \\
\bottomrule
\multicolumn{3}{l}{\footnotesize Standard errors in parentheses}\\
\multicolumn{3}{l}{\footnotesize Data: CORNWELL.DTA}\\
\multicolumn{3}{l}{\footnotesize Cornwell and Trumball (1994)}\\
\multicolumn{3}{l}{\footnotesize \sym{*} \(p<0.10\), \sym{**} \(p<0.05\), \sym{***} \(p<0.01\)}\\
\end{tabular}
\end{table}
\\
We might expect that crime rate will decrease as the probability of arrest and probability of conviction increase. As shown in (1) the elasticities of crime or coefficient on probability of arrest and probability of conviction have negative sign as expected and are statistically significant. The elasticities of crime on probability of serving prison term and average sentence length showing positive sign but statistically not significant.

\item[b.] Add $\log(crmrte)$ for 1986 as an additional explanatory variable, and comment on how the estimated elasticities differ from part a.
\\ Answer:\\
As shown in estimate in (2) in Table 3, by adding $\log(crmrte)$ for 1986 as an additional explanatory variable there are some change in the elasticities. First, the elasticity of crime rate on probability of arrest decrease significantly and still statistically significant. Second, the elasticity of crime rate with respect to probability of conviction also decrease significantly and become statistically insignificant. The elasticities on probability of serving prison term and on average sentence length remain insignificant and both change sign, with the latter become statistically significant at 10\% level. The elasticities on the lagged crime rate is comparably large and statistically significant.

\item[c.] Compute the $F$ statistic for joint significance of all of the wage variables (again in logs), using the restricted model from part b.
\\ Answer:\\
We can use the $F$-test for joint significance of all of the wage variables with the following hypothesis, in this case we use assume homoskedasticity.
\begin{align*}
    &\text{H}_0:\ \beta_{lwcon}=\ldots=\beta_{lwloc}=0\\
    &\text{H}_1:\ \text{at least one of }\beta_{lwcon}\text{ or }\ldots\text{ or }\beta_{lwloc} \text{ is not zero}
\end{align*}
\begin{stlog}. reg lcrmrte lprbarr lprbconv lprbpris lavgsen lcrmrte1 lwcon-lwloc if d87
{\smallskip}
      Source {\VBAR}       SS           df       MS      Number of obs   =        90
\HLI{13}{\PLUS}\HLI{34}   F(14, 75)       =     43.81
       Model {\VBAR}  23.8798774        14  1.70570553   Prob > F        =    0.0000
    Residual {\VBAR}  2.91982063        75  .038930942   R-squared       =    0.8911
\HLI{13}{\PLUS}\HLI{34}   Adj R-squared   =    0.8707
       Total {\VBAR}   26.799698        89  .301120202   Root MSE        =    .19731
{\smallskip}
\HLI{13}{\TOPT}\HLI{64}
     lcrmrte {\VBAR} Coefficient  Std. err.      t    P>|t|     [95\% conf. interval]
\HLI{13}{\PLUS}\HLI{64}
     lprbarr {\VBAR}  -.1725122   .0659533    -2.62   0.011    -.3038978   -.0411265
    lprbconv {\VBAR}  -.0683639    .049728    -1.37   0.173    -.1674273    .0306994
    lprbpris {\VBAR}  -.2155553   .1024014    -2.11   0.039    -.4195493   -.0115614
     lavgsen {\VBAR}  -.1960546   .0844647    -2.32   0.023     -.364317   -.0277923
    lcrmrte1 {\VBAR}   .7453414   .0530331    14.05   0.000     .6396942    .8509887
       lwcon {\VBAR}  -.2850008   .1775178    -1.61   0.113    -.6386344    .0686327
       lwtuc {\VBAR}   .0641312    .134327     0.48   0.634    -.2034619    .3317244
       lwtrd {\VBAR}    .253707   .2317449     1.09   0.277    -.2079525    .7153665
       lwfir {\VBAR}  -.0835258   .1964974    -0.43   0.672    -.4749687    .3079171
       lwser {\VBAR}   .1127542   .0847427     1.33   0.187    -.0560619    .2815703
       lwmfg {\VBAR}   .0987371   .1186099     0.83   0.408    -.1375459    .3350201
       lwfed {\VBAR}   .3361278   .2453134     1.37   0.175    -.1525615    .8248172
       lwsta {\VBAR}   .0395089   .2072112     0.19   0.849    -.3732769    .4522947
       lwloc {\VBAR}  -.0369855   .3291546    -0.11   0.911    -.6926951    .6187241
       _cons {\VBAR}  -3.792525   1.957472    -1.94   0.056    -7.692009    .1069593
\HLI{13}{\BOTT}\HLI{64}
{\smallskip}
. testparm lwcon-lwloc
{\smallskip}
 ( 1)  lwcon = 0
 ( 2)  lwtuc = 0
 ( 3)  lwtrd = 0
 ( 4)  lwfir = 0
 ( 5)  lwser = 0
 ( 6)  lwmfg = 0
 ( 7)  lwfed = 0
 ( 8)  lwsta = 0
 ( 9)  lwloc = 0
{\smallskip}
       F(  9,    75) =    1.50
            Prob > F =    0.1643
{\smallskip}
\end{stlog}

From the $F$-test we have that all of the wage variables are jointly insignificant since the $P$-value is more than 0.1. We might also interested in looking at individual elasticity and find out that the sign of the elasticity are not consistent. 

\item[d.] Redo part c, but make the test robust to heteroskedasticity of unknown form.
\\ Answer:\\
Analogous to c, we use the $F$-test for joint significance of all of the wage variables with the following hypothesis, now we use assume heteroskedasticity and use robust standard error.
\begin{align*}
    &\text{H}_0:\ \beta_{lwcon}=\ldots=\beta_{lwloc}=0\\
    &\text{H}_1:\ \text{at least one of }\beta_{lwcon}\text{ or }\ldots\text{ or }\beta_{lwloc} \text{ is not zero}
\end{align*}
\begin{stlog}. reg lcrmrte lprbarr lprbconv lprbpris lavgsen lcrmrte1 lwcon-lwloc if d87, robust
{\smallskip}
Linear regression                               Number of obs     =         90
                                                F(14, 75)         =     110.75
                                                Prob > F          =     0.0000
                                                R-squared         =     0.8911
                                                Root MSE          =     .19731
{\smallskip}
\HLI{13}{\TOPT}\HLI{64}
             {\VBAR}               Robust
     lcrmrte {\VBAR} Coefficient  std. err.      t    P>|t|     [95\% conf. interval]
\HLI{13}{\PLUS}\HLI{64}
     lprbarr {\VBAR}  -.1725122   .0831236    -2.08   0.041    -.3381028   -.0069215
    lprbconv {\VBAR}  -.0683639   .0874696    -0.78   0.437    -.2426123    .1058844
    lprbpris {\VBAR}  -.2155553   .0895319    -2.41   0.019    -.3939121   -.0371986
     lavgsen {\VBAR}  -.1960546   .0976231    -2.01   0.048    -.3905298   -.0015794
    lcrmrte1 {\VBAR}   .7453414   .1594535     4.67   0.000     .4276937    1.062989
       lwcon {\VBAR}  -.2850008   .1276141    -2.23   0.029    -.5392212   -.0307805
       lwtuc {\VBAR}   .0641312   .1108165     0.58   0.565    -.1566265     .284889
       lwtrd {\VBAR}    .253707   .1712913     1.48   0.143    -.0875227    .5949368
       lwfir {\VBAR}  -.0835258   .1477461    -0.57   0.574     -.377851    .2107995
       lwser {\VBAR}   .1127542   .0715635     1.58   0.119    -.0298077     .255316
       lwmfg {\VBAR}   .0987371   .1083497     0.91   0.365    -.1171065    .3145807
       lwfed {\VBAR}   .3361278   .4416827     0.76   0.449    -.5437491    1.216005
       lwsta {\VBAR}   .0395089   .1829791     0.22   0.830    -.3250042     .404022
       lwloc {\VBAR}  -.0369855   .2825442    -0.13   0.896    -.5998425    .5258714
       _cons {\VBAR}  -3.792525   3.383901    -1.12   0.266     -10.5336    2.948552
\HLI{13}{\BOTT}\HLI{64}
{\smallskip}
. testparm lwcon-lwloc
{\smallskip}
 ( 1)  lwcon = 0
 ( 2)  lwtuc = 0
 ( 3)  lwtrd = 0
 ( 4)  lwfir = 0
 ( 5)  lwser = 0
 ( 6)  lwmfg = 0
 ( 7)  lwfed = 0
 ( 8)  lwsta = 0
 ( 9)  lwloc = 0
{\smallskip}
       F(  9,    75) =    2.19
            Prob > F =    0.0319
{\smallskip}
\end{stlog}

Since the $P$-value is less than 0.1 we reject the null hypothesis. Now the $F$-test show that we have all of the wage variables jointly significant.
\end{enumerate}

\subsection*{Problem 4.14}
Use the data in ATTEND.RAW to answer this question
\begin{enumerate}
\item[a.] To determine the effects of attending lecture on final exam performance, estimate a model relating $stndfnl$ (the standardized final exam score) to $atndrte$ (the percent of lectures attended). Include the binary variables $frosh$ and $soph$ as explanatory variables. Interpret the coefficient on $atndrte$, and discuss its significance.
\\ Answer:\\
\begin{table}[htbp]\centering
\def\sym#1{\ifmmode^{#1}\else\(^{#1}\)\fi}
\caption{Regression result for Problem 4.14.a., 4.14.c, 4.14.e, and 4.14.f}
\begin{tabular}{l*{4}{c}}
\toprule
                    &\multicolumn{1}{c}{(1)}         &\multicolumn{1}{c}{(2)}         &\multicolumn{1}{c}{(3)}         &\multicolumn{1}{c}{(4)}         \\
\midrule
percent classes attended&       0.008\sym{***}&       0.005\sym{**} &       0.006\sym{***}&       0.006         \\
                    &     (0.002)         &     (0.002)         &     (0.002)         &     (0.011)         \\
\addlinespace
=1 if freshman      &      -0.290\sym{**} &      -0.049         &      -0.105         &      -0.105         \\
                    &     (0.116)         &     (0.108)         &     (0.107)         &     (0.107)         \\
\addlinespace
=1 if sophomore     &      -0.118         &      -0.160\sym{*}  &      -0.181\sym{**} &      -0.181\sym{**} \\
                    &     (0.099)         &     (0.090)         &     (0.089)         &     (0.089)         \\
\addlinespace
cumulative GPA prior to term&                     &       0.427\sym{***}&      -1.526\sym{***}&      -1.525\sym{***}\\
                    &                     &     (0.082)         &     (0.474)         &     (0.476)         \\
\addlinespace
ACT score           &                     &       0.084\sym{***}&      -0.112         &      -0.112         \\
                    &                     &     (0.011)         &     (0.098)         &     (0.098)         \\
\addlinespace
priGPA^2            &                     &                     &       0.368\sym{***}&       0.368\sym{***}\\
                    &                     &                     &     (0.089)         &     (0.089)         \\
\addlinespace
ACT^2               &                     &                     &       0.004\sym{*}  &       0.004\sym{*}  \\
                    &                     &                     &     (0.002)         &     (0.002)         \\
\addlinespace
atndrte^2           &                     &                     &                     &       0.000         \\
                    &                     &                     &                     &     (0.000)         \\
\addlinespace
Constant            &      -0.502\sym{**} &      -3.297\sym{***}&       1.385         &       1.394         \\
                    &     (0.196)         &     (0.309)         &     (1.239)         &     (1.267)         \\
\midrule
Observations        &         680         &         680         &         680         &         680         \\
\bottomrule
\multicolumn{5}{l}{\footnotesize Standard errors in parentheses}\\
\multicolumn{5}{l}{\footnotesize Data: ATTEND.DTA}\\
\multicolumn{5}{l}{\footnotesize Wooldridge (2011)}\\
\multicolumn{5}{l}{\footnotesize \sym{*} \(p<0.10\), \sym{**} \(p<0.05\), \sym{***} \(p<0.01\)}\\
\end{tabular}
\end{table}

\item[b.] How confident are you that the OLS estimates from part a are estimating the causal effect of attendance? Explain.
\\ Answer:\\

\item[c.] As proxy variables for student ability, add to the regression $priGPA$ (prior cumulative GPA) and $ACT$ (achievement test score). Now what is the effect of $atndrte$? Discuss how the effect differs from that in part a.
\\ Answer:\\

\item[d.] What happens to the significance of the dummy variables in part c as compared with part a? Explain.
\\ Answer:\\

\item[e.] Add the squares of $priGPA$ and $ACT$ to the equation. What happens to the coefficient $atndrte$? Are the quadratics jointly significant?
\\ Answer:\\

\item[f.] To test for a nonlinear effect of $atndrte$, add its square to the equation from part e. What do you conclude? 
\\ Answer:\\
\end{enumerate}

\subsection*{Problem 4.15}
Assume that $y$ and $x_j$ have finite second moments, and write the linear projection of $y$ on $(1,x_1,\ldots,x_K)$ as
\begin{align*}
    &y=\beta_0+\beta_1 x_1+\cdots +\beta_K x_K +u=\beta_0 + \textbf{x}\pmb{\beta}+u,\\
    &E(u)=0,\ \ \ \ \ E(x_j u)=0, \ \ \ \ \ j=1,2,\ldots,K.
\end{align*}
\begin{enumerate}
\item[a.] Show that $\sigma_y^2=\V(\textbf{x}\pmb{\beta})+\sigma_u^2.$ 
\\ Answer:\\

\item[b.] For a random draw $i$ from the population, write $y_i=\beta_0+\textbf{x}_i\pmb{\beta}+u_i.$ Evaluate the following assumption, which has been known to appear in econometrics textbooks: $``\V(u_i)=\sigma^2=\V(y_i) \text{ for all }i."$
\\ Answer:\\

\item[c.] Define the population \textit{R}-squared by $\rho^2 \equiv 1-\sigma_u^2 / \sigma_y^2=\V(\textbf{x}\pmb{\beta})/\sigma_y^2.$ Show that the \textit{R}-squared, $R^2=1-SSR/SST,$ is a consistent estimator of $\rho^2,$ where SSR is the OLS sum of squared residuals and SST$=\sum_{i=1}^N (y_i-\bar{y})^2$ is the total sum of squares.
\\ Answer:\\

\item[d.] Evaluate the following statement: ``In the presence of heteroskedasticity, the \textit{R}-squared from an OLS regression is meaningless" (This kind of statement also tends to appear in econometrics texts.)
\\ Answer:\\
\end{enumerate}

\section*{Chapter 5}
\subsection*{Problem 5.1}
In this problem you are to establish the algebraic equivalence between 2SLS and OLS estimation of an equation containing additional regressor. Although the result is completely general, for simplicity consider a model with a single (suspected) endogenous variable:
\begin{align*}
    &y_1=\textbf{z}_1\pmb{\delta}_1+\alpha_1y_2+u_1,\\
    &y_2=\textbf{z}\pmb{\pi}_2+v_2.
\end{align*}
For notational clarity, we use $y_2$ as the suspected endogenous variable and \textbf{z} as the vector of all exogenous variables. The second equation is the reduced form for $y_2.$ Assume that \textbf{z} has at least one more element than $\textbf{z}_1.$
We know that one estimator of $(\pmb{\delta}_1,\alpha_1)$ is the 2SLS estimator using instruments \textbf{x}. Consider an alternative estimator of $(\pmb{\delta}_1,\alpha_1):$ (a) estimate the reduced form by OLS, save the residuals $\hat{v}_2$; (b) estimate the following equation by OLS:
\begin{align}
    y_1=\textbf{z}_1\pmb{\delta}_1+alpha_1+y_2+\rho_1+\hat{v}_2+error. \tag{5.52} \label{5.52}
\end{align}
Show that the OLS estimates of $\pmb{\delta}_1$ and $alpha_1$ from this regression are identical to the 2SLS estimators. (Hint: Use the partitioned regression algebra of OLS. In particular, if $\hat{y}=\textbf{x}_1\hat{\pmb{\beta}}_1+\textbf{x}_2\hat{\pmb{\beta}}_2$ is an OLS regression, $\hat{\pmb{\beta}}_1$ can be obtained by first regressing $\textbf{x}_1$ on $\textbf{x}_2$, getting the residuals, say $\ddot{\textbf{x}}_1$, and then regressing $y$ on $\ddot{\textbf{x}}_1$; see, for example, Davidson and MacKinnon (1993, Section 1.4). You must also use the fact that $\textbf{z}_1$ and $\hat{v}_2$ are orthogonal in the sample.)\\ 
Answer: \\

\subsection*{Problem 5.3}
Consider the following model to estimate the effects of several variables, including cigarette smoking, on the weight of newborns:
\begin{align}
    \log(bwght)=\beta_0+\beta_1male+\beta_2parity+\beta_3\log(faminc)+\beta_4packs+u, \tag{5.54}\label{5.54}
\end{align}
where $male$ is a binary indicator equal to one if the child is male, $parity$ is the birth order of this child, $faminc$ is family income, and $packs$ is the average number of packs of cigarettes smoked per day during pregnancy.
\begin{enumerate}
\item[a.] Why might you expect $packs$ to be correlated with $u$?
\\ Answer:\\

\item[b.] Suppose that you have data on average cigarette price in each woman's state of residence. Discuss whether this information is likely to satisfy the properties of a good instrumental variable for packs.
\\ Answer:\\

\item[c.] Use the data in BWGHT.RAW to estimate equation \eqref{5.54}. First, use OLS. Then, use 2SLS, where $cigprice$ is an instrument for packs. Discuss any important differences in the OLS and 2SLS estimates.
\\ Answer:\\
\begin{stlog}. reg lbwght male parity lfaminc packs
{\smallskip}
      Source {\VBAR}       SS           df       MS      Number of obs   =     1,388
\HLI{13}{\PLUS}\HLI{34}   F(4, 1383)      =     12.55
       Model {\VBAR}  1.76664363         4  .441660908   Prob > F        =    0.0000
    Residual {\VBAR}    48.65369     1,383  .035179819   R-squared       =    0.0350
\HLI{13}{\PLUS}\HLI{34}   Adj R-squared   =    0.0322
       Total {\VBAR}  50.4203336     1,387  .036352079   Root MSE        =    .18756
{\smallskip}
\HLI{13}{\TOPT}\HLI{64}
      lbwght {\VBAR} Coefficient  Std. err.      t    P>|t|     [95\% conf. interval]
\HLI{13}{\PLUS}\HLI{64}
        male {\VBAR}   .0262407   .0100894     2.60   0.009     .0064486    .0460328
      parity {\VBAR}   .0147292   .0056646     2.60   0.009     .0036171    .0258414
     lfaminc {\VBAR}   .0180498   .0055837     3.23   0.001     .0070964    .0290032
       packs {\VBAR}  -.0837281   .0171209    -4.89   0.000    -.1173139   -.0501423
       _cons {\VBAR}   4.675618   .0218813   213.68   0.000     4.632694    4.718542
\HLI{13}{\BOTT}\HLI{64}
{\smallskip}
\end{stlog}

\item[d.] Estimate the reduced form for $packs$. What do you conclude about identification of equation \eqref{5.54} using $cigprice$ as an instrument for $packs$? What bearing does this conclusion have on your answer from part c?
\\ Answer:\\
Reduce form Regression for $packs$\\
\begin{stlog}. /***
> Reduce form Regression for $packs$\\\\
> ***/
. reg packs male parity lfaminc cigprice
{\smallskip}
      Source {\VBAR}       SS           df       MS      Number of obs   =     1,388
\HLI{13}{\PLUS}\HLI{34}   F(4, 1383)      =     10.86
       Model {\VBAR}  3.76705108         4   .94176277   Prob > F        =    0.0000
    Residual {\VBAR}  119.929078     1,383  .086716615   R-squared       =    0.0305
\HLI{13}{\PLUS}\HLI{34}   Adj R-squared   =    0.0276
       Total {\VBAR}  123.696129     1,387  .089182501   Root MSE        =    .29448
{\smallskip}
\HLI{13}{\TOPT}\HLI{64}
       packs {\VBAR} Coefficient  Std. err.      t    P>|t|     [95\% conf. interval]
\HLI{13}{\PLUS}\HLI{64}
        male {\VBAR}  -.0047261   .0158539    -0.30   0.766    -.0358264    .0263742
      parity {\VBAR}   .0181491   .0088802     2.04   0.041     .0007291    .0355692
     lfaminc {\VBAR}  -.0526374   .0086991    -6.05   0.000    -.0697023   -.0355724
    cigprice {\VBAR}    .000777   .0007763     1.00   0.317    -.0007459    .0022999
       _cons {\VBAR}   .1374075   .1040005     1.32   0.187    -.0666084    .3414234
\HLI{13}{\BOTT}\HLI{64}
{\smallskip}
\end{stlog}

\end{enumerate}

\subsection*{Problem 5.5}
One occasionally sees the following reasoning used in applied work for choosing instrumental variables in the context of omitted variables. The model is
\begin{align*}
    y_1=\textbf{z}_1\pmb{\delta}_1+\alpha_1y_2+\gamma q+a_1.
\end{align*}
where $q$ is the omitted factor. We assume that $a_1$ satisfies the structural error assumption $\E(a_1|\textbf{z}_1,y_2,q)=0$, that $\textbf{z}_1$ is exogenous in the sense that $\E(q|\textbf{z}_1)=0,$ but that $y_2$ and $q$ may be correlated. Let $\textbf{z}_2$ be a vector of instrumental variable candidates for $y_2$. Suppose it is known that $\textbf{z}_2$ appears in the linear projection of $y_2$ onto $(\textbf{z}_1,\textbf{z}_2)$, and so the requirement that $\textbf{z}_2$ be partially correlated with $y_2$ is satisfied. Also, we are willing to assume that $\textbf{z}_2$ is redundant in the structural equation, so that $a_1$ is uncorrelated with $\textbf{z}_2$. What we are unsure of is whether $\textbf{z}_2$ is correlated with the omitted variable $q$, in which case $\textbf{z}_2$ would not contain valid IVs.
To ``test" whether $\textbf{z}_2$ is in fact uncorrelated with $q$, it has been suggested to use OLS on the equation
\begin{align}
    y=\textbf{z}_1\pmb{\delta}_1+\alpha_1y_2+\textbf{z}_2\pmb{\psi}_1+u_1, \tag{5.55}\label{5.55}
\end{align}
where $u_1=\gamma q+a_1,$ and test $\text{H}_0:\pmb{\psi}_1=0.$ Why does this method not work?
\\ Answer:\\

\section*{Non-textbook Problem}
Show that IV estimation can be implemented as the 2SLS procedure.
\\ Answer:\\

\end{document}