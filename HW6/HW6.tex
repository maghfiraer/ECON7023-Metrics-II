\documentclass[10pt]{article}
 
\usepackage[margin=1in]{geometry} 
\usepackage{amsmath,amsthm,amssymb,amsfonts, graphicx, multicol, array}
\usepackage{mathtools}
\usepackage{booktabs}
\usepackage{stata/stata}
\usepackage{wrapfig}
\usepackage{enumitem}
\usepackage{hyperref}

\graphicspath{ {images/} }

\newcommand\iid{\stackrel{\mathclap{iid}}{\sim}}
\newcommand\asym{\stackrel{\mathclap{a}}{\sim}}
\newcommand\convprob{\xrightarrow{p}}
\newcommand\convdist{\xrightarrow{d}}
\newcommand{\N}{\mathbb{N}}
\newcommand{\Z}{\mathbb{Z}}
\newcommand{\E}{\text{E}}
\newcommand{\V}{\text{Var}}
\newcommand{\Av}{\text{Avar}}
\newcommand{\se}{\text{se}}
\newcommand{\corr}{\text{Corr}}
\newcommand{\cov}{\text{Cov}}
\newcommand{\norm}{\text{Normal}}
\newcommand{\indep}{\perp \!\!\! \perp}
\newcommand{\Hy}{\text{H}}

 
\newenvironment{problem}[2][Problem]{\begin{trivlist}
\item[\hskip \labelsep {\bfseries #1}\hskip \labelsep {\bfseries #2.}]}{\end{trivlist}}

\begin{document}
 
\title{Homework 6}
\author{ECON 7023: Econometrics II\\
Maghfira Ramadhani\\
April 18, 2023}
\date{Spring 2023}
\maketitle

\section*{Chapter 11}
\subsection*{Problem 11.2}
Consider the following unobserved components model:
\[y_{it}=\textbf{z}_{it}\pmb{\gamma}+\delta w_{it}+c_i+u_{it},\ \ \ \ t=1,2,\ldots,T,\]
where $\textbf{z}_{it}$ is a $1\times K$ vector of time-varying variables (which could include time-period dummies), $w_{it}$ is a time-varying scalar, $c_i$ is a time-constant unobserved effect, and $u_{it}$ is the idiosyncratic error. The $\textbf{z}_{it}$ are strictly exogenous in the sense that
\[\E(\textbf{z}_{is}'u_{it})=0,\ \ \ \ \text{all }s,t=1,2,\ldots,T,\tag{11.99}\label{11.99}\]
but $c_i$ is allowed to be arbitrarily correlated with each $\textbf{z}_{it}$. The variable $w_{it}$ is endogenous in the sense that it can be correlated with $u_{it}$ (as well as with $c_i$).
\begin{enumerate}[label=\alph*.]
\item Suppose that $T=2$, and that assumption \eqref{11.99} contains the only available orthogonality conditions. What are the properties of the OLS estimator of $\pmb{\gamma}$ and $\delta$ on the differenced data? Support your claim (but do not include asymptotic derivations).
\\ \textit{Answer:}\\
The OLS estimates for the differenced equation will not be consistent because $w_{it}$ is correlated with $u_{it}$ which means $\Delta w_{it}$ is correlated with $\Delta u_{it}$.

\item Under assumption \eqref{11.99}, still with $T=2$, write the linear reduced form for the difference $\Delta w_i$ as $\Delta w_i=\textbf{z}_{i1}\pmb{\pi}_1+\textbf{z}_{i2}\pmb{\pi}_2+r_i$ where, by construction, $r_i$ is uncorrelated with both $\textbf{z}_{i1}$ and $\textbf{z}_{i2}$. What condition on $(\pmb{\pi}_1,\pmb{\pi}_2)$ is needed to identify $\pmb{\gamma}$ and $\delta$? (Hint: It is useful to rewrite the reduced form of $\Delta w_i$ in terms of $\Delta \textbf{z}_i$ and, say, $\textbf{z}_{i1}$.) How can you test this condition?
\\ \textit{Answer:}\\
From assumption \eqref{11.99} we have that $u_{i1}$, $u_{i1}$ are not correlated with $\textbf{z}_{i1},\textbf{z}_{i2}$. Thus, we can all the variable in the following differenced equation is exogenous.
\[\Delta y_i=\Delta \textbf{z}_i\pmb{\gamma}+\delta \Delta w_i+\Delta u_i.\]
Write $\Delta w_i$ as linear projection of $\textbf{z}_{i1}, \textbf{z}_{i2}$:
\[\Delta w_i=\textbf{z}_{i1}\pmb{\pi}_1+ \textbf{z}_{i2}\pmb{\pi}_2+v_i\Leftrightarrow \Delta w_i=\textbf{z}_{i1}(\pmb{\pi}_1-\pmb{\pi}_2)+ \Delta \textbf{z}_{i}\pmb{\pi}_2+v_i. \tag{1}\label{h6.1}\]
We can test by running regression for equation \eqref{h6.1}, if the coefficient on $\textbf{z}_{i1}$ is zero then the reduced form only depends on $\Delta \textbf{z}_{i}$. However, in this condition, we can not use $\Delta \textbf{z}_{i}$ as the instrument alone as it is already in  \eqref{h6.1}. Thus we need both $\Delta \textbf{z}_{i}, textbf{z}_{i1}$ as instrument.

\item Now consider the general $T$ case, where we add to assumption \eqref{11.99} the assumption $\E(w_{is}u_{it})=0,s<t$, so that previous values of $w_{it}$ are uncorrelated with $u_{it}$. Explain carefully, including equations where appropriate, how would you estimate $\pmb{\gamma}$ and $\delta$?
\\ \textit{Answer:}\\
Write the differenced equation as follow.
\[\Delta y_{it}=\Delta \textbf{z}_{it}\pmb{\gamma}+\delta \Delta w_{it}+\Delta u_{it}.\]
We have an additional assumption that $w_{is}$ is not correlated with $u_{it}$ for $s<t$. Thus we have instruments for $\Delta w_{it}$ at time $t$, that are $w_{i,t-2},\ldots,w_{i,1}$. Thus we need at least $T=3$. When $T=3$ we use the following equation to estimate:
\[\Delta y_{i3}=\Delta \textbf{z}_{i3}\pmb{\gamma}+\delta \underbrace{\Delta w_{i3}}_{\displaystyle \text{IV: } w_{i1}}+\Delta u_{i3}.\]


\item Again consider the general $T$ case, but now use the fixed effects transformation to eliminate $c_i$:
\[\ddot{y}_{it}=\ddot{\textbf{z}}_{it}\pmb{\gamma}+\delta \ddot{w}_{it}+\ddot{u}_it.\]
What are the properties of the IV estimators if you use $\ddot{\textbf{z}}_{it}$ and $w_{i,t-p},p\geq 1,$ as instruments in estimating this equation by pooled IV? (You can only use time periods $p+1,\ldots,T$ after the initial demeaning.)
\\ \textit{Answer:}\\
Using an IV estimator for fixed effect transformation will be inconsistent because then by time demeaning we will include errors from all time periods in the transformed error, $\ddot{u}_{it}$. This means in almost all periods, $\ddot{u}_{it}$ is correlated with $w_{it}$.
\end{enumerate}


\subsection*{Problem 11.9}
Consider model (11.1) under Assumptions FEIV.1 and FEIV.2.
\begin{enumerate}[label=\alph*.]
\item Show that, under the additional Assumption FEIV.3, the asymptotic variance of $\sqrt{N}(\hat{\pmb{\beta}}-{\pmb{\beta}})$ is $\sigma_u^2\{\E(\ddot{\textbf{X}}_i'\ddot{\textbf{Z}}_i)[\E(\ddot{\textbf{Z}}_i\ddot{\textbf{Z}}_i)]^{-1}\E(\ddot{\textbf{Z}}_i'\ddot{\textbf{X}}_i)\}^{-1}$.
\\ \textit{Answer:}\\
Recall the GMM estimation results
\[\Av\sqrt{N}(\hat{\pmb{\beta}}-{\pmb{\beta}})=(\textbf{C'WC})^{-1}\textbf{C'W}\pmb{\Lambda}\textbf{WC}(\textbf{C'WC})^{-1}.\]
In this case, we use all the time demeaned variable, so $\textbf{C}=\E(\ddot{\textbf{X}}_i'\ddot{\textbf{Z}}_i),\textbf{W}=[\E(\ddot{\textbf{Z}}_i'\ddot{\textbf{Z}}_i)]^{-1},\pmb{\Lambda}=\E(\ddot{\textbf{Z}}_i'\ddot{\textbf{u}}_i\ddot{\textbf{u}}_i'\ddot{\textbf{Z}}_i)$. Under FEIV.3. we have $\E(\ddot{\textbf{u}}_i\ddot{\textbf{u}}_i'|\ddot{\textbf{Z}}_i)=\sigma_u^2 \textbf{I}_T$. Thus we can directly obtain
\[\Av\sqrt{N}(\hat{\pmb{\beta}}-{\pmb{\beta}})=\sigma_u^2\{\E(\ddot{\textbf{X}}_i'\ddot{\textbf{Z}}_i)[\E(\ddot{\textbf{Z}}_i\ddot{\textbf{Z}}_i)]^{-1}\E(\ddot{\textbf{Z}}_i'\ddot{\textbf{X}}_i)\}^{-1}.\]

\item Propose a consistent estimator of $\sigma_u^2$.
\\ \textit{Answer:}\\
From FE method, we know that $\sum_{i=1}^T\E(\ddot{u}_{it})=\sigma_u^2(T-1)$. We can used the pooled 2SLS to the time-demeaned data and get the residual, $\hat{\ddot{u}}_{it}=\ddot{y}_{it}-\ddot{\textbf{x}}_{it}\hat{\pmb{\beta}}.$ Then a consistent estimator of $\sigma_u^2$ will be $SSR/(N(T-1)-K).$

\item Show that the 2SLS estimator of $\pmb{\beta}$ from part a can be obtained by means of a dummy variable approach: estimate
\[y_{it}=c_1d1_i+\cdots+c_NdN_i+\textbf{x}_{it}\pmb{\beta}+u_{it},\]
by P2SLS, using instruments $(d1_i,d2_i,\ldots,dN_i,\textbf{z}_{it})$.(Hint: Use the obvious extension of Problem 5.1 to P2SLS, and repeatedly apply the algebra of partial regression.) This is another case where, even though we cannot estimate the $c_i$ consistently with fixed $T$, we still get a consistent estimator of $\pmb{\beta}$.
\\ \textit{Answer:}\\
The 2SLS:
\begin{enumerate}
\item Regress $\textbf{x}_{it}$ on $d1_i,\ldots,dN_i,\textbf{z}_{it}$ across $i,t$ and save the residual $\hat{\textbf{r}}_{it}.$
\item Get $\hat{c}_1,\ldots,\hat{c}_N,\pmb{\beta}$ from pooled regression $y_{it}$ on $d1_i,\ldots,dN_i,\textbf{x}_{it},\hat{\textbf{r}}_{it}$
\end{enumerate}
The P2SLS:
\begin{enumerate}
\item Regress $\ddot{\textbf{x}}_{it}$ on $\ddot{\textbf{z}}_{it}$ and save the residual $\hat{\textbf{s}}_{it}.$
\item Regress $y_{it}$ on $\ddot{\textbf{x}}_{it},\hat{\textbf{s}}_{it}$
\end{enumerate}
Both will have the same estimates of $\pmb{\beta}$.

\item In using the 2SLS approach from part c, explain why the usually reported standard errors are valid under Assumption FEIV.3.
\\\textit{Answer:}\\
When we do the two regressions in part c, the degree of freedom will be the same. In the first regression, we will have the degree of freedom to be $NT-N-K$, where the $-N$ is for the additional coefficients on the dummy variables. While as we see in part b, we have exactly the same degree of freedom.

\item How would you obtain valid standard errors for 2SLS without Assumption FEIV.3?
\\\textit{Answer:}\\
If FEIV.3 is violated we use the result from GMM:
\[\Av\sqrt{N}(\hat{\pmb{\beta}}-{\pmb{\beta}})=(\textbf{C'WC})^{-1}\textbf{C'W}\pmb{\Lambda}\textbf{WC}(\textbf{C'WC})^{-1}.\]
Using the analogy principle by setting $\hat{\textbf{C}}=\ddot{\textbf{X}}_i'\ddot{\textbf{Z}}_i,\hat{\textbf{W}}=[(\ddot{\textbf{Z}}_i'\ddot{\textbf{Z}}_i)/N]^{-1},\hat{\ddot{\textbf{x}}}_i=\hat{\ddot{\textbf{y}}}_i-\hat{\ddot{\textbf{X}}}_i\hat{\pmb{\beta}},\\ \hat{\pmb{\Lambda}}=N^{-1}\sum_{i=1}^N\ddot{\textbf{Z}}_i'\ddot{\textbf{u}}_i\ddot{\textbf{u}}_i'\ddot{\textbf{Z}}_i$.
\end{enumerate}

\subsection*{Problem 11.12}
An unobserved effects model explaining current murder rates in terms of the number of executions in the last three years is
\[mrdrte_{it}=\theta_t+\beta_1exec_{it}+\beta_2unem_{it}+c_i+u_{it},\]
where $mrdrte_{it}$ is the number of murders in state $i$ during year $t$, per 10,000 people; $exec_{it}$ is the total number of executions for the current and prior two years; and $unem_{it}$ is the current unemployment rate, included as a control.
\begin{enumerate}[label=\alph*.]
\item Using the data for 1990 and 1993 in MURDER.RAW, estimate this model by first differencing. Notice that you should allow different year intercepts.
\\ \textit{Answer:}\\
First differenced estimates

\item Under what circumstances would $exec_{it}$ not be strictly exogenous (conditional on $c_i$)? Assuming that no further lags of $exec$ appear in the model and that $unem$ is strictly exogenous, propose a method for consistently estimating $\pmb{\beta}$ when $exec$ is not strictly exogenous.
\\\textit{Answer:}\\
It will violate strict exogeneity if future execution is correlated with the all past murder rate. Otherwise, we can use $\Delta exec_{i,t-1}$ as IV for $\Delta exec_{it}$ assuming only $exec_{it}$ appear in the differenced equation at specific $t$

\item Apply the method from part b to the data in MURDER.RAW. Be sure to also test the rank condition. Do your results differ much from those in part a? 
\\\textit{Answer:}\\

\item What happens to the estimates from parts a and c if Texas is dropped from the analysis?
\\\textit{Answer:}\\
\end{enumerate}

\subsection*{Problem 11.15}
Use the data in JTRAIN1.RAW for this question.
\begin{enumerate}[label=\alph*.]
\item Consider the simple equation \[\log(scrap_{it})=\theta_t+\beta_1hrsemp_{it}+c_i+u_{it},\] where $scrap_{it}$ is the scrap rate for firm $i$ in year $t$, and $hrsemp_{it}$ is hours of training per employee. Suppose that you difference to remove $c_i$, but you still think that $\Delta hrsemp_{it}$ and $\Delta \log(scrap_{it})$ are simultaneously determined. Under what assumption is $\Delta grant_{it}$ a valid IV for $\Delta hrsemp_{it}$?
\\ \textit{Answer:}\\
We need to assume strict exogeneity for $grant_{it}.$ This means the grant is allowed to depend on firm characteristic, $c_i$, and it is not correlated with unobserved idiosyncratic error at all times. Also, the standard IV assumption, is that grants affect the scrap rates but only through the job-training channels.

\item Using the differences from 1987 to 1988 only, test the rank condition for identification for the method described in part a.
\\ \textit{Answer:}\\

\item Estimate the FD equation by IV, and discuss the results.
\\\textit{Answer:}\\

\item Compare the IV estimates on the first differences with the OLS estimates on the first differences.
\\\textit{Answer:}\\

\item Use the IV method described in part a, but use all three years of data. How does the estimate of $\beta_1$ compare with only using two years of data? 
\\\textit{Answer:}\\

\end{enumerate}

\subsection*{Problem 11.18}
Let $\hat{\pmb{\beta}}$ be the REIV estimator.
\begin{enumerate}[label=\alph*.]
\item Derive $\Av[\sqrt{N}(\hat{\pmb{\beta}}_{REIV}-{\pmb{\beta}})]$ without Assumption REIV.3.
\\ \textit{Answer:}\\
We have
\begin{align*}
    \hat{\pmb{\beta}}_{REIV}=&\left[\left(\sum_{i=1}^N\textbf{X}_i'\hat{\pmb\Omega}^{-1}\textbf{Z}_i\right)
    \left(\sum_{i=1}^N\textbf{Z}_i'\hat{\pmb\Omega}^{-1}\textbf{Z}_i\right)^{-1}
    \left(\sum_{i=1}^N\textbf{Z}_i'\hat{\pmb\Omega}^{-1}\textbf{X}_i\right)
    \right]^{-1}\\&\left[\left(\sum_{i=1}^N\textbf{X}_i'\hat{\pmb\Omega}^{-1}\textbf{Z}_i\right)
    \left(\sum_{i=1}^N\textbf{Z}_i'\hat{\pmb\Omega}^{-1}\textbf{Z}_i\right)^{-1}
    \left(\sum_{i=1}^N\textbf{Z}_i'\hat{\pmb\Omega}^{-1}\textbf{y}_i\right)
    \right].
\end{align*}
Using the standard derivation for asymptotic variance, apply CLT, Slutsky's theorem principle. For notation let $\pmb{\Lambda}=plim(\hat{\pmb{\Omega}}), \textbf{C}=\E(\textbf{Z}_i'\pmb{\Lambda}^{-1}\textbf{X}_i'), \textbf{D}=\E(\textbf{Z}_i'\pmb{\Lambda}^{-1}\textbf{Z}_i), \textbf{A}=\textbf{C}'\textbf{D}^{-1}\textbf{C}.$ We will have
\[\sqrt{N}(\hat{\pmb{\beta}}_{REIV}-{\pmb{\beta}})\convdist \N(\textbf{0},\textbf{A}^{-1}\textbf{B}\textbf{A}^{-1})\]
with $\textbf{B}=\textbf{C}'\textbf{D}^{-1}\E(\textbf{Z}_i'\pmb{\Lambda}^{-1}\textbf{u}_i\textbf{u}_i'\pmb{\Lambda}^{-1}\textbf{Z}_i)\textbf{D}^{-1}\textbf{C}$.

\item Show how to consistently estimate the asymptotic variance in part a.
\\ \textit{Answer:}\\
We can use analogy principle by setting \begin{align*}
    &\hat{\textbf{C}}=N^{-1}\sum_{i=1}^N\textbf{Z}_i'\hat{\pmb{\Omega}}^{-1}\textbf{X}_i', \hat{\textbf{D}}=N^{-1}\sum_{i=1}^N\textbf{Z}_i'\hat{\pmb{\Omega}}^{-1}\textbf{Z}_i,\\ &\hat{\textbf{A}}=\hat{\textbf{C}}'\hat{\textbf{D}}^{-1}\hat{\textbf{C}},\\ &\hat{\textbf{B}}=\hat{\textbf{C}}'\hat{\textbf{D}}^{-1}\left(N^{-1}\sum_{i=1}^N\textbf{Z}_i'\hat{\pmb{\Omega}}^{-1}\hat{\textbf{u}}_i\hat{\textbf{u}}_i'\hat{\pmb{\Omega}}^{-1}\textbf{Z}_i\right)\hat{\textbf{D}}^{-1}\hat{\textbf{C}}
\end{align*} with
$\hat{\textbf{u}}_i=\textbf{y}_i-\textbf{X}_i\hat{\pmb{\beta}}_{REIV}$.
\end{enumerate}

\subsection*{Problem 11.19}
Use the data in AIRFARE.RAW for this exercise.
\begin{enumerate}[label=\alph*.]
\item Estimate the reduced forms underlying the REIV and FEIV analysis in Example 11.1. Using fully robust $t$ statistics, is $concen$ sufficiently (partially) correlated with $lfare$?
\\ \textit{Answer:}\\

\item Redo the REIV estimation, but drop the route distance variables. What happens to the estimated elasticity of passenger demand with respect to $fare$?
\\ \textit{Answer:}\\

\item Now consider a model where the elasticity can depend on route distance:
\begin{align*}
    lpassen_{it}=&\theta_{t1}+\alpha_1 lfrare_{it}+\delta_1 ldist_i+\delta_2 ldist_i^2+\gamma_1(ldist_i-\mu_1)lfare_{it}\\
    &+\gamma_2(ldist_i^2-\mu_2)lfare_{it}+c_{i1}+u_{it1},
\end{align*}
where $\mu_1=\E(ldist_i)$ and $\mu_2=\E(ldist_i^2)$. The means are subtracted before forming the interactions so that $\alpha_1$ is the average partial effect. In using REIV or FEIV to estimate this model, what should be the IVs for the interaction terms?
\\ \textit{Answer:}\\
The endogenous variables are $lfare,\ (ldist-\mu_1)lfare,\ (ldist^2-\mu_2)lfare$ since $lfare$ is endogenous. If $concen$ partially correlated with $lfare$, we can use $concen,\ (ldist-\mu_1)concen,\ (ldist^2-\mu_2)concen$ as IVs.

\item Use the data in AIRFARE.RAW to estimate the model in part c, replacing $\mu_1$ and $\mu_2$ with their sample averages. How do the REIV and FEIV estimates of $\alpha_1$ compare with the estimates in Table 11.1?
\\ \textit{Answer:}\\

\item Obtain fully robust standard errors for the FEIV estimation and obtain a fully robust test of joint significance of the interaction terms. (Ignore the estimation of $\mu_1$ and $\mu_2$.) What is the robust 95 percent confidence interval for $\alpha_1$?
\\ \textit{Answer:}\\

\item Find the estimated elasticities for $dist=500$ ad $dist=1,500.$ What do you conclude?
\end{enumerate}
\\ \textit{Answer:}\\

\end{document}