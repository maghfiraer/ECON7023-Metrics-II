\documentclass[10pt]{article}
 
\usepackage[margin=1in]{geometry} 
\usepackage{amsmath,amsthm,amssymb,amsfonts, graphicx, multicol, array}
\usepackage{mathtools}
\usepackage{booktabs}
\usepackage{stata/stata}
\usepackage{wrapfig}
\usepackage{enumitem}
\usepackage{hyperref}

\graphicspath{ {images/} }

\newcommand\iid{\stackrel{\mathclap{iid}}{\sim}}
\newcommand\asym{\stackrel{\mathclap{a}}{\sim}}
\newcommand\convprob{\xrightarrow{p}}
\newcommand\convdist{\xrightarrow{d}}
\newcommand{\N}{\mathbb{N}}
\newcommand{\Z}{\mathbb{Z}}
\newcommand{\E}{\text{E}}
\newcommand{\V}{\text{Var}}
\newcommand{\Av}{\text{Avar}}
\newcommand{\se}{\text{se}}
\newcommand{\corr}{\text{Corr}}
\newcommand{\cov}{\text{Cov}}
\newcommand{\norm}{\text{Normal}}
\newcommand{\indep}{\perp \!\!\! \perp}
\newcommand{\Hy}{\text{H}}

 
\newenvironment{problem}[2][Problem]{\begin{trivlist}
\item[\hskip \labelsep {\bfseries #1}\hskip \labelsep {\bfseries #2.}]}{\end{trivlist}}

\begin{document}
 
\title{Homework 6}
\author{ECON 7023: Econometrics II\\
Maghfira Ramadhani\\
April 18, 2023}
\date{Spring 2023}
\maketitle

\section*{Chapter 11}
\subsection*{Problem 11.2}
Consider the following unobserved components model:
\[y_{it}=\textbf{z}_{it}\pmb{\gamma}+\delta w_{it}+c_i+u_{it},\ \ \ \ t=1,2,\ldots,T,\]
where $\textbf{z}_{it}$ is a $1\times K$ vector of time-varying variables (which could include time-period dummies), $w_{it}$ is a time-varying scalar, $c_i$ is a time-constant unobserved effect, and $u_{it}$ is the idiosyncratic error. The $\textbf{z}_{it}$ are strictly exogenous in the sense that
\[\E(\textbf{z}_{is}'u_{it})=0,\ \ \ \ \text{all }s,t=1,2,\ldots,T,\tag{11.99}\label{11.99}\]
but $c_i$ is allowed to be arbitrarily correlated with each $\textbf{z}_{it}$. The variable $w_{it}$ is endogenous in the sense that it can be correlated with $u_{it}$ (as well as with $c_i$).
\begin{enumerate}[label=\alph*.]
\item Suppose that $T=2$, and that assumption \eqref{11.99} contains the only available orthogonality conditions. What are the properties of the OLS estimator of $\pmb{\gamma}$ and $\delta$ on the differenced data? Support your claim (but do not include asymptotic derivations).
\\ \textit{Answer:}\\
The OLS estimates for the differenced equation will not be consistent because $w_{it}$ is correlated with $u_{it}$ which means $\Delta w_{it}$ is correlated with $\Delta u_{it}$.

\item Under assumption \eqref{11.99}, still with $T=2$, write the linear reduced form for the difference $\Delta w_i$ as $\Delta w_i=\textbf{z}_{i1}\pmb{\pi}_1+\textbf{z}_{i2}\pmb{\pi}_2+r_i$ where, by construction, $r_i$ is uncorrelated with both $\textbf{z}_{i1}$ and $\textbf{z}_{i2}$. What condition on $(\pmb{\pi}_1,\pmb{\pi}_2)$ is needed to identify $\pmb{\gamma}$ and $\delta$? (Hint: It is useful to rewrite the reduced form of $\Delta w_i$ in terms of $\Delta \textbf{z}_i$ and, say, $\textbf{z}_{i1}$.) How can you test this condition?
\\ \textit{Answer:}\\
From assumption \eqref{11.99} we have that $u_{i1}$, $u_{i1}$ are not correlated with $\textbf{z}_{i1},\textbf{z}_{i2}$. Thus, we can all the variable in the following differenced equation is exogenous.
\[\Delta y_i=\Delta \textbf{z}_i\pmb{\gamma}+\delta \Delta w_i+\Delta u_i.\]
Write $\Delta w_i$ as linear projection of $\textbf{z}_{i1}, \textbf{z}_{i2}$:
\[\Delta w_i=\textbf{z}_{i1}\pmb{\pi}_1+ \textbf{z}_{i2}\pmb{\pi}_2+v_i\Leftrightarrow \Delta w_i=\textbf{z}_{i1}(\pmb{\pi}_1-\pmb{\pi}_2)+ \Delta \textbf{z}_{i}\pmb{\pi}_2+v_i. \tag{1}\label{h6.1}\]
We can test by running regression for equation \eqref{h6.1}, if the coefficient on $\textbf{z}_{i1}$ is zero then the reduced form only depends on $\Delta \textbf{z}_{i}$. However, in this condition, we can not use $\Delta \textbf{z}_{i}$ as the instrument alone as it is already in  \eqref{h6.1}. Thus we need both $\Delta \textbf{z}_{i}, textbf{z}_{i1}$ as instrument.

\item Now consider the general $T$ case, where we add to assumption \eqref{11.99} the assumption $\E(w_{is}u_{it})=0,s<t$, so that previous values of $w_{it}$ are uncorrelated with $u_{it}$. Explain carefully, including equations where appropriate, how would you estimate $\pmb{\gamma}$ and $\delta$?
\\ \textit{Answer:}\\
Write the differenced equation as follow.
\[\Delta y_{it}=\Delta \textbf{z}_{it}\pmb{\gamma}+\delta \Delta w_{it}+\Delta u_{it}.\]
We have an additional assumption that $w_{is}$ is not correlated with $u_{it}$ for $s<t$. Thus we have instruments for $\Delta w_{it}$ at time $t$, that are $w_{i,t-2},\ldots,w_{i,1}$. Thus we need at least $T=3$. When $T=3$ we use the following equation to estimate:
\[\Delta y_{i3}=\Delta \textbf{z}_{i3}\pmb{\gamma}+\delta \underbrace{\Delta w_{i3}}_{\displaystyle \text{IV: } w_{i1}}+\Delta u_{i3}.\]


\item Again consider the general $T$ case, but now use the fixed effects transformation to eliminate $c_i$:
\[\ddot{y}_{it}=\ddot{\textbf{z}}_{it}\pmb{\gamma}+\delta \ddot{w}_{it}+\ddot{u}_it.\]
What are the properties of the IV estimators if you use $\ddot{\textbf{z}}_{it}$ and $w_{i,t-p},p\geq 1,$ as instruments in estimating this equation by pooled IV? (You can only use time periods $p+1,\ldots,T$ after the initial demeaning.)
\\ \textit{Answer:}\\
Using an IV estimator for fixed effect transformation will be inconsistent because then by time demeaning we will include errors from all time periods in the transformed error, $\ddot{u}_{it}$. This means in almost all periods, $\ddot{u}_{it}$ is correlated with $w_{it}$.
\end{enumerate}


\subsection*{Problem 11.9}
Consider model (11.1) under Assumptions FEIV.1 and FEIV.2.
\begin{enumerate}[label=\alph*.]
\item Show that, under the additional Assumption FEIV.3, the asymptotic variance of $\sqrt{N}(\hat{\pmb{\beta}}-{\pmb{\beta}})$ is $\sigma_u^2\{\E(\ddot{\textbf{X}}_i'\ddot{\textbf{Z}}_i)[\E(\ddot{\textbf{Z}}_i\ddot{\textbf{Z}}_i)]^{-1}\E(\ddot{\textbf{Z}}_i'\ddot{\textbf{X}}_i)\}^{-1}$.
\\ \textit{Answer:}\\
Recall the GMM estimation results
\[\Av\sqrt{N}(\hat{\pmb{\beta}}-{\pmb{\beta}})=(\textbf{C'WC})^{-1}\textbf{C'W}\pmb{\Lambda}\textbf{WC}(\textbf{C'WC})^{-1}.\]
In this case, we use all the time demeaned variable, so $\textbf{C}=\E(\ddot{\textbf{X}}_i'\ddot{\textbf{Z}}_i),\textbf{W}=[\E(\ddot{\textbf{Z}}_i'\ddot{\textbf{Z}}_i)]^{-1},\pmb{\Lambda}=\E(\ddot{\textbf{Z}}_i'\ddot{\textbf{u}}_i\ddot{\textbf{u}}_i'\ddot{\textbf{Z}}_i)$. Under FEIV.3. we have $\E(\ddot{\textbf{u}}_i\ddot{\textbf{u}}_i'|\ddot{\textbf{Z}}_i)=\sigma_u^2 \textbf{I}_T$. Thus we can directly obtain
\[\Av\sqrt{N}(\hat{\pmb{\beta}}-{\pmb{\beta}})=\sigma_u^2\{\E(\ddot{\textbf{X}}_i'\ddot{\textbf{Z}}_i)[\E(\ddot{\textbf{Z}}_i\ddot{\textbf{Z}}_i)]^{-1}\E(\ddot{\textbf{Z}}_i'\ddot{\textbf{X}}_i)\}^{-1}.\]

\item Propose a consistent estimator of $\sigma_u^2$.
\\ \textit{Answer:}\\
From FE method, we know that $\sum_{i=1}^T\E(\ddot{u}_{it})=\sigma_u^2(T-1)$. We can used the pooled 2SLS to the time-demeaned data and get the residual, $\hat{\ddot{u}}_{it}=\ddot{y}_{it}-\ddot{\textbf{x}}_{it}\hat{\pmb{\beta}}.$ Then a consistent estimator of $\sigma_u^2$ will be $SSR/(N(T-1)-K).$

\item Show that the 2SLS estimator of $\pmb{\beta}$ from part a can be obtained by means of a dummy variable approach: estimate
\[y_{it}=c_1d1_i+\cdots+c_NdN_i+\textbf{x}_{it}\pmb{\beta}+u_{it},\]
by P2SLS, using instruments $(d1_i,d2_i,\ldots,dN_i,\textbf{z}_{it})$.(Hint: Use the obvious extension of Problem 5.1 to P2SLS, and repeatedly apply the algebra of partial regression.) This is another case where, even though we cannot estimate the $c_i$ consistently with fixed $T$, we still get a consistent estimator of $\pmb{\beta}$.
\\ \textit{Answer:}\\
The 2SLS:
\begin{enumerate}
\item Regress $\textbf{x}_{it}$ on $d1_i,\ldots,dN_i,\textbf{z}_{it}$ across $i,t$ and save the residual $\hat{\textbf{r}}_{it}.$
\item Get $\hat{c}_1,\ldots,\hat{c}_N,\pmb{\beta}$ from pooled regression $y_{it}$ on $d1_i,\ldots,dN_i,\textbf{x}_{it},\hat{\textbf{r}}_{it}$
\end{enumerate}
The P2SLS:
\begin{enumerate}
\item Regress $\ddot{\textbf{x}}_{it}$ on $\ddot{\textbf{z}}_{it}$ and save the residual $\hat{\textbf{s}}_{it}.$
\item Regress $y_{it}$ on $\ddot{\textbf{x}}_{it},\hat{\textbf{s}}_{it}$
\end{enumerate}
Both will have the same estimates of $\pmb{\beta}$.

\item In using the 2SLS approach from part c, explain why the usually reported standard errors are valid under Assumption FEIV.3.
\\\textit{Answer:}\\
When we do the two regressions in part c, the degree of freedom will be the same. In the first regression, we will have the degree of freedom to be $NT-N-K$, where the $-N$ is for the additional coefficients on the dummy variables. While as we see in part b, we have exactly the same degree of freedom.

\item How would you obtain valid standard errors for 2SLS without Assumption FEIV.3?
\\\textit{Answer:}\\
If FEIV.3 is violated we use the result from GMM:
\[\Av\sqrt{N}(\hat{\pmb{\beta}}-{\pmb{\beta}})=(\textbf{C'WC})^{-1}\textbf{C'W}\pmb{\Lambda}\textbf{WC}(\textbf{C'WC})^{-1}.\]
Using the analogy principle by setting $\hat{\textbf{C}}=\ddot{\textbf{X}}_i'\ddot{\textbf{Z}}_i,\hat{\textbf{W}}=[(\ddot{\textbf{Z}}_i'\ddot{\textbf{Z}}_i)/N]^{-1},\hat{\ddot{\textbf{x}}}_i=\hat{\ddot{\textbf{y}}}_i-\hat{\ddot{\textbf{X}}}_i\hat{\pmb{\beta}},\\ \hat{\pmb{\Lambda}}=N^{-1}\sum_{i=1}^N\ddot{\textbf{Z}}_i'\ddot{\textbf{u}}_i\ddot{\textbf{u}}_i'\ddot{\textbf{Z}}_i$.
\end{enumerate}

\subsection*{Problem 11.12}
An unobserved effects model explaining current murder rates in terms of the number of executions in the last three years is
\[mrdrte_{it}=\theta_t+\beta_1exec_{it}+\beta_2unem_{it}+c_i+u_{it},\]
where $mrdrte_{it}$ is the number of murders in state $i$ during year $t$, per 10,000 people; $exec_{it}$ is the total number of executions for the current and prior two years; and $unem_{it}$ is the current unemployment rate, included as a control.
\begin{enumerate}[label=\alph*.]
\item Using the data for 1990 and 1993 in MURDER.RAW, estimate this model by first differencing. Notice that you should allow different year intercepts.
\\ \textit{Answer:}\\
FD Estimate
\begin{stlog}. reg cmrdrte cexec cunem if d93
{\smallskip}
      Source {\VBAR}       SS           df       MS      Number of obs   =        51
\HLI{13}{\PLUS}\HLI{34}   F(2, 48)        =      2.96
       Model {\VBAR}   6.8879023         2  3.44395115   Prob > F        =    0.0614
    Residual {\VBAR}  55.8724857        48  1.16401012   R-squared       =    0.1097
\HLI{13}{\PLUS}\HLI{34}   Adj R-squared   =    0.0727
       Total {\VBAR}   62.760388        50  1.25520776   Root MSE        =    1.0789
{\smallskip}
\HLI{13}{\TOPT}\HLI{64}
     cmrdrte {\VBAR}      Coef.   Std. Err.      t    P>|t|     [95\% Conf. Interval]
\HLI{13}{\PLUS}\HLI{64}
       cexec {\VBAR}  -.1038396   .0434139    -2.39   0.021    -.1911292     -.01655
       cunem {\VBAR}  -.0665914   .1586859    -0.42   0.677    -.3856509     .252468
       _cons {\VBAR}   .4132665   .2093848     1.97   0.054    -.0077298    .8342628
\HLI{13}{\BOTT}\HLI{64}
{\smallskip}
\end{stlog}


\item Under what circumstances would $exec_{it}$ not be strictly exogenous (conditional on $c_i$)? Assuming that no further lags of $exec$ appear in the model and that $unem$ is strictly exogenous, propose a method for consistently estimating $\pmb{\beta}$ when $exec$ is not strictly exogenous.
\\\textit{Answer:}\\
It will violate strict exogeneity if future execution is correlated with the all past murder rate. Otherwise, we can use $\Delta exec_{i,t-1}$ as IV for $\Delta exec_{it}$ assuming only $exec_{it}$ appear in the differenced equation at specific $t$

\item Apply the method from part b to the data in MURDER.RAW. Be sure to also test the rank condition. Do your results differ much from those in part a? 
\\\textit{Answer:}\\
Testing Rank Condition
\begin{stlog}. reg cmrdrte cexec_1 cunem if d93
{\smallskip}
      Source {\VBAR}       SS           df       MS      Number of obs   =        51
\HLI{13}{\PLUS}\HLI{34}   F(2, 48)        =      1.23
       Model {\VBAR}   3.0478366         2   1.5239183   Prob > F        =    0.3028
    Residual {\VBAR}  59.7125514        48  1.24401149   R-squared       =    0.0486
\HLI{13}{\PLUS}\HLI{34}   Adj R-squared   =    0.0089
       Total {\VBAR}   62.760388        50  1.25520776   Root MSE        =    1.1154
{\smallskip}
\HLI{13}{\TOPT}\HLI{64}
     cmrdrte {\VBAR}      Coef.   Std. Err.      t    P>|t|     [95\% Conf. Interval]
\HLI{13}{\PLUS}\HLI{64}
     cexec_1 {\VBAR}   .1083461   .0719723     1.51   0.139    -.0363639    .2530561
       cunem {\VBAR}   -.070735   .1640407    -0.43   0.668    -.4005611     .259091
       _cons {\VBAR}   .3795394   .2156246     1.76   0.085     -.054003    .8130817
\HLI{13}{\BOTT}\HLI{64}
{\smallskip}
\end{stlog}
IV Estimates
\begin{stlog}. reg cmrdrte cexec cunem (cexec_1 cunem) if d93
{\smallskip}
Instrumental variables (2SLS) regression
{\smallskip}
      Source {\VBAR}       SS           df       MS      Number of obs   =        51
\HLI{13}{\PLUS}\HLI{34}   F(2, 48)        =      1.31
       Model {\VBAR}  6.87925253         2  3.43962627   Prob > F        =    0.2796
    Residual {\VBAR}  55.8811355        48  1.16419032   R-squared       =    0.1096
\HLI{13}{\PLUS}\HLI{34}   Adj R-squared   =    0.0725
       Total {\VBAR}   62.760388        50  1.25520776   Root MSE        =     1.079
{\smallskip}
\HLI{13}{\TOPT}\HLI{64}
     cmrdrte {\VBAR}      Coef.   Std. Err.      t    P>|t|     [95\% Conf. Interval]
\HLI{13}{\PLUS}\HLI{64}
       cexec {\VBAR}  -.1000972   .0643241    -1.56   0.126    -.2294293     .029235
       cunem {\VBAR}  -.0667262   .1587074    -0.42   0.676    -.3858289    .2523764
       _cons {\VBAR}    .410966   .2114237     1.94   0.058    -.0141298    .8360617
\HLI{13}{\BOTT}\HLI{64}
{\smallskip}
\end{stlog}


\item What happens to the estimates from parts a and c if Texas is dropped from the analysis?
\\\textit{Answer:}\\
Estimates if Texas is dropped out
OLS
\begin{stlog}. reg cmrdrte cexec cunem if (d93==1 \& state!="TX")
{\smallskip}
      Source {\VBAR}       SS           df       MS      Number of
>  obs   =        50
\HLI{13}{\PLUS}\HLI{34}   F(2, 47) 
>        =      0.32
       Model {\VBAR}  .755191109         2  .377595555   Prob > F 
>        =    0.7287
    Residual {\VBAR}  55.7000012        47  1.18510641   R-squared
>        =    0.0134
\HLI{13}{\PLUS}\HLI{34}   Adj R-squ
> ared   =   -0.0286
       Total {\VBAR}  56.4551923        49  1.15214678   Root MSE 
>        =    1.0886
{\smallskip}
\HLI{13}{\TOPT}\HLI{46}
\HLI{18}
     cmrdrte {\VBAR}      Coef.   Std. Err.      t    P>|t|     [9
> 5\% Con                                                    
>       f. Interval]
\HLI{13}{\PLUS}\HLI{46}
\HLI{18}
       cexec {\VBAR}   -.067471    .104913    -0.64   0.523    -.2
> 785288                                                    
>           .1435868
       cunem {\VBAR}  -.0700316   .1603712    -0.44   0.664    -.3
> 926569                                                    
>           .2525936
       _cons {\VBAR}   .4125226   .2112827     1.95   0.057    -.0
> 125233                                                    
>           .8375686
\HLI{13}{\BOTT}\HLI{46}
\HLI{18}
{\smallskip}
\end{stlog}
IV Estimates
\begin{stlog}. reg cmrdrte cexec cunem (cexec_1 cunem) if (d93==1 \& state!="TX")
{\smallskip}
Instrumental variables (2SLS) regression
{\smallskip}
      Source {\VBAR}       SS           df       MS      Number of obs   =        50
\HLI{13}{\PLUS}\HLI{34}   F(2, 47)        =      0.11
       Model {\VBAR} -1.65785462         2 -.828927308   Prob > F        =    0.8939
    Residual {\VBAR}  58.1130469        47  1.23644781   R-squared       =         .
\HLI{13}{\PLUS}\HLI{34}   Adj R-squared   =         .
       Total {\VBAR}  56.4551923        49  1.15214678   Root MSE        =     1.112
{\smallskip}
\HLI{13}{\TOPT}\HLI{64}
     cmrdrte {\VBAR}      Coef.   Std. Err.      t    P>|t|     [95\% Conf. Interval]
\HLI{13}{\PLUS}\HLI{64}
       cexec {\VBAR}    .082233    .804114     0.10   0.919    -1.535436    1.699902
       cunem {\VBAR}  -.0826635   .1770735    -0.47   0.643    -.4388895    .2735624
       _cons {\VBAR}   .3939505   .2373797     1.66   0.104    -.0835958    .8714968
\HLI{13}{\BOTT}\HLI{64}
{\smallskip}
\end{stlog}

\end{enumerate}

\subsection*{Problem 11.15}
Use the data in JTRAIN1.RAW for this question.
\begin{enumerate}[label=\alph*.]
\item Consider the simple equation \[\log(scrap_{it})=\theta_t+\beta_1hrsemp_{it}+c_i+u_{it},\] where $scrap_{it}$ is the scrap rate for firm $i$ in year $t$, and $hrsemp_{it}$ is hours of training per employee. Suppose that you difference to remove $c_i$, but you still think that $\Delta hrsemp_{it}$ and $\Delta \log(scrap_{it})$ are simultaneously determined. Under what assumption is $\Delta grant_{it}$ a valid IV for $\Delta hrsemp_{it}$?
\\ \textit{Answer:}\\
We need to assume strict exogeneity for $grant_{it}.$ This means the grant is allowed to depend on firm characteristic, $c_i$, and it is not correlated with unobserved idiosyncratic error at all times. Also, the standard IV assumption, is that grants affect the scrap rates but only through the job-training channels.

\item Using the differences from 1987 to 1988 only, test the rank condition for identification for the method described in part a.
\\ \textit{Answer:}\\
Simple Regresison
\begin{stlog}. reg chrsemp cgrant if d88
{\smallskip}
      Source {\VBAR}       SS           df       MS      Number of
>  obs   =       125
\HLI{13}{\PLUS}\HLI{34}   F(1, 123)
>        =     79.37
       Model {\VBAR}  18117.5987         1  18117.5987   Prob > F 
>        =    0.0000
    Residual {\VBAR}  28077.3319       123  228.270991   R-squared
>        =    0.3922
\HLI{13}{\PLUS}\HLI{34}   Adj R-squ
> ared   =    0.3873
       Total {\VBAR}  46194.9306       124  372.539763   Root MSE 
>        =    15.109
{\smallskip}
\HLI{13}{\TOPT}\HLI{46}
\HLI{18}
     chrsemp {\VBAR}      Coef.   Std. Err.      t    P>|t|     [9
> 5\% Con                                                    
>       f. Interval]
\HLI{13}{\PLUS}\HLI{46}
\HLI{18}
      cgrant {\VBAR}   27.87793   3.129216     8.91   0.000     21
> .68384                                                    
>           34.07202
       _cons {\VBAR}   .5093234   1.558337     0.33   0.744     -2
> .57531                                                    
>           3.593956
\HLI{13}{\BOTT}\HLI{46}
\HLI{18}
{\smallskip}
\end{stlog}
Exluding missing value
\begin{stlog}. reg chrsemp cgrant if d88 \& clscrap{\tytilde}=.
{\smallskip}
      Source {\VBAR}       SS           df       MS      Number of obs   =        45
\HLI{13}{\PLUS}\HLI{34}   F(1, 43)        =     22.23
       Model {\VBAR}  6316.65458         1  6316.65458   Prob > F        =    0.0000
    Residual {\VBAR}  12217.3517        43  284.124457   R-squared       =    0.3408
\HLI{13}{\PLUS}\HLI{34}   Adj R-squared   =    0.3255
       Total {\VBAR}  18534.0062        44  421.227414   Root MSE        =    16.856
{\smallskip}
\HLI{13}{\TOPT}\HLI{64}
     chrsemp {\VBAR}      Coef.   Std. Err.      t    P>|t|     [95\% Conf. Interval]
\HLI{13}{\PLUS}\HLI{64}
      cgrant {\VBAR}   24.43691   5.182712     4.72   0.000     13.98498    34.88885
       _cons {\VBAR}   1.580598   3.185483     0.50   0.622     -4.84354    8.004737
\HLI{13}{\BOTT}\HLI{64}
{\smallskip}
\end{stlog}


\item Estimate the FD equation by IV, and discuss the results.
\\\textit{Answer:}\\
IV estimate
\begin{stlog}. ivreg clscrap (chrsemp = cgrant) if d88
{\smallskip}
Instrumental variables (2SLS) regression
{\smallskip}
      Source {\VBAR}       SS           df       MS      Number of obs   =        45
\HLI{13}{\PLUS}\HLI{34}   F(1, 43)        =      3.20
       Model {\VBAR}  .274951237         1  .274951237   Prob > F        =    0.0808
    Residual {\VBAR}  17.0148885        43  .395695081   R-squared       =    0.0159
\HLI{13}{\PLUS}\HLI{34}   Adj R-squared   =   -0.0070
       Total {\VBAR}  17.2898397        44  .392950903   Root MSE        =    .62904
{\smallskip}
\HLI{13}{\TOPT}\HLI{64}
     clscrap {\VBAR}      Coef.   Std. Err.      t    P>|t|     [95\% Conf. Interval]
\HLI{13}{\PLUS}\HLI{64}
     chrsemp {\VBAR}  -.0141532   .0079147    -1.79   0.081    -.0301148    .0018084
       _cons {\VBAR}  -.0326684   .1269512    -0.26   0.798    -.2886898     .223353
\HLI{13}{\BOTT}\HLI{64}
Instrumented:  chrsemp
Instruments:   cgrant
\HLI{78}
{\smallskip}
\end{stlog}


\item Compare the IV estimates on the first differences with the OLS estimates on the first differences.
\\\textit{Answer:}\\
OLS estimate
\begin{stlog}. reg clscrap chrsemp if d88
{\smallskip}
      Source {\VBAR}       SS           df       MS      Number of obs   =        45
\HLI{13}{\PLUS}\HLI{34}   F(1, 43)        =      2.84
       Model {\VBAR}  1.07071245         1  1.07071245   Prob > F        =    0.0993
    Residual {\VBAR}  16.2191273        43  .377189007   R-squared       =    0.0619
\HLI{13}{\PLUS}\HLI{34}   Adj R-squared   =    0.0401
       Total {\VBAR}  17.2898397        44  .392950903   Root MSE        =    .61416
{\smallskip}
\HLI{13}{\TOPT}\HLI{64}
     clscrap {\VBAR}      Coef.   Std. Err.      t    P>|t|     [95\% Conf. Interval]
\HLI{13}{\PLUS}\HLI{64}
     chrsemp {\VBAR}  -.0076007   .0045112    -1.68   0.099    -.0166984    .0014971
       _cons {\VBAR}  -.1035161    .103736    -1.00   0.324    -.3127197    .1056875
\HLI{13}{\BOTT}\HLI{64}
{\smallskip}
\end{stlog}


\item Use the IV method described in part a, but use all three years of data. How does the estimate of $\beta_1$ compare with only using two years of data? 
\\\textit{Answer:}\\
IV estimate
\begin{stlog}. ivreg clscrap d89 (chrsemp = cgrant)
{\smallskip}
Instrumental variables (2SLS) regression
{\smallskip}
      Source {\VBAR}       SS           df       MS      Number of
>  obs   =        91
\HLI{13}{\PLUS}\HLI{34}   F(2, 88) 
>        =      0.90
       Model {\VBAR}  .538688387         2  .269344194   Prob > F 
>        =    0.4087
    Residual {\VBAR}  33.2077492        88  .377360787   R-squared
>        =    0.0160
\HLI{13}{\PLUS}\HLI{34}   Adj R-squ
> ared   =   -0.0064
       Total {\VBAR}  33.7464376        90  .374960418   Root MSE 
>        =     .6143
{\smallskip}
\HLI{13}{\TOPT}\HLI{46}
\HLI{18}
     clscrap {\VBAR}      Coef.   Std. Err.      t    P>|t|     [9
> 5\% Con                                                    
>       f. Interval]
\HLI{13}{\PLUS}\HLI{46}
\HLI{18}
     chrsemp {\VBAR}  -.0028567   .0030577    -0.93   0.353    -.0
> 089332                                                    
>           .0032198
         d89 {\VBAR}  -.1387379   .1296916    -1.07   0.288    -.3
> 964728                                                    
>           .1189969
       _cons {\VBAR}  -.1548094   .0973592    -1.59   0.115    -.3
> 482902                                                    
>           .0386715
\HLI{13}{\BOTT}\HLI{46}
\HLI{18}
Instrumented:  chrsemp
Instruments:   d89 cgrant
\HLI{60}
{\smallskip}
\end{stlog}


\end{enumerate}

\subsection*{Problem 11.18}
Let $\hat{\pmb{\beta}}$ be the REIV estimator.
\begin{enumerate}[label=\alph*.]
\item Derive $\Av[\sqrt{N}(\hat{\pmb{\beta}}_{REIV}-{\pmb{\beta}})]$ without Assumption REIV.3.
\\ \textit{Answer:}\\
We have
\begin{align*}
    \hat{\pmb{\beta}}_{REIV}=&\left[\left(\sum_{i=1}^N\textbf{X}_i'\hat{\pmb\Omega}^{-1}\textbf{Z}_i\right)
    \left(\sum_{i=1}^N\textbf{Z}_i'\hat{\pmb\Omega}^{-1}\textbf{Z}_i\right)^{-1}
    \left(\sum_{i=1}^N\textbf{Z}_i'\hat{\pmb\Omega}^{-1}\textbf{X}_i\right)
    \right]^{-1}\\&\left[\left(\sum_{i=1}^N\textbf{X}_i'\hat{\pmb\Omega}^{-1}\textbf{Z}_i\right)
    \left(\sum_{i=1}^N\textbf{Z}_i'\hat{\pmb\Omega}^{-1}\textbf{Z}_i\right)^{-1}
    \left(\sum_{i=1}^N\textbf{Z}_i'\hat{\pmb\Omega}^{-1}\textbf{y}_i\right)
    \right].
\end{align*}
Using the standard derivation for asymptotic variance, apply CLT, Slutsky's theorem principle. For notation let $\pmb{\Lambda}=plim(\hat{\pmb{\Omega}}), \textbf{C}=\E(\textbf{Z}_i'\pmb{\Lambda}^{-1}\textbf{X}_i'), \textbf{D}=\E(\textbf{Z}_i'\pmb{\Lambda}^{-1}\textbf{Z}_i), \textbf{A}=\textbf{C}'\textbf{D}^{-1}\textbf{C}.$ We will have
\[\sqrt{N}(\hat{\pmb{\beta}}_{REIV}-{\pmb{\beta}})\convdist \N(\textbf{0},\textbf{A}^{-1}\textbf{B}\textbf{A}^{-1})\]
with $\textbf{B}=\textbf{C}'\textbf{D}^{-1}\E(\textbf{Z}_i'\pmb{\Lambda}^{-1}\textbf{u}_i\textbf{u}_i'\pmb{\Lambda}^{-1}\textbf{Z}_i)\textbf{D}^{-1}\textbf{C}$.

\item Show how to consistently estimate the asymptotic variance in part a.
\\ \textit{Answer:}\\
We can use analogy principle by setting \begin{align*}
    &\hat{\textbf{C}}=N^{-1}\sum_{i=1}^N\textbf{Z}_i'\hat{\pmb{\Omega}}^{-1}\textbf{X}_i', \hat{\textbf{D}}=N^{-1}\sum_{i=1}^N\textbf{Z}_i'\hat{\pmb{\Omega}}^{-1}\textbf{Z}_i,\\ &\hat{\textbf{A}}=\hat{\textbf{C}}'\hat{\textbf{D}}^{-1}\hat{\textbf{C}},\\ &\hat{\textbf{B}}=\hat{\textbf{C}}'\hat{\textbf{D}}^{-1}\left(N^{-1}\sum_{i=1}^N\textbf{Z}_i'\hat{\pmb{\Omega}}^{-1}\hat{\textbf{u}}_i\hat{\textbf{u}}_i'\hat{\pmb{\Omega}}^{-1}\textbf{Z}_i\right)\hat{\textbf{D}}^{-1}\hat{\textbf{C}}
\end{align*} with
$\hat{\textbf{u}}_i=\textbf{y}_i-\textbf{X}_i\hat{\pmb{\beta}}_{REIV}$.
\end{enumerate}

\subsection*{Problem 11.19}
Use the data in AIRFARE.RAW for this exercise.
\begin{enumerate}[label=\alph*.]
\item Estimate the reduced forms underlying the REIV and FEIV analysis in Example 11.1. Using fully robust $t$ statistics, is $concen$ sufficiently (partially) correlated with $lfare$?
\\ \textit{Answer:}\\
RE estimate
\begin{stlog}. xtreg lfare concen ldist ldistsq y98 y99 y00, re cluster(i
> d)
{\smallskip}
Random-effects GLS regression                   Number of ob
> s     =      4,596
Group variable: id                              Number of gr
> oups  =      1,149
{\smallskip}
R-sq:                                           Obs per grou
> p:
     within  = 0.1348                                       
>   min =          4
     between = 0.4176                                       
>   avg =        4.0
     overall = 0.4030                                       
>   max =          4
{\smallskip}
                                                Wald chi2(6)
>       =    1222.73
corr(u_i, X)   = 0 (assumed)                    Prob > chi2 
>       =     0.0000
{\smallskip}
                                 (Std. Err. adjusted for 1,1
> 49 clusters in id)
\HLI{13}{\TOPT}\HLI{46}
\HLI{18}
             {\VBAR}               Robust
       lfare {\VBAR}      Coef.   Std. Err.      z    P>|z|     [9
> 5\% Con                                                    
>       f. Interval]
\HLI{13}{\PLUS}\HLI{46}
\HLI{18}
      concen {\VBAR}   .2089935   .0422459     4.95   0.000      .
> 126193                                                    
>           .2917939
       ldist {\VBAR}  -.8520921   .2720902    -3.13   0.002    -1.
> 385379                                                    
>          -.3188051
     ldistsq {\VBAR}   .0974604   .0201417     4.84   0.000     .0
> 579833                                                    
>           .1369375
         y98 {\VBAR}   .0224743   .0041461     5.42   0.000      .
> 014348                                                    
>           .0306005
         y99 {\VBAR}   .0366898   .0051318     7.15   0.000     .0
> 266317                                                    
>            .046748
         y00 {\VBAR}    .098212   .0055241    17.78   0.000     .0
> 873849                                                    
>            .109039
       _cons {\VBAR}   6.222005   .9144067     6.80   0.000     4.
> 429801                                                    
>           8.014209
\HLI{13}{\PLUS}\HLI{46}
\HLI{18}
     sigma_u {\VBAR}  .31933841
     sigma_e {\VBAR}  .10651186
         rho {\VBAR}  .89988885   (fraction of variance due to u_i
> )
\HLI{13}{\BOTT}\HLI{46}
\HLI{18}
{\smallskip}
\end{stlog}
FE estimate
\begin{stlog}. xtreg lfare concen y98 y99 y00, fe cluster(id)
{\smallskip}
Fixed-effects (within) regression               Number of ob
> s                                                         
>       =                                                   
>              4,596
Group variable: id                              Number of gr
> oups                                                      
>       =                                                   
>              1,149
{\smallskip}
R-sq:                                           Obs per grou
> p:
     within  = 0.1352                                       
>   min                                                     
>       =                                                   
>                  4
     between = 0.0576                                       
>   avg                                                     
>       =                                                   
>                4.0
     overall = 0.0083                                       
>   max                                                     
>       =                                                   
>                  4
{\smallskip}
                                                F(4,1148)   
>       =                                                   
>             120.06
corr(u_i, Xb)  = -0.2033                        Prob > F    
>       =                                                   
>             0.0000
{\smallskip}
                                 (Std. Err. adjusted for 1,1
> 49 clusters in id)
\HLI{13}{\TOPT}\HLI{46}
\HLI{18}
             {\VBAR}               Robust
       lfare {\VBAR}      Coef.   Std. Err.      t    P>|t|     [9
> 5\% Con                                                    
>       f. Interval]
\HLI{13}{\PLUS}\HLI{46}
\HLI{18}
      concen {\VBAR}    .168859   .0494587     3.41   0.001     .0
> 718194                                                    
>           .2658985
         y98 {\VBAR}   .0228328    .004163     5.48   0.000     .0
> 146649                                                    
>           .0310007
         y99 {\VBAR}   .0363819   .0051275     7.10   0.000     .0
> 263215                                                    
>           .0464422
         y00 {\VBAR}   .0977717   .0055054    17.76   0.000     .0
> 869698                                                    
>           .1085735
       _cons {\VBAR}   4.953331   .0296765   166.91   0.000     4.
> 895104                                                    
>           5.011557
\HLI{13}{\PLUS}\HLI{46}
\HLI{18}
     sigma_u {\VBAR}  .43389176
     sigma_e {\VBAR}  .10651186
         rho {\VBAR}  .94316439   (fraction of variance due to u_i
> )
\HLI{13}{\BOTT}\HLI{46}
\HLI{18}
{\smallskip}
\end{stlog}


\item Redo the REIV estimation, but drop the route distance variables. What happens to the estimated elasticity of passenger demand with respect to $fare$?
\\ \textit{Answer:}\\
RE estimate
\begin{stlog}. xtivreg lpassen y98 y99 y00 (lfare = concen), re
{\smallskip}
G2SLS random-effects IV regression              Number of ob
> s                                                         
>       =                                                   
>              4,596
Group variable: id                              Number of gr
> oups                                                      
>       =                                                   
>              1,149
{\smallskip}
R-sq:                                           Obs per grou
> p:
     within  = 0.4327                                       
>   min                                                     
>       =                                                   
>                  4
     between = 0.0487                                       
>   avg                                                     
>       =                                                   
>                4.0
     overall = 0.0578                                       
>   max                                                     
>       =                                                   
>                  4
{\smallskip}
                                                Wald chi2(4)
>       =                                                   
>             219.33
corr(u_i, X)       = 0 (assumed)                Prob > chi2 
>       =                                                   
>             0.0000
{\smallskip}
\HLI{13}{\TOPT}\HLI{46}
\HLI{18}
     lpassen {\VBAR}      Coef.   Std. Err.      z    P>|z|     [9
> 5\% Con                                                    
>       f. Interval]
\HLI{13}{\PLUS}\HLI{46}
\HLI{18}
       lfare {\VBAR}  -.6540984   .4019123    -1.63   0.104    -1.
> 441832                                                    
>           .1336351
         y98 {\VBAR}   .0342955    .011701     2.93   0.003      .
> 011362                                                    
>            .057229
         y99 {\VBAR}   .0847852   .0154938     5.47   0.000     .0
> 544178                                                    
>           .1151525
         y00 {\VBAR}    .146605   .0390819     3.75   0.000      .
> 070006                                                    
>           .2232041
       _cons {\VBAR}    9.28363   2.032528     4.57   0.000     5.
> 299949                                                    
>           13.26731
\HLI{13}{\PLUS}\HLI{46}
\HLI{18}
     sigma_u {\VBAR}  .91384976
     sigma_e {\VBAR}  .16964171
         rho {\VBAR}   .9666879   (fraction of variance due to u_i
> )
\HLI{13}{\BOTT}\HLI{46}
\HLI{18}
{\lbr}p 0 16 -18{\rbr}Instrumented:   lfare{\lbr}p_end{\rbr}
{\lbr}p 0 16 -18{\rbr}Instruments:    y98 y99 y00 concen{\lbr}p_end{\rbr}
\HLI{60}
\HLI{18}
{\smallskip}
\end{stlog}
RE estimate+Control
\begin{stlog}. xtivreg lpassen ldist ldistsq y98 y99 y00 (lfare = concen)
> , re
{\smallskip}
G2SLS random-effects IV regression              Number of ob
> s                                                         
>       =                                                   
>              4,596
Group variable: id                              Number of gr
> oups                                                      
>       =                                                   
>              1,149
{\smallskip}
R-sq:                                           Obs per grou
> p:
     within  = 0.4075                                       
>   min                                                     
>       =                                                   
>                  4
     between = 0.0542                                       
>   avg                                                     
>       =                                                   
>                4.0
     overall = 0.0641                                       
>   max                                                     
>       =                                                   
>                  4
{\smallskip}
                                                Wald chi2(6)
>       =                                                   
>             231.10
corr(u_i, X)       = 0 (assumed)                Prob > chi2 
>       =                                                   
>             0.0000
{\smallskip}
\HLI{13}{\TOPT}\HLI{46}
\HLI{18}
     lpassen {\VBAR}      Coef.   Std. Err.      z    P>|z|     [9
> 5\% Con                                                    
>       f. Interval]
\HLI{13}{\PLUS}\HLI{46}
\HLI{18}
       lfare {\VBAR}  -.5078761    .229698    -2.21   0.027    -.9
> 580759                                                    
>          -.0576762
       ldist {\VBAR}  -1.504805   .6933146    -2.17   0.030    -2.
> 863677                                                    
>          -.1459332
     ldistsq {\VBAR}   .1176012   .0546255     2.15   0.031     .0
> 105373                                                    
>           .2246651
         y98 {\VBAR}   .0307363   .0086054     3.57   0.000     .0
> 138699                                                    
>           .0476027
         y99 {\VBAR}   .0796548     .01038     7.67   0.000     .0
> 593104                                                    
>           .0999992
         y00 {\VBAR}   .1325795   .0229831     5.77   0.000     .0
> 875335                                                    
>           .1776255
       _cons {\VBAR}   13.29643   2.626949     5.06   0.000     8.
> 147707                                                    
>           18.44516
\HLI{13}{\PLUS}\HLI{46}
\HLI{18}
     sigma_u {\VBAR}  .94920686
     sigma_e {\VBAR}  .16964171
         rho {\VBAR}  .96904799   (fraction of variance due to u_i
> )
\HLI{13}{\BOTT}\HLI{46}
\HLI{18}
{\lbr}p 0 16 -18{\rbr}Instrumented:   lfare{\lbr}p_end{\rbr}
{\lbr}p 0 16 -18{\rbr}Instruments:    ldist ldistsq y98 y99 y00 concen
> {\lbr}p_end{\rbr}
\HLI{60}
\HLI{18}
{\smallskip}
\end{stlog}


\item Now consider a model where the elasticity can depend on route distance:
\begin{align*}
    lpassen_{it}=&\theta_{t1}+\alpha_1 lfrare_{it}+\delta_1 ldist_i+\delta_2 ldist_i^2+\gamma_1(ldist_i-\mu_1)lfare_{it}\\
    &+\gamma_2(ldist_i^2-\mu_2)lfare_{it}+c_{i1}+u_{it1},
\end{align*}
where $\mu_1=\E(ldist_i)$ and $\mu_2=\E(ldist_i^2)$. The means are subtracted before forming the interactions so that $\alpha_1$ is the average partial effect. In using REIV or FEIV to estimate this model, what should be the IVs for the interaction terms?
\\ \textit{Answer:}\\
The endogenous variables are $lfare,\ (ldist-\mu_1)lfare,\ (ldist^2-\mu_2)lfare$ since $lfare$ is endogenous. If $concen$ partially correlated with $lfare$, we can use $concen,\ (ldist-\mu_1)concen,\ (ldist^2-\mu_2)concen$ as IVs.

\item Use the data in AIRFARE.RAW to estimate the model in part c, replacing $\mu_1$ and $\mu_2$ with their sample averages. How do the REIV and FEIV estimates of $\alpha_1$ compare with the estimates in Table 11.1?
\\ \textit{Answer:}\\
Create Interaction Terms
\begin{stlog}. egen avg_ldist=mean(ldist)
{\smallskip}
. gen dmldist=ldist - avg_ldist
{\smallskip}
. egen avg_ldistsq=mean(ldistsq)
{\smallskip}
. gen dmldistsq=ldistsq - avg_ldistsq
{\smallskip}
. gen ldistxlfare=dmldist*lfare
{\smallskip}
. gen ldistsqxlfare=dmldistsq*lfare
{\smallskip}
. gen ldistxconcen=dmldist*concen
{\smallskip}
. gen ldistsqxconcen=dmldistsq*concen
{\smallskip}
\end{stlog}
RE IV Estimate
\begin{stlog}. xtivreg lpassen ldist ldistsq y98 y99 y00 (lfare ldistxlfare ldistsqxlfare= ///
> concen ldistxconcen ldistsqxconcen), re
{\smallskip}
G2SLS random-effects IV regression              Number of obs     =      4,596
Group variable: id                              Number of groups  =      1,149
{\smallskip}
R-sq:                                           Obs per group:
     within  = 0.1318                                         min =          4
     between = 0.0006                                         avg =        4.0
     overall = 0.0016                                         max =          4
{\smallskip}
                                                Wald chi2(8)      =     180.64
corr(u_i, X)       = 0 (assumed)                Prob > chi2       =     0.0000
{\smallskip}
\HLI{14}{\TOPT}\HLI{64}
      lpassen {\VBAR}      Coef.   Std. Err.      z    P>|z|     [95\% Conf. Interval]
\HLI{14}{\PLUS}\HLI{64}
        lfare {\VBAR}  -1.049017   .3251395    -3.23   0.001    -1.686279    -.411755
  ldistxlfare {\VBAR}   29.65005   7.961528     3.72   0.000     14.04574    45.25436
ldistsqxlfare {\VBAR}   -2.33133   .6384701    -3.65   0.000    -3.582708   -1.079951
        ldist {\VBAR}  -157.9176   42.78275    -3.69   0.000    -241.7703   -74.06495
      ldistsq {\VBAR}   12.45567   3.439382     3.62   0.000       5.7146    19.19673
          y98 {\VBAR}   .0319535   .0105573     3.03   0.002     .0112615    .0526454
          y99 {\VBAR}   .0799954   .0127614     6.27   0.000     .0549836    .1050072
          y00 {\VBAR}   .1570266   .0265846     5.91   0.000     .1049217    .2091315
        _cons {\VBAR}   505.0837   131.5074     3.84   0.000     247.3339    762.8334
\HLI{14}{\PLUS}\HLI{64}
      sigma_u {\VBAR}  1.3686882
      sigma_e {\VBAR}  .19436268
          rho {\VBAR}  .98023276   (fraction of variance due to u_i)
\HLI{14}{\BOTT}\HLI{64}
Instrumented:   lfare ldistxlfare ldistsqxlfare
Instruments:    ldist ldistsq y98 y99 y00 concen ldistxconcen ldistsqxconcen
\HLI{79}
{\smallskip}
\end{stlog}
FE IV Estimate
\begin{stlog}. xtivreg lpassen y98 y99 y00 (lfare ldistxlfare ldistsqxlfa
> re= ///
> concen ldistxconcen ldistsqxconcen), re
{\smallskip}
G2SLS random-effects IV regression              Number of ob
> s                                                         
>       =                                                   
>              4,596
Group variable: id                              Number of gr
> oups                                                      
>       =                                                   
>              1,149
{\smallskip}
R-sq:                                           Obs per grou
> p:
     within  = 0.3172                                       
>   min                                                     
>       =                                                   
>                  4
     between = 0.0273                                       
>   avg                                                     
>       =                                                   
>                4.0
     overall = 0.0362                                       
>   max                                                     
>       =                                                   
>                  4
{\smallskip}
                                                Wald chi2(6)
>       =                                                   
>             173.26
corr(u_i, X)       = 0 (assumed)                Prob > chi2 
>       =                                                   
>             0.0000
{\smallskip}
\HLI{13}{\TOPT}\HLI{46}
\HLI{18}
     lpassen {\VBAR}      Coef.   Std. Err.      z    P>|z|     [9
> 5\% Con                                                    
>       f. Interval]
\HLI{13}{\PLUS}\HLI{46}
\HLI{18}
       lfare {\VBAR}  -.2914916   .2755414    -1.06   0.290    -.8
> 315429                                                    
>           .2485597
 ldistxlfare {\VBAR}  -.1252512   .1561083    -0.80   0.422    -.4
> 312177                                                    
>           .1807154
ldistsqxlf{\tytilde}e {\VBAR}   .0119725   .0119824     1.00   0.318    -.0
> 115124                                                    
>           .0354575
         y98 {\VBAR}   .0257759   .0099998     2.58   0.010     .0
> 061766                                                    
>           .0453752
         y99 {\VBAR}   .0724779   .0122562     5.91   0.000     .0
> 484562                                                    
>           .0964997
         y00 {\VBAR}   .1126858   .0278098     4.05   0.000     .0
> 581796                                                    
>            .167192
       _cons {\VBAR}   7.443569   1.390625     5.35   0.000     4.
> 717995                                                    
>           10.16914
\HLI{13}{\PLUS}\HLI{46}
\HLI{18}
     sigma_u {\VBAR}  .87890439
     sigma_e {\VBAR}  .19436268
         rho {\VBAR}  .95337629   (fraction of variance due to u_i
> )
\HLI{13}{\BOTT}\HLI{46}
\HLI{18}
{\lbr}p 0 16 -18{\rbr}Instrumented:   lfare ldistxlfare ldistsqxlfare{\lbr}
> p_end{\rbr}
{\lbr}p 0 16 -18{\rbr}Instruments:    y98 y99 y00 concen ldistxconcen 
> ldistsqxconcen{\lbr}p_end{\rbr}
\HLI{60}
\HLI{18}
{\smallskip}
\end{stlog}


\item Obtain fully robust standard errors for the FEIV estimation and obtain a fully robust test of joint significance of the interaction terms. (Ignore the estimation of $\mu_1$ and $\mu_2$.) What is the robust 95 percent confidence interval for $\alpha_1$?
\\ \textit{Answer:}\\
Robust FE IV Estimate using 
\begin{stlog}. xtivreg2 lpassen y98 y99 y00 (lfare ldistxlfare ldistsqxlfare= ///
> concen ldistxconcen ldistsqxconcen), fe cluster(id)
{\smallskip}
FIXED EFFECTS ESTIMATION
\HLI{24}
Number of groups =      1149                    Obs per group: min =         4
                                                               avg =       4.0
                                                               max =         4
{\smallskip}
IV (2SLS) estimation
\HLI{20}
{\smallskip}
Estimates efficient for homoskedasticity only
Statistics robust to heteroskedasticity and clustering on id
{\smallskip}
Number of clusters (id) =         1149                Number of obs =     4596
                                                      F(  6,  1148) =    14.90
                                                      Prob > F      =   0.0000
Total (centered) SS     =  128.0991685                Centered R2   =  -0.0148
Total (uncentered) SS   =  128.0991685                Uncentered R2 =  -0.0148
Residual SS             =  129.9901441                Root MSE      =    .1942
{\smallskip}
\HLI{14}{\TOPT}\HLI{64}
              {\VBAR}               Robust
      lpassen {\VBAR}      Coef.   Std. Err.      z    P>|z|     [95\% Conf. Interval]
\HLI{14}{\PLUS}\HLI{64}
        lfare {\VBAR}  -1.011863   .7124916    -1.42   0.156    -2.408321     .384595
  ldistxlfare {\VBAR}   24.11579   11.26762     2.14   0.032      2.03166    46.19993
ldistsqxlfare {\VBAR}  -1.905021   .8941964    -2.13   0.033    -3.657614   -.1524285
          y98 {\VBAR}   .0322146   .0167737     1.92   0.055    -.0006613    .0650905
          y99 {\VBAR}    .080772   .0261059     3.09   0.002     .0296053    .1319387
          y00 {\VBAR}    .155485   .0625692     2.49   0.013     .0328515    .2781184
\HLI{14}{\BOTT}\HLI{64}
Underidentification test (Kleibergen-Paap rk LM statistic):              1.913
                                                   Chi-sq(1) P-val =    0.1666
\HLI{78}
Weak identification test (Cragg-Donald Wald F statistic):                5.984
                         (Kleibergen-Paap rk Wald F statistic):          0.715
Stock-Yogo weak ID test critical values:                       <not available>
\HLI{78}
Hansen J statistic (overidentification test of all instruments):         0.000
                                                 (equation exactly identified)
\HLI{78}
Instrumented:         lfare ldistxlfare ldistsqxlfare
Included instruments: y98 y99 y00
Excluded instruments: concen ldistxconcen ldistsqxconcen
\HLI{78}
{\smallskip}
\end{stlog}
Test of Joint Significance
\begin{stlog}. test ldistxlfare ldistsqxlfare
{\smallskip}
 ( 1)  ldistxlfare = 0
 ( 2)  ldistsqxlfare = 0
{\smallskip}
           chi2(  2) =    4.59
         Prob > chi2 =    0.1008
{\smallskip}
\end{stlog}


\item Find the estimated elasticities for $dist=500$ ad $dist=1,500.$ What do you conclude?
\\ \textit{Answer:}\\
Calculate elasticity at dist=500, and 1500


\end{enumerate}


\end{document}