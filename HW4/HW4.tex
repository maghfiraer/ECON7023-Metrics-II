\documentclass[10pt]{article}
 
\usepackage[margin=1in]{geometry} 
\usepackage{amsmath,amsthm,amssymb,amsfonts, graphicx, multicol, array}
\usepackage{mathtools}
\usepackage{booktabs}
\usepackage{stata/stata}
\usepackage{wrapfig}

\graphicspath{ {images/} }

\newcommand\iid{\stackrel{\mathclap{iid}}{\sim}}
\newcommand\asym{\stackrel{\mathclap{a}}{\sim}}
\newcommand\convprob{\xrightarrow{p}}
\newcommand\convdist{\xrightarrow{d}}
\newcommand{\N}{\mathbb{N}}
\newcommand{\Z}{\mathbb{Z}}
\newcommand{\E}{\text{E}}
\newcommand{\V}{\text{Var}}
\newcommand{\Av}{\text{Avar}}
\newcommand{\se}{\text{se}}
\newcommand{\corr}{\text{Corr}}
\newcommand{\cov}{\text{Cov}}
\newcommand{\norm}{\text{Normal}}
\newcommand{\indep}{\perp \!\!\! \perp}
\newcommand{\H}{\text{H}}

 
\newenvironment{problem}[2][Problem]{\begin{trivlist}
\item[\hskip \labelsep {\bfseries #1}\hskip \labelsep {\bfseries #2.}]}{\end{trivlist}}

\begin{document}
 
\title{Homework 4}
\author{ECON 7023: Econometrics II\\
Maghfira Ramadhani\\
March 18, 2022}
\date{Spring 2023}
\maketitle

\section*{Chapter 6}
\subsection*{Problem 6.8}
The data in FERTIL1.RAW are a pooled cross section on more than a thousand U.S. women for the even years between 1972 and 1984, inclusive; the data set is similar to the one used by Sande (1992). These data can be used to study the relationship between women's education and fertility.
\begin{enumerate}
\item[a.] Use OLS to estimate a model relating number of children ever born to a woman ($kids$) to years of education, age, region, race, and type of environment reared in. You should use a quadratic in age and should include year dummies. What is the estimated relationship between fertility and education? Holding other factors fixed, has there been any notable secular change in fertility over the time period?
\\ Answer: \\

\item[b.] Reestimate the model in part a, but use $motheduc$ and $fatheduc$ as instruments for $educ$. First check that these instruments are sufficiently partially correlated with $educ$. Test whether $educ$ is in fact exogenous in the fertility equation.
\\ Answer: \\

\item[c.]Now allow the effect of education to change over time by including interaction terms such as $y74\cdot educ$, $y76\cdot educ$, and so on in the model. Use interactions of time dummies and parents’ education as instruments for the interaction terms. Test that there has been no change in the relationship between fertility and education over time.
\\ Answer: \\

\end{enumerate}

\subsection*{Problem 6.9}
Use the data in INJURY.RAW for this question. 
\begin{enumerate}
\item[a.] Using the data for Kentucky, reestimate equation (6.54) adding as explanatory variables $male$, $married$, and a full set of industry- and injury-type dummy variables. How does the estimate on $afchnge\cdot highearn$ change when these other factors are controlled for? Is the estimate still statistically significant? 
\\ Answer: \\

\item[b.] What do you make of the small R-squared from part a? Does this mean the equation is useless? 
\\ Answer: \\

\item[c.] Estimate equation (6.54) using the data for Michigan. Compare the estimate on the interaction term for Michigan and Kentucky, as well as their statistical significance.
\\ Answer: \\

\end{enumerate}


\subsection*{Problem 6.11}
The following wage equation represents the populations of working people in 1978 and 1985:
\begin{align*}
    \log(wage)=&\beta_0+ \delta_0 y85+\beta_1 educ+\delta_1 y85\cdot educ + \beta_2 exper \\
    &+\beta_3 exper^2 + \beta_4 union +\beta_5 female +\delta_5 y85\cdot female+u,
\end{align*}
where the explanatory variables are standard. The variable $union$ is a dummy variable equal to one if the person belongs to a union and zero otherwise. The variable $y85$ is a dummy variable equal to one if the observation comes from 1985 and zero if it comes from 1978. In the file CPS78\_85.RAW, there are 550 workers in the sample in 1978 and a different set of 534 people in 1985. 
\begin{enumerate}
\item[a.] Estimate this equation and test whether the return to education has changed over the seven-year period.
\\ Answer: \\

\item[b.] What has happened to the gender gap over the period? 
\\ Answer: \\

\item[c.] Wages are measured in nominal dollars. What coefficients would change if we measure $wage$ in 1978 dollars in both years? (Hint: Use the fact that for all 1985 observations, $\log(wage_i/P85)=\log(wage_i)-\log(P85)$, where $P85$ is the common deflator; $P85=1.65$ according to the Consumer Price Index.)
\\ Answer: \\

\item[d.] Is there evidence that the variance of the error has changed over time?
\\ Answer: \\

\item[e.] With wages measured nominally, and holding other factors fixed, what is the estimated increase in nominal wage for a male with 12 years of education? Propose a regression to obtain a confidence interval for this estimate. (Hint: You must replace $y85\cdot educ$ with something else.)
\\ Answer: \\
 
\end{enumerate}

\section*{Chapter 7}
\subsection*{Problem 7.2}
In model (7.11), maintain Assumptions SOLS.1 and SOLS.2, and assume $\E(\textbf{X}_i'\textbf{u}_i\textbf{u}_i'\textbf{X}_i)=\E(\textbf{X}_i'\pmb{\Omega}\textbf{X}_i)$, where $\pmb{\Omega}\equiv \E(\textbf{u}_i\textbf{u}_i').$ (The last assumption is a different way of stating the homoskedasticity assumption for systems of equations; it always holds if assumption (7.53) holds.) Let $\hat{\pmb{\beta}}_{SOLS}$ denote the system OLS estimator.
\begin{enumerate}
\item[a.] Show that $\Av(\hat{\pmb{beta}}_{SOLS})=[\E(\textbf{X}_i'\textbf{X}_i)]^{-1}[\E(\textbf{X}_i'\pmb{\Omega}\textbf{X}_i)][\E(\textbf{X}_i'\textbf{X}_i)]^{-1}/N.$
\\ Answer:\\

\item[b.] How would you estimate the asymptotic variance in part a?
\\ Answer:\\

\item[c.] Now add Assumptions SGLS.1-SGLS.3. Show that $\Av(\hat{\pmb{\beta}}_{SOLS})-\Av(\hat{\pmb{\beta}}_{FGLS})$ is positive semidefinite. (Hint: Show that $[\Av(\hat{\pmb{\beta}}_{FGLS})]^{-1}-[\Av(\hat{\pmb{\beta}}_{SOLS})]^{-1}$ is p.s.d.)
\\ Answer:\\

\item[d.] If, in addition to the previous assumptions, $\pmb{\Omega}=\sigma^2 \textbf{I}_G$, show that SOLS and FGLS have the same asymptotic variance.
\\ Answer:\\

\item[e.] Evaluate the following statement: ``Under the assumption of part c, FGLS is never asymptotically worse that SOLS, even if $\pmb{\Omega}=\sigma^2 \textbf{I}_G$."
\\ Answer:\\

\end{enumerate}

\subsection*{Problem 7.7}
Consider the panel data model
\begin{align*}
    &y_{it}=\textbf{x}_{it}\pmb{\beta}+u_{it},  &t=1,2,\ldots,T, \\
    &\E(u_{it}|\textbf{x}_{it},u_{i,t-1},\textbf{x}_{i,t-1},\ldots,)=0, &\\
    &\E(u_{it}^2|\textbf{x}_{it})=\E(u_{it}^2)=\sigma_t^2, &t=1,2,\ldots,T.
\end{align*}
(Note that $\E(u_{it}^2|\textbf{x}_{it})$ does not depend on $\textbf{x}_{it}$, but it is allowed to be a different constant in each time period.)
\begin{enumerate}
\item[a.] Show that $\pmb{\Omega}= \E(\textbf{u}_i\textbf{u}_i')$ is a diagonal matrix.
\\ Answer: \\

\item[b.] Write down the GLS estimator assuming that $\pmb{\Omega}$ is know.
\\ Answer: \\

\item[c.] Argue that Assumption SGLS.1 does not necessarily hold under the assumptions made. (Setting $\textbf{x}_{it}=y_{i,t-1}$ might help in answering this part.) Nevertheless, show that the GLS estimator from part b $is$ consistent for $\pmb{\beta}$ by showing that $\E(\textbf{X}_i'\pmb{\Omega}^{-1}\textbf{u}_i)=0.$ (This proof shows that Assumption SGLS.1 is sufficient, but not necessary, for consistency. Sometimes $\E(\textbf{X}_i'\pmb{\Omega}^{-1}\textbf{u}_i)=0$ even though Assumption SGLS.1 does not hold.)
\\ Answer: \\

\item[d.] Show that Assumptions SGLS.3 holds under the given assumptions.
\\ Answer: \\

\item[e.] Explain how to consistently estimate each $\sigma_t^2$ (as $N\to \infty$).
\\ Answer: \\

\item[f.] Argue that, under the assumptions made, valid inference is obtained by weighting each observation $(y_{it},\textbf{x}_{it})$ by $1/\hat{\sigma}_t$ and then running pooled OLS.
\\ Answer: \\

\item[g.] What happen if we assume that $\sigma_t^2=\sigma^2$ for all $t=1,\ldots,T$?
\\ Answer: \\

\end{enumerate}

\section*{Chapter 8}
\subsection*{Problem 8.1}
\begin{enumerate}
\item[a.] Show that GMM estimator that solves the problem (8.27) satisfies the first order-condition
\[\left(\sum_{i=1}^N \textbf{Z}_i'\textbf{X}_i\right)'\hat{\textbf{W}}\left(\sum_{i=1}^N \textbf{Z}_i'(\textbf{y}_i-\textbf{X}_i \hat{\pmb{\beta}})\right)=0\]
\\ Answer: \\

\item[b.] Use this expression to obtain (8.28)
\\ Answer: \\

\end{enumerate}
\subsection*{Problem 8.5}
Verify that the difference $(\textbf{C}'\pmb{\Lambda}^{-1}\textbf{C})-(\textbf{C}'\textbf{W}\textbf{C})(\textbf{C}'\textbf{W}\pmb{\Lambda}\textbf{W}\textbf{C})\textbf{C})^{-1}(\textbf{C}'\textbf{W}\textbf{C})$ in expression (8.34) is positive semidefinite for any symmetric positive definite matrices $\textbf{W}$ and $\pmb{\Lambda}$. (Hint: Show that the difference can be expressed as $\textbf{C}'\pmb{\Lambda}^{-1/2}[\textbf{I}_L-\textbf{D}(\textbf{D}'\textbf{D})^{-1}\textbf{D}']\pmb{\Lambda}^{-1/2}\textbf{C}$ where $D\equiv\pmb{\Lambda}^{1/2}\textbf{WC}.$ Then, note that for any $L\times K$ matrix $\textbf{D},\textbf{I}_L-\textbf{D}(\textbf{D}'\textbf{D})^{-1}\textbf{D}'$ is a symmetric, idempotent matrix, and therefore positive semidefinite.)

\section*{Chapter 9}
\subsection*{Problem 9.8}
\begin{enumerate}
\item[a.] Extend Problem 5.4b using CARD.RAW to allow $educ^2$ to appear in the $\log(wage)$ equation, without using $nearc2$ as an instrument. Specifically, use interactions of $nearc4$ with some or all of the other exogenous variables in the $\log(wage)$ equation as instruments for $educ^22$. Compute a heteroskedasticity-robust test to be sure that at least one of these additional instruments appears in the linear projection of $educ^22$ onto your entire list of instruments. Test whether $educ^22$ needs to be in the $\log(wage)$ equation. 
\\ Answer:\\

\item[b.] Start again with the model estimated in Problem 5.4b, but suppose we add the interaction $black\cdot educ$. Explain why $black\cdot z_j$ is a potential IV for $black\cdot educ$, where $z_j$ is any exogenous variable in the system (including $nearc4$). 
\\ Answer:\\

\item[c.] In Example 6.2 we used $black\cdot nearc4$ as the IV for $black\cdot educ$. Now use 2SLS with $black\cdot\hat{educ}$ as the IV for $black\cdot educ$, where $\hat{educ}$ are the fitted values from the first-stage regression of $educ$ on all exogenous variables (including $nearc4$). What do you find? 
\\ Answer:\\

\item[d.] If $\E(educ|\textbf{z})$ is linear and $\V(u_2|\textbf{z})=\sigma_1^2$, where \textbf{z} is the set of all exogenous variables and $u_1$ is the error in the $\log(wage)$ equation, explain why the estimator using $black\cdot\hat{educ}$ as the IV is asymptotically more efficient than the estimator using $black\cdot nearc4$ as the IV.
\\ Answer:\\

\end{enumerate}

\end{document}